\documentclass[screen, aspectratio=43]{beamer}
\usepackage[T1]{fontenc}
\usepackage[utf8]{inputenc}

\usepackage{stmaryrd}
\usepackage{tikz}

% Use the NTNU-temaet for beamer 
% \usetheme[style=ntnu|simple|vertical|horizontal, 
%     language=bm|nn|en, 
%     smalltitle, 
%     city=all|trondheim|alesund|gjovik]{ntnu2017}
\usetheme[style=ntnu,language=en, city=trondheim]{ntnu2017}

\usepackage[english]{babel}
\usepackage[style=numeric,backend=biber,natbib=false,sorting=none]{biblatex}

\title{McKay correspondence}
\author[J. F. Grevstad]{Jacob Fjeld Grevstad}
\institute[NTNU]{Department of Mathematical sciences, NTNU}
\date{31 May 2019}
%\date{} % To have an empty date

\addbibresource{example.bib} % Add bibliography database

% Set the reference style to numeric.
% See here: http://tex.stackexchange.com/questions/68080/beamer-bibliography-icon
\setbeamertemplate{bibliography item}[text] 

% Set bibliography fonts to a small size.
\renewcommand*{\bibfont}{\footnotesize}

%\newtheorem{theorem}{Theorem}[section]
\newtheorem{prop}[theorem]{Proposition}
%\newtheorem{lemma}[theorem]{Lemma}
\theoremstyle{definition}
\newtheorem{defin}[theorem]{Definition}
%\newtheorem{example}[theorem]{Example}

\newcommand{\C}{\mathbb{C}}
\newcommand{\Z}{\mathbb{Z}}
\DeclareMathOperator{\Hom}{Hom}
\DeclareMathOperator{\Ext}{Ext}
\DeclareMathOperator{\Tor}{Tor}
\DeclareMathOperator{\End}{End}
\DeclareMathOperator{\Aut}{Aut}
\DeclareMathOperator{\Image}{Im}
\DeclareMathOperator{\Ker}{Ker}
\DeclareMathOperator{\Cok}{Cok}
\DeclareMathOperator{\depth}{depth}
\DeclareMathOperator{\height}{height}

\begin{document}

\begin{frame}
  \titlepage
\end{frame}

% Alternatively, special title page command to get a different background
% \ntnutitlepage

\begin{frame}
	\frametitle{Overview}
	\begin{itemize}
		\item Statement of the McKay correspondance
		\item History and motivation
		\item examples and proofs
	\end{itemize}
\end{frame}

\begin{frame}
	\frametitle{McKay correspondance}
	The theorem involves $G \leq SL_2(\C)$, $S= \C \llbracket x, y \rrbracket$, and $R = S^G$. It establishes a correspondence between:
	\begin{itemize}
		\item The irreducible representations of $G$; 
		\item The indecomposable projective modules of the skew group algebra $S\#G$;
		\item The indecomposable projective modules of the endomorphism ring $\End_R(S)$;
		\item The indecomposable MCM modules of $R$.
	\end{itemize}
\end{frame}

\begin{frame}
	\frametitle{History and motivation}
	\begin{itemize}
		\item Kleinian singularities, $\C^2 / G$
		%Felix Klein 1884
		\item Resolution graphs ADE Dynkin diagrams 
		\item McKay connected the geometry to representation theory
		%McKay 1983
		\item Herzog showed that the MCM $R$-modules are direct summands of $S$
		\item Auslander showed the relationship between the projective $S\#G$-modules and the MCM $R$-modules
	\end{itemize}
\end{frame}

\begin{frame}
	\frametitle{The McKay quiver}
	\begin{defin}
		The McKay quiver of $G$ has verticies the irreducible representations of $G$, and an arrow from $W$ to $W'$ iff $W'$ appears as a direct summand of $V \otimes_\C W$, where $V$ is the cannonical representation.
	\end{defin}
	\begin{example}
		$G = \langle g \rangle \cong \Z/5\Z$, $g = \begin{pmatrix}
		\omega^2 & 0\\
		0 & \omega^3
		\end{pmatrix}$, $\omega = \exp(2\pi i/5)$,	
		\begin{tikzpicture}[scale=1, baseline={([yshift=-.8ex]current bounding box.center)}]
			\node at (0,1) (v0) {$V_0$};
			\node at (0.951056516,0.309016994) (v1) {$V_1$};
			\node at (0.587785252,-0.809016994) (v2) {$V_2$};
			\node at (-0.587785252,-0.809016994) (v3) {$V_3$};
			\node at (-0.951056516,0.309016994) (v4) {$V_4$};
			\draw[<->] (v0) edge (v3) (v3) edge (v1) (v1) edge (v4) (v4) edge (v2) (v2) edge (v0);
		\end{tikzpicture}
	\end{example}
\end{frame}

\begin{frame}
	\frametitle{Non-Dynkin example}
	\begin{example}
		$G = \langle \mu, \rho \rangle \cong S_3$, $\mu = \begin{pmatrix}
		0 & 1\\
		1 & 0
		\end{pmatrix}$, $\rho = \begin{pmatrix}
		\exp(2\pi i/3) & 0\\
		0 & \exp(-2\pi i/3)
		\end{pmatrix}$
		\begin{center}
		\begin{tikzpicture}[scale=1.5]%, baseline={([yshift=-.8ex]current bounding box.center)}]
			\node at (-1,0) (v0) {$V_0$};
			\node at (0,0) (v) {$V$};
			\node at (1,0) (vm) {$V_\mu$};
			\draw[<->] (v0) -- (v);
			\draw[<->] (v) -- (vm);
			\path[->] (v) edge  [loop above] (v);
		\end{tikzpicture}
		\end{center}
	\end{example}
\end{frame}

\begin{frame}
	\frametitle{The skew group algebra}
	\begin{defin}
		If $S$ is an algebra and $G$ is a subgroup of $\Aut(S)$, then the \textit{skew group algebra} $S\#G$ is the algebra generated by $s \cdot g$ with $s \in S$ and $g \in G$. The multiplication is given by $$(s \cdot g)(t \cdot h) = s t^g \cdot gh$$
	\end{defin}
	\begin{example}
		Let $S= \C \llbracket x, y \rrbracket$, and $G = \left\{ e:= \begin{pmatrix}
		1 & 0\\
		0 & 1
		\end{pmatrix}, g:= \begin{pmatrix}
		0 & 1\\
		-1 & 0
		\end{pmatrix} \right\}$. \\
		Then $x \cdot e + y\cdot g$ and $(x-y) \cdot g$ are in $S\#G$, and their product is $$(x \cdot e + y\cdot g)((x-y) \cdot g) = x(x-y) \cdot g - y(x+y) \cdot e$$
	\end{example}
\end{frame}

\begin{frame}
	\frametitle{Gabriel quiver}
\end{frame}

\begin{frame}
  \frametitle{Main theorem}
  \begin{defin}
    {\LaTeX} makes things easier.
  \end{defin}
  \begin{theorem}
  	This is a theorem 
  \begin{proof}
    For details, see Flynn~\cite{latex}.
  \end{proof}
  \end{theorem}
\end{frame}

\begin{frame}
  \frametitle{References}
  \printbibliography
\end{frame}

\end{document}
