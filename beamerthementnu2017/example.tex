\documentclass[screen, aspectratio=43]{beamer}
\usepackage[T1]{fontenc}
\usepackage[utf8]{inputenc}

\usepackage{stmaryrd}
\usepackage{tikz}
\usepackage{tikz-cd}

% Use the NTNU-temaet for beamer 
% \usetheme[style=ntnu|simple|vertical|horizontal, 
%     language=bm|nn|en, 
%     smalltitle, 
%     city=all|trondheim|alesund|gjovik]{ntnu2017}
\usetheme[style=ntnu,language=en, city=trondheim]{ntnu2017}

\usepackage[english]{babel}
\usepackage[style=numeric,backend=biber,natbib=false,sorting=none]{biblatex}

\title{McKay correspondence}
\author[J. F. Grevstad]{Jacob Fjeld Grevstad}
\institute[NTNU]{Department of Mathematical sciences, NTNU}
\date{31 May 2019}
%\date{} % To have an empty date

\addbibresource{example.bib} % Add bibliography database

% Set the reference style to numeric.
% See here: http://tex.stackexchange.com/questions/68080/beamer-bibliography-icon
\setbeamertemplate{bibliography item}[text] 

% Set bibliography fonts to a small size.
\renewcommand*{\bibfont}{\footnotesize}

%\newtheorem{theorem}{Theorem}[section]
\newtheorem{prop}[theorem]{Proposition}
%\newtheorem{lemma}[theorem]{Lemma}
\theoremstyle{definition}
\newtheorem{defin}[theorem]{Definition}
%\newtheorem{example}[theorem]{Example}

\newcommand{\C}{\mathbb{C}}
\newcommand{\Z}{\mathbb{Z}}
\DeclareMathOperator{\Hom}{Hom}
\DeclareMathOperator{\Ext}{Ext}
\DeclareMathOperator{\Tor}{Tor}
\DeclareMathOperator{\End}{End}
\DeclareMathOperator{\Aut}{Aut}
\DeclareMathOperator{\Image}{Im}
\DeclareMathOperator{\Ker}{Ker}
\DeclareMathOperator{\Cok}{Cok}
\DeclareMathOperator{\depth}{depth}
\DeclareMathOperator{\height}{height}

\begin{document}

\begin{frame}
  \titlepage
\end{frame}

% Alternatively, special title page command to get a different background
% \ntnutitlepage

\begin{frame}
	\frametitle{Overview}
	\begin{itemize}
		\item Statement of the McKay correspondance
		\item History and motivation
		\item examples and proofs
	\end{itemize}
\end{frame}

\begin{frame}
	\frametitle{McKay correspondance}
	The theorem involves $G \leq SL_2(\C)$ finite group, $S= \C \llbracket x, y \rrbracket$, and $R = S^G$. It establishes a correspondence between:
	\begin{itemize}
		\item The irreducible representations of $G$; 
		\item The indecomposable projective modules of the skew group algebra $S\#G$;
		\item The indecomposable projective modules of the endomorphism ring $\End_R(S)$;
		\item The indecomposable MCM modules of $R$.
	\end{itemize}
\end{frame}

\begin{frame}
	\frametitle{History and motivation}
	\begin{itemize}
		\item Kleinian singularities, $\C^2 / G$
		%Felix Klein 1884
		\item Resolution graphs ADE Dynkin diagrams 
		\item McKay connected the geometry to representation theory
		%McKay 1983
		\item Herzog showed that the MCM $R$-modules are direct summands of $S$
		\item Auslander showed the relationship between the projective $S\#G$-modules and the MCM $R$-modules
	\end{itemize}
\end{frame}

\begin{frame}
	\frametitle{The McKay quiver}
	\begin{defin}
		The McKay quiver of $G$ has verticies the irreducible representations of $G$, and an arrow from $W$ to $W'$ iff $W'$ appears as a direct summand of $V \otimes_\C W$, where $V$ is the cannonical representation.
	\end{defin}
	\begin{example}
		$G = \langle g \rangle \cong \Z/5\Z$, $g = \begin{pmatrix}
		\omega^2 & 0\\
		0 & \omega^3
		\end{pmatrix}$, $\omega = \exp(2\pi i/5)$,	
		\begin{tikzpicture}[scale=1, baseline={([yshift=-.8ex]current bounding box.center)}]
			\node at (0,1) (v0) {$V_0$};
			\node at (0.951056516,0.309016994) (v1) {$V_1$};
			\node at (0.587785252,-0.809016994) (v2) {$V_2$};
			\node at (-0.587785252,-0.809016994) (v3) {$V_3$};
			\node at (-0.951056516,0.309016994) (v4) {$V_4$};
			\draw[<->] (v0) edge (v3) (v3) edge (v1) (v1) edge (v4) (v4) edge (v2) (v2) edge (v0);
		\end{tikzpicture}
	\end{example}
\end{frame}

\begin{frame}
	\frametitle{Non-Dynkin example}
	\begin{example}
		$G = \langle \mu, \rho \rangle \cong S_3$, $\mu = \begin{pmatrix}
		0 & 1\\
		1 & 0
		\end{pmatrix}$, $\rho = \begin{pmatrix}
		\exp(2\pi i/3) & 0\\
		0 & \exp(-2\pi i/3)
		\end{pmatrix}$
		\begin{center}
		\begin{tikzpicture}[scale=1.5]%, baseline={([yshift=-.8ex]current bounding box.center)}]
			\node at (-1,0) (v0) {$V_0$};
			\node at (0,0) (v) {$V$};
			\node at (1,0) (vm) {$V_\mu$};
			\draw[<->] (v0) -- (v);
			\draw[<->] (v) -- (vm);
			\path[->] (v) edge  [loop above] (v);
		\end{tikzpicture}
		\end{center}
	\end{example}
\end{frame}

\begin{frame}
	\frametitle{The skew group algebra}
	\begin{defin}
		If $S$ is an algebra and $G$ is a subgroup of $\Aut(S)$, then the \textit{skew group algebra} $S\#G$ is the algebra generated by $s \cdot g$ with $s \in S$ and $g \in G$. The multiplication is given by $$(s \cdot g)(t \cdot h) = s t^g \cdot gh$$
	\end{defin}
	\begin{example}
		Let $S= \C \llbracket x, y \rrbracket$, and $G = \left\{ e:= \begin{pmatrix}
		1 & 0\\
		0 & 1
		\end{pmatrix}, g:= \begin{pmatrix}
		0 & 1\\
		1 & 0
		\end{pmatrix} \right\}$. \\
		Then $x \cdot e + y\cdot g$ and $(x-y) \cdot g$ are in $S\#G$, and their product is $$(x \cdot e + y\cdot g)((x-y) \cdot g) = x(x-y) \cdot g + y(y-x) \cdot e$$
	\end{example}
\end{frame}

\begin{frame}
	\frametitle{Gabriel quiver}
	\begin{defin}
		The \textit{Gabriel quiver} of an algebra, $A$ (where projective covers exist) has verticies the simple $A$-modules. If $M$ and $N$ are simple $A$-modules with minimal projective resolutions:
		\begin{center}
		$
		\cdots \to P^M_1 \to P^M_0 \to 0
		$
		\hspace{1cm}
		and
		\hspace{1cm}
		$
		\cdots \to P^N_1 \to P^N_0 \to 0
		$.
		\end{center}
		Then there is an arrow from $M$ to $N$ iff $P^N_0$ is a direct summand of $P^M_1$.
	\end{defin}
\end{frame}

\begin{frame}[fragile]
	\frametitle{Example of Gabriel quiver}
	\begin{example}
		\begin{center}
		$S = \C\llbracket x, y \rrbracket$, $G= \langle g \rangle$, $g = \begin{pmatrix}
		0 & 1\\
		1 & 0
		\end{pmatrix}$, $\mathfrak{m} = \langle x, y \rangle_{S\#G}$
		\end{center}
		$$ S\#G \cong S \otimes_\C \langle I + g \rangle \oplus S \otimes_\C \langle I - g \rangle $$
		%$$\mathfrak{m} \otimes_\C \langle I + g \rangle \cong S(x+y) \otimes_\C \langle I + g \rangle  \oplus S(x-y) \otimes_\C \langle I + g \rangle $$
		%$$ \mathfrak{m} \otimes_\C \langle I - g \rangle \cong S(x+y) \otimes_\C \langle I - g \rangle  \oplus S(x-y) \otimes_\C \langle I - g \rangle $$
		\begin{center}
		\begin{tikzcd}
		\langle I + g \rangle \ar[r, <->] \ar[loop left] & \langle I - g \rangle \ar[loop right]
		\end{tikzcd}
		\end{center}
	\end{example}
\end{frame}

\begin{frame}[fragile]
	\frametitle{The indecomposable projective $S\#G$-modules}
	\begin{theorem}
		There's a correspondance between the irreducible $G$ representations and the indecomposable projective $S\#G$-modules given by
		\begin{center}
		\begin{tikzcd}
		\left\lbrace \begin{matrix}
		\text{indecomposable projective}\\
		S\#G\text{-modules}
		\end{matrix} \right\rbrace \ar[r, <->]
		& 
		\left\lbrace \begin{matrix}
		\text{irreducible }\\
		\C G\text{ representations}
		\end{matrix} \right\rbrace\\
		\mathcal{F}: P \arrow[mapsto]{r} & P/\mathfrak{m}P\\
		\mathcal{G}: S \otimes_\C W & W \arrow[mapsto]{l}
		\end{tikzcd}
		\end{center}
	\end{theorem}
\end{frame}

\begin{frame}[fragile]
	\frametitle{McKay and Gabriel quiver}
	\begin{theorem}
		The McKay quiver of $G$ and the Gabriel quiver of $S\#G$ are isomorphic when $G$ is finite subgroup of $GL_n(\C)$.
		\begin{proof}
		Let $V = \mathfrak{m}/\mathfrak{m}^2$. Then the minimal projective resolution of $\C = S / \mathfrak{m}$ is
		\begin{center}
		\begin{tikzcd}
		0  \arrow{r} & S \otimes_\C \bigwedge\limits^n V  \arrow{r}{\partial_n} & \cdots \arrow{r}{\partial_2} & S \otimes_\C \bigwedge\limits^{1} V \arrow{r}{\partial_1} & S \arrow{r} & 0.
		\end{tikzcd}
		\end{center}
		Applying $- \otimes_\C W$ we get
		\begin{center}
		\begin{tikzcd}
		\cdots \ar[r] & S \otimes_\C V \otimes W \ar[r] & S \otimes_\C W \ar[r] & 0.
		\end{tikzcd}
		\end{center}
		\end{proof}
	\end{theorem}
\end{frame}

\begin{frame}
  %\frametitle{Questions?}
	\begin{center}  
	\vspace{2cm}
    \begingroup
      \fontsize{40pt}{12pt}\selectfont
      Questions? 
    \endgroup
  	\end{center}	
\end{frame}

%\begin{frame}
%  \frametitle{References}
%  \printbibliography
%\end{frame}

\end{document}
