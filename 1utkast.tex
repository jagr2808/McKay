\documentclass[11pt, a4paper, norsk]{article}
\usepackage[utf8]{inputenc}
\usepackage{babel, amsmath, amsthm, amssymb, amsfonts, enumitem, mathtools, centernot}
\usepackage{tikz-cd}
\usepackage{stmaryrd}
\newcommand\tab[1][1cm]{\hspace*{1}}
\DeclarePairedDelimiter{\ceil}{\lceil}{\rceil}

\newtheorem*{prop}{Proposition}
\newtheorem*{lemma}{Lemma}
\newtheorem*{theorem}{Theorem}
\newtheorem*{defin}{Definition}

\newcommand{\C}{\mathbb{C}}

\begin{document}
\title{McKay correspondence}
\author{Jacob Fjeld Grevstad}
\maketitle

\section*{Finite subgroups of SL(2,C)}

\section*{characters and irreducible representations}
Recall that the trace of a matrix is defined to be the sum of its diagonal element and that the trace satisfies two important equations. Namely
$$tr(A+B)=tr(A)+tr(B) \text{  and  } tr(AB)=tr(BA)$$

For a given representation of $G$, $\rho: G \to GL_n(\mathbb{C})$ we define its characther by $\chi_\rho : G \to \mathbb{C}$, $\chi_\rho(g) = tr(\rho(g))$.

\begin{prop}
Conjugate elements in $G$ take the same value under a character.
\begin{proof}
Let $g$ and $g'$ be in the same conjugacy class. Then there exists an element $h$ such that $h^{-1}gh=g'$. Then we have
\begin{equation*}
\begin{split}
\chi(g')=\chi(h^{-1}gh) = tr(\rho(h)^{-1}\rho(g)\rho(h)) \stackrel{\mathclap{\normalfont\mbox{\tiny{*}}}}{=} tr(\rho(g)\rho(h)\rho(h)^{-1}) = tr(\rho(g))=\chi(g)
\end{split}
\end{equation*}
In (*) we use the fact that $tr(AB)=tr(BA)$.
\end{proof}
\end{prop}

\begin{prop}
The dimension of the representation with character $\chi$ is $\chi(1)$.
\begin{proof}
$$\chi(1) = tr\left(\begin{bmatrix}
1 & \cdots & 0\\
\vdots & \ddots & \vdots\\
0 & \cdots & 1
\end{bmatrix}\right) = n$$
\end{proof}
\end{prop}

\begin{lemma}
For a finite abelian group $G$ any irreducible representation must be 1-dimensional.
\begin{proof}
Let $\rho: G \to GL(V)$ be an irreducible representation. Since $G$ is abelian we have that $\rho(g)\rho(h)v = \rho(h)\rho(g)v$, and thus $\rho(g)$ is a homomorphism of $G$-representations. Then by Schur's lemma $\rho(g)$ must be a scalar multiplication. This implies that $\rho$ can be written as a direct sum of 1-dimensional representations, but since $\rho$ is irreducible $\rho$ must be 1-dimensional.
\end{proof}
\end{lemma}

\section*{The McKay quiver}

\section*{Skew algebra S\#G indecomposable projectives}
\begin{defin}
If $G$ is a subgroup of $GL_n(\C)$, we can extend the group action of $G$ to $\C\llbracket x_1, \cdots, x_n\rrbracket$. We then define the \underline{skew algebra $\C \llbracket x_1, \cdots, x_n \rrbracket \# G$} to be the algebra generated by elements of the form $f \cdot g$ with $f \in \C\llbracket x_1, \cdots, x_n\rrbracket$ and $g \in G$, and we define the multiplication by
$$ (f_1 \cdot g_1) \cdot (f_2 \cdot g_2) = (f_1 \cdot f_2^{g_1}) \cdot (g_1 \cdot g_2) $$
Where $f^g$ denotes the image of $f$ under the action of $g$.
\end{defin}

\begin{theorem}
We have an isomorphism of rings
$$ e \C\llbracket x, y \rrbracket \# G e \simeq \C\llbracket x, y \rrbracket^G $$
where $e = \frac{1}{|G|} \sum_{g \in G} g$.

\begin{proof}
Let $f^g$ denote the image of $f$ under the action of $g$. Then if we let $f(x,y)g$ be an element of the skew algebra we get that $e f(x,y)g e = f(x, y)^e \cdot ege = f(x, y)^e \cdot e$. It then follows that $  e \C\llbracket x, y\rrbracket \# G e$ is isomorphic to the image of $e$. Since $ge=g$ for all $g\in G$ it is clear that the image of $e$ is contained in the fixed ring. For the converse you just need to notice that the fixed ring is fixed under $e$ and thus is contained in the image.
\end{proof}
\end{theorem}

\section*{$S^G$-direct summands of $End_{S^G}(S)$}

\end{document}