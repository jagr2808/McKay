\documentclass[11pt, a4paper, english]{article}
\usepackage[utf8]{inputenc}
\usepackage{babel, amsmath, amsthm, amssymb, amsfonts, enumitem, mathtools, centernot}
\usepackage{tikz-cd}
\usepackage{stmaryrd}
\usepackage{cite}
\usepackage[toc,page]{appendix}
\newcommand\tab[1][1cm]{\hspace*{1}}
\DeclarePairedDelimiter{\ceil}{\lceil}{\rceil}

\newtheorem{prop}{Proposition}
\numberwithin{prop}{section}
\newtheorem{lemma}{Lemma}
\numberwithin{lemma}{section}
\newtheorem{theorem}{Theorem}
\numberwithin{theorem}{section}
\newtheorem{defin}{Definition}
\numberwithin{defin}{section}
\newtheorem{example}{Example}
\numberwithin{example}{section}

\newcommand{\C}{\mathbb{C}}
\DeclareMathOperator{\Hom}{Hom}
\DeclareMathOperator{\Ext}{Ext}
\DeclareMathOperator{\End}{End}
\DeclareMathOperator{\Image}{Im}
\DeclareMathOperator{\Ker}{Ker}
\DeclareMathOperator{\Cok}{Cok}

\begin{document}
\title{McKay correspondence}
\author{Jacob Fjeld Grevstad}
\maketitle

\begin{abstract} %Be less vague on GLn or drop it entirely.
The goal of this thesis is to establish a 1-1 correspondence between quivers created from the four following sets whenever $G$ is a finite subgroup of $SL(2,\C)$ and $S$ is the power series ring $\C \llbracket x, y \rrbracket$
\begin{itemize}
\item The Maximal Cohen-Macaulay modules of the fixed ring $S^G$.
\item The indecomposable projective modules of the skew group algerba $S\#G$.
\item The indecomposable projective modules of $\End_{S^G}(S)$.
\item The irreducible representations of $G$ (indecomposable $\C G$-modules).
\end{itemize}
Much of the thesis will be used to define these four quivers and to develope tools to establish such a correspondence. A similar correspondence can be established for a general field $k$ and a finite subgroup of $GL(n, k)$ with order nonzero in $k$, but the case for $SL(2,\C)$ is the most interesting as the quivers will be extended Dynkin diagrams.
\end{abstract}

\tableofcontents

\section{Finite subgroups of SL(2,C)}

\section{Characters and irreducible representations}
This section is largely based on the book by \cite{RCG}.

Recall that the trace of a matrix is defined to be the sum of its diagonal elements and that the trace satisfies two important equations. Namely
$$tr(A+B)=tr(A)+tr(B) \text{  and  } tr(AB)=tr(BA)$$
For a given representation of $G$, $\rho: G \to GL_n(\mathbb{C})$, we define its character by $\chi_\rho : G \to \mathbb{C}$, $\chi_\rho(g) = tr(\rho(g))$.

\begin{prop}
Conjugate elements in $G$ take the same value under a character.
\begin{proof}
Let $g$ and $g'$ be in the same conjugacy class. Then there exists an element $h$ such that $h^{-1}gh=g'$. Then we have
\begin{equation*}
\begin{split}
\chi(g')=\chi(h^{-1}gh) = tr(\rho(h)^{-1}\rho(g)\rho(h)) \stackrel{\mathclap{\normalfont\mbox{\tiny{*}}}}{=} tr(\rho(g)\rho(h)\rho(h)^{-1}) = tr(\rho(g))=\chi(g)
\end{split}
\end{equation*}
In (*) we use the fact that $tr(AB)=tr(BA)$.
\end{proof}
\end{prop}

\begin{lemma}
For a finite abelian group $G$ any irreducible representation must be 1-dimensional.
\begin{proof}
Let $\rho: G \to GL(V)$ be an irreducible representation. Since $G$ is abelian we have that $\rho(g)\rho(h)v = \rho(h)\rho(g)v$. Thus multiplication by $\rho(g)$ respects the action of $G$ and we have that $\rho(g)$ is a homomorphism of $G$-representations between $\rho$ and itself. Then by Schur's lemma\footnote{Statement and proof of Schur's lemma can be found in the appendix \ref{schur}} $\rho(g)$ must be a scalar multiplication. In other words every matrix $\rho(g)$ for $g \in G$ is diagonal (it is a scaling of identity). This implies that $\rho$ can be written as a direct sum of 1-dimensional representations, but since $\rho$ is irreducible $\rho$ must be 1-dimensional.
\end{proof}
\end{lemma}

\begin{prop}
If $\chi$ is the character of a representation, $\rho$, with dimension $n$ of a group $G$, and $g$ is an element of $G$ with order $m$, then the following holds
\begin{itemize}
 \item[(1)] $\chi(1) = n$
 \item[(2)] $\chi(g)$ is the sum of $m$-th roots of unity.
 \item[(3)] $\chi(g^{-1}) = \overline{\chi(g)}$
\end{itemize}
\begin{proof}

\begin{itemize}
\item[]
\item[(1)]
The first result is immidiate.
$$\chi(1) = tr\left(\begin{bmatrix}
1 & \cdots & 0\\
\vdots & \ddots & \vdots\\
0 & \cdots & 1
\end{bmatrix}\right) = n$$
\item[(2)]
Since $\langle g \rangle$ is an abelian group, $\rho$ decomposes into $n$ 1-dimensional $\langle g \rangle$-representations. Then there is a basis such that $\rho(g)$ is diagonal. Since $g$ has order $m$ it follows that the diagonal entries of $\rho(g)$ must be $m$-th roots of unity. Thus $\chi(g) = tr(\rho(g))$ must be the sum of $m$-th roots of unity.
\item[(3)]
Using the same basis as above and the fact that $\omega^{-1} = \overline{\omega}$ when $\omega$ is a root of unity we see that $\chi(g^{-1}) = tr(\rho(g)^{-1}) = \overline{tr(\rho(g))} = \overline{\chi(g)}$.
\end{itemize}
\end{proof}
\end{prop}

\section{The McKay quiver}
\begin{defin}
Let $G$ be a finite subgroup of $GL(n, \C)$, and let $V$ be the cannonical representation (the one that sends $g$ to $g$). Then we define the \underline{McKay quiver} of $G$ to be the quiver with verticies the irreducible representations of $G$, denoted $V_i$. For two irreducible representations $V_i$ and $V_j$ we say there is an arrow from the former to the latter if and only if $V_j$ is a direct summand of $V \otimes V_i$. 
\end{defin}

\begin{example}
Let $G$ be the group generated by $g =\begin{bmatrix}
\omega^2 & 0\\
0 & \omega^{3}
\end{bmatrix}$, where $\omega$ is the primitive fifth root of unity. Then there are five different irreducible representations, the one sending $g$ to $\omega$, $\omega^2$, $\omega^3$, $\omega^4$ respectively, and the trivial representation. Denote the representation sending $g$ to $\omega^i$ by $V_i$, and let $V = V_2 \oplus V_3$ be the cannonical representation. Note that $V_i \otimes V_j = V_{i+j}$, where $i+j$ is understood to be modulo 5. Then we get the following McKay-quiver
$$
\begin{tikzcd}
& V_0 \arrow[<->]{ddr} \arrow[<->]{ddl} & \\
V_4 && V_1 \arrow[<->]{ll}\arrow[<->]{dll}\\
V_3 && V_2 \arrow[<->]{ull}
\end{tikzcd}  
$$
\end{example}

\section{Krull-Remack-Schmidt}
This section is largely based on the book by \cite{CMR}.
Here we will prove the Krull-Remack-Schmidt theorem for complete local noetherian rings.

We say a ring satisfies Krull-Remack-Schmidt if the following condition holds:
\begin{itemize}
	\item[(i)] Any finitely generated module can be written as the finite direct sum of indecomposable modules.
	\item[(ii)] If $$\bigoplus_{i=1}^m M_i \cong \bigoplus_{j=1}^n N_j$$
	for indecomposable $M_i$'s and $N_j$'s, then $m=n$ and there is a permutation, $\sigma \in S_n$, such that $M_i \cong N_{\sigma(i)}$ for all $i=1,2, \cdots, n$.  
\end{itemize}

It's clear that (i) golds for any noetherian ring, since any decomposition of a noetherian module must eventually reach an indecomposable. In this chapter we will focus on proving (ii).

\section{Skew group algebra S\#G indecomposable projectives}
This section is largely based on the book by \cite{CMR}. This section will use definitions and theorems from representation theory as taught in the courses MA3203 - Ring Theory and MA3204 - homological algebra. Since I do not assume knowledge of this I have created appendix \ref{appendix}. I will try to use footnotes to indicate where such theorems are used.
\begin{defin}
If $G$ is a subgroup of $GL_n(\C)$, we can extend the group action of $G$ to $\C\llbracket x_1, \cdots, x_n\rrbracket$. We then define the \underline{skew group algebra $\C \llbracket x_1, \cdots, x_n \rrbracket \# G$} to be the algebra generated by elements of the form $f \cdot g$ with $f \in \C\llbracket x_1, \cdots, x_n\rrbracket$ and $g \in G$, and we define the multiplication by
$$ (f_1 \cdot g_1) \cdot (f_2 \cdot g_2) = (f_1 \cdot f_2^{g_1}) \cdot (g_1 \cdot g_2) $$
Where $f^g$ denotes the image of $f$ under the action of $g$.
\end{defin}

\begin{theorem}
We have an isomorphism of rings
$$ e \C\llbracket x, y \rrbracket \# G e \simeq \C\llbracket x, y \rrbracket^G $$
where $e = \frac{1}{|G|} \sum_{g \in G} g$.

\begin{proof}
Let $f^g$ denote the image of $f$ under the action of $g$. Then if we let $f(x,y)g$ be an element of the skew algebra we get that $e f(x,y)g e = f(x, y)^e \cdot ege = f(x, y)^e \cdot e = e \cdot f(x, y)$. It then follows that $  e \C\llbracket x, y\rrbracket \# G e$ is isomorphic to the image of $\C\llbracket x, y \rrbracket$ under the action of $e$. Since $ge=g$ for all $g\in G$ it is clear that the image of $e$ is contained in the fixed ring. For the converse you just need to notice that the fixed ring is fixed under $e$ and thus is contained in the image.
\end{proof}
\end{theorem}

\begin{lemma}
\label{lem:S proj => SG proj}
Let $S = \C\llbracket x, y \rrbracket$. An $S\#G$-module is projective if and only if it is projective as an $S$-module.

\begin{proof}
Onlyifity follows from $S\#G$ being a free $S$-module, it is isomorphic to $\bigoplus_{g \in G} S$. Thus we need only show ifity.

First we need to see that an $S\#G$-linear map is just an $S$-linear map such that $f(g(m))=g(f(m))$ for all $g \in G$. Equivalently $f(m) = g(f(g^{-1}(m)))$. This allows us to define a group action on $S$-linear maps by $f^g(m) = g(f(g^{-1}(m)))$. Then we can restate it as $$ \Hom_{S\#G}(M,N) = \Hom_S(M,N)^G$$
Clearly if $f$ is $S\#G$-linear then it's in $\Hom_S(M,N)^G$. To see the other inclusion, let $f$ be an $S$-linear map that is fixed under $G$. Then $f(s\cdot g m) = s f(g m) = s\cdot g(f(g^{-1} g m)) = s \cdot g f(m) $, and hence $f$ is $S\#G$-linear. Nextly I want to show that $-^G$ is an exact functor.

If $K$ is the kernel of a map $f: M \to N$, then the kernel of the inuced map $f^G : M^G \to N^G$ is of course just $K \cap M^G$ which equals $K^G$. Assume $f$ is epi and let $n \in N^G$. Consider a preimage $m$ such that $f(m)=n$. Let $\theta = \frac{1}{|G|}\sum_{g \in G} g(m)$. Then $\theta$ is in $M^G$ and $f(\theta) = \frac{1}{|G|}\sum_{g \in G} g(f(m)) = \frac{1}{|G|}\sum_{g \in G} n = n$.

This implies that if $\Hom_S(P, -)$ is exact then $\Hom_S(P, -)^G = \Hom_{S\#G}(P, -)$ is exact and our lemma follows.
\end{proof}
\end{lemma}

\begin{theorem}
Let $S = \C\llbracket x, y \rrbracket$ and let $\mathfrak{m} = \langle x, y \rangle_S$ be the radical of $S$. Then there are bijections between the indecomposable projective $S\#G$-modules and the indecomposable $\C G$-modules given by

\begin{tikzcd}
\left\lbrace \begin{matrix}
\text{indecomposable projective}\\
S\#G\text{-modules}
\end{matrix} \right\rbrace 
& 
\left\lbrace \begin{matrix}
\text{indecomposable}\\
\C G\text{-modules}
\end{matrix} \right\rbrace\\
\mathcal{F}: P \arrow[mapsto]{r} & P/\mathfrak{m}P\\
\mathcal{G}: S \otimes_\C W & W \arrow[mapsto]{l}
\end{tikzcd}
\\
Where the $S\#G$-module structure on $S \otimes_\C W$ is given by $(s \cdot g) \cdot f \otimes v = sf^g \otimes v^g$.

\begin{proof}
First we should show that $S \otimes_\C W$ is an irreducible projective $S\#G$-module and that $P/\mathfrak{m}P$ is infact an irreducible $\C G$-module. Since $S \otimes_\C W$ is a free $S$-module it follows from lemma \ref{lem:S proj => SG proj} that it is projective. To see that it is irreducible we will first study it as an $S$-module and exploit the fact that $\Hom_{S\#G}(M,N) \subseteq \Hom_S(M,N)$. 

Since $\mathfrak{m}$ is the radical of $S$ we have that 
\begin{align*}
\frac{S \otimes_\C W}{\mathfrak{m}S \otimes_\C W} \cong S/\mathfrak{m} \otimes_\C W \cong \C \otimes_\C W \cong W
\end{align*}
$W$ is the top of $S \otimes_\C W$. Further since the projection is $S\#G$-linear we have that $S \otimes_\C W$ is the projective cover of $W$ also as $S\#G$-modules. Then since $W$ is simple it follows that $S \otimes_\C W$ is an indecomposable projective.

It's clear that $P/\mathfrak{m}P$ is a $\C G$-module, because $\C G$ is a subring of $S\#G$. To see that it's indecomposable we will first show that it's indecomposable as an $S\#G$-module. By considering $P$ and $P/\mathfrak{m}P$ as $S$-modules and using the same argument as above we see that $P$ is the projective cover of $P/\mathfrak{m}P$. Then since $P$ is indecomposable $P/\mathfrak{m}P$ must also be indecomposable as an $S\#G$-module.

To see that this implies $P/\mathfrak{m}P$ is indecomposable as a $\C G$-module notice that $P/\mathfrak{m}P$ is annihilated by the ideal $\langle \mathfrak{m} \rangle$. This means it's an indecomposable as $S\#G$-module if and only if it's indecomposable as an $S\#G/\langle \mathfrak{m} \rangle$-module. Then since $S\#G/\langle \mathfrak{m} \rangle \cong \C G$ it follows that $P/\mathfrak{m}P$ is an indecomposable $\C G$-module.

To see that the given maps are bijections we will show that they are mutual inverses. First to see that $\mathcal{F}(\mathcal{G}(W)) \cong W$ we simply look at the definition
\begin{equation*}
\begin{split}
\frac{S \otimes_\C W}{\mathfrak{m}S \otimes_\C W} \cong S/\mathfrak{m} \otimes_\C W \cong \C \otimes_\C W \cong W
\end{split}
\end{equation*}
Next we consider $\mathcal{G}(\mathcal{F}(P)) = S \otimes_\C P/\mathfrak{m}P$. We have already seen that it's projective. Both $P$ and $S \otimes_\C P/\mathfrak{m}P$ have a natural projection onto $P/\mathfrak{m}P$, and by projectivity we get an induced $S\#G$-linear map from $S \otimes_\C P/\mathfrak{m}P$ to $P$:
\begin{center}
\begin{tikzcd}
& S \otimes_\C P/\mathfrak{m}P \arrow[twoheadrightarrow]{d} \arrow[dashrightarrow]{dl}\\
P \arrow[->>]{r} & P/\mathfrak{m}P
\end{tikzcd}
\end{center}
Further since $\mathfrak{m}$ is the radical of $S$, both $P$ and $S \otimes_\C P/\mathfrak{m}P$ are projective covers of $P/\mathfrak{m}P$ (as $S$-modules). This means that the map is an isomorphism of $S$-modules, and therefor it is also an isomorphism of $S\#G$-modules.
\end{proof}

\iffalse
\begin{proof}
To see that this are bijections we will show that they are mutuall inverses. First to see that $\mathcal{F}(\mathcal{G}(W)) \cong W$ we simply look at the definition
\begin{equation*}
\begin{split}
\frac{S \otimes_\C W}{\mathfrak{m}S \otimes_\C W} \cong S/\mathfrak{m} \otimes_\C W \cong \C \otimes_\C W \cong W
\end{split}
\end{equation*}
Next we consider $\mathcal{G}(\mathcal{F}(P)) = S \otimes_\C P/\mathfrak{m}P$. Notice that the top of $ S \otimes_\C P/\mathfrak{m}P$ is isomorphic to $P/\mathfrak{m}P$. Then by the uniquness of tops we have that $ S \otimes_\C P/\mathfrak{m}P \cong P$.

The only thing that remains to show is that $\mathcal{F}$ and $\mathcal{G}$ are well-defined maps with the correct images. Namely that $\mathcal{F}(P)$ is an indecomposable $\C G$-module and that $\mathcal{G}(W)$ is an indecomposable projective $S\#G$-module.

Since $P$ is an indecomposable projective we have that $\mathcal{F}(P)$ is a simple $S\#G$-module. By the natural inclusion $\C G \hookrightarrow S\#G$ $\mathcal{F}(P)$ becomes a $\C G$-module. Assume that $\mathcal{F}(P)$ decomposes as $P_1 \oplus P_2$ as a $\C G$-module. Then since the action of $x$ and $y$ are trivial on $\mathcal{F}(P)$, $P_1 \oplus P_2$ is a decomposition of $S\#G$-modules. This implies that $P_1 = 0$ or $P_2=0$, and we have that $\mathcal{F}(P)$ is indecomposable.

Lastly we want to show that $\mathcal{G}(W)$ is projective and indecomposable. Since $S \otimes_\C W$ is free as an $S$-module it follows from lemma \ref{lem:S proj => SG proj}, that it is a projective $S\#G$-module. To see that it is indecomposable, we just need to notice that its top, $S/\mathfrak{m} \otimes W \cong W$, is simple.
\end{proof}
\fi

\end{theorem}

\subsection{The Gabriel quiver}

\begin{defin}
For a skew group algebra $S\#G$ we define its \underline{Gabriel quiver} to be the quiver with verticies as the indecomposable projective modules of $S\#G$. The arrows are given by taking the minimal projective resolution of $P/\mathfrak{m}P$, where $\mathfrak{m}$ is as defined above. If the minimal projective resolution of $P/\mathfrak{m}P$ is given by
\begin{center}
\begin{tikzcd}
\cdots \arrow{r} & Q_1 \arrow{r} & Q_0 \arrow{r} & 0
\end{tikzcd}
\end{center}
We say there is an arrow from $P$ to $P'$ if $P'$ appears as a direct summand of $Q_1$.
\end{defin}

\begin{defin}
Exterior algebra 
.............................
\end{defin}

\begin{prop}
If $S$ is the ring of formal power series over $\C$ in $n$ variables, and $G$ is a finite group acting on $S$, let $V=\mathfrak{m}/\mathfrak{m}^2$. Then the minimal projective resolution of $\C \cong S/\mathfrak{m}$ is given by
\begin{center}
\begin{tikzcd}
0  \arrow{r} & S \otimes_\C \bigwedge\limits^n V  \arrow{r}{\partial_n} & \cdots \arrow{r}{\partial_2} & S \otimes_\C \bigwedge\limits^{1} V \arrow{r}{\partial_1} & S \arrow{r} & 0
\end{tikzcd}
\end{center}
Where $\partial_p$ is the $S\#G$-linear map defined by
\begin{align*}
\partial_p(s \otimes x_{i_1} \wedge x_{i_2} \wedge \cdots \wedge x_{i_p}) = \sum_{j=1}^{p} (-1)^{j+1} sx_{i_j} \otimes x_{i_1} \wedge \cdots \wedge \hat{x}_{i_{j}} \wedge \cdots \wedge x_{i_{p}} 
\end{align*}
Where $x_{i_1} \wedge x_{i_2} \wedge \cdots \wedge x_{i_p}$ is one of the standard basis vectors for $\bigwedge\limits^n V$, namely $i_1 < i_2 < \cdots < i_p$, and  $\hat{x}_j$ means that $x_j$ is ommited.

\begin{proof}
First we should show that this is a projective resolution. Note that since the maps are $S\#G$-linear, showing that it's a minimal free resolution as an $S$-module implies it is a minimal projective resolution as an $S\#G$-module. Then what we need to show is 
\begin{itemize}
\item[(i)] $\Cok \partial_1 = \C$
\item[(ii)] $\partial_{p-1} \circ \partial_{p} = 0$ for all $p$
\item[(iii)] $\Image \partial_{p+1} = \Ker \partial_p$ for $p \geq 2$
\end{itemize}
(i) is clear since the image of $\partial_1$ is $\mathfrak{m}$. (ii) can be shown through a quick computation
\begin{align*}
\partial_{p-1} \circ \partial_{p}(s \otimes x_{i_1} \wedge \cdots \wedge x_{i_p}) &=\\ 
\partial_{p-1} \left(\sum_{j=1}^p (-1)^{j+1} sx_{i_j} \otimes x_{i_1} \wedge \cdots \hat{x}_j \wedge \cdots \wedge x_{i_p} \right)&=\\
\sum_{j=1}^p (-1)^{j+1} \Bigg(\sum_{k=1}^{j-1} (-1)^{k+1} sx_{i_j}x_{i_k} \otimes x_{i_1}\wedge \cdots \hat{x}_{i_k} \wedge \cdots \wedge \cdots \hat{x}_j \wedge \cdots \wedge x_{i_p}& +\\
\sum_{k=j+1}^{p} (-1)^k sx_{i_j}x_{i_k} \otimes x_{i_1}\wedge \cdots \hat{x}_{i_j} \wedge \cdots \wedge \cdots \hat{x}_k \wedge \cdots \wedge x_{i_p}\Bigg) &
\end{align*}
From here we notice that the term with $j < k$ is canceled by the term where $k < j$, because they are the negatives of each other. Thus the composition is 0. This would then imply that $\Image \partial_{p+1} \subseteq \Ker \partial_p$, so for part (iii) we need only show that $\Ker \partial_p \subseteq \Image \partial_{p+1}$.

First some notation: let $\mathfrak{I}_p$ be the set of all tuples $(i_1, i_2, \cdots, i_p)$ with $i_1 < i_2 < \cdots < i_p$ and $1 \leq i_j \leq n$, and let $x_I$ denote $x_{i_1} \wedge \cdots x_{i_p}$ when $I=(i_1, \cdots, i_p)$. Then assume $$ \sum_{I \in \mathfrak{I}_p} s_I \otimes x_I$$ is in the kernel of $\partial_p$. 


Maybe just prove n=2, its simpler....

\end{proof}
\end{prop}

\section{The endomorphism ring of $S$ as an $S^G$-module}
Isomorphism to $S\#G$ implies proj SG <-> $S^G$ direct summands of $S$.

\begin{theorem}
Let $S$ be the complex power series ring in two variables, $G$ be a finite subgroup of $GL_2(\C)$, $R = S^G$ the fixed ring of $S$ under the action of $G$, and $(S\#G)^G$ be the fixed ring of $S\#G$ under left multiplication by $G$. Then $S$ is isomorphic to $(S\#G)^G$ as $R$-modules.

\begin{proof}
To see this we will define an injective $R$-linear map from $S$ to $S\#G$ and show that it's image is $(S\#G)^G$. Let $\rho: S \to S\#G$ be given by
\begin{align*}
\rho(s) = \sum_{g \in G} s^g \cdot g.
\end{align*}
It's clear that it's injective and it is $R$-linear because $$\rho(rs) = \sum_{g \in G} r^gs^g \cdot g = r\sum_{g \in G} s^g \cdot g.$$ It should also be clear that the image is contained in $(S\#G)^G$ becuase
\begin{align*}
h \cdot \rho(s) = \sum_{g \in G} h \cdot s^g \cdot g = \sum_{g \in G} s^{hg} \cdot hg = \rho(s).
\end{align*}
To see that the image is all of $(S\#G)^G$ consider an arbitrary element in $(S\#G)^G$, $\psi = \sum_{g\in G} s_g \cdot g$. Since $\psi$ is fixed under left multiplication by $G$ we must have that
\begin{align*}
\sum_{g\in G} s_g^h \cdot hg = \sum_{g\in G} s_g \cdot g,
\end{align*}
in particular $s_h$ must equal $s_1^h$ and it follows that $\psi = \rho(s_1)$.
\end{proof}
\end{theorem}

\begin{theorem}

\end{theorem}

\section{Maximal Cohen-Macaulay modules of $S^G$}
\begin{defin}
If $R$ is a local ring with residual field $k$ we define the \underline{depth} of a module, $M$, to be the minimal $n$ such that the extension $\Ext^n_R(k, M)$ is non-zero.
\end{defin}

\begin{defin}
If $R$ is a commutative ring and $M$ is an $R$-module, a \underline{regular sequence} is a sequence of elements of $R$, $r_1, r_2, \cdots r_n$ such that $M/\langle r_1, \cdots, r_i \rangle M$ is non-zero and multiplication by $r_i$ is injective on $M/\langle r_1, \cdots, r_{i-1} \rangle M$.
\end{defin}

\begin{defin}
If $M$ is a module over a local ring $R$ with Krull-dimension $d$ we say that $M$ is \underline{maximal Cohen Macaulay (MCM)} if the depth of $M$ equals $d$.
\end{defin}

\begin{theorem}
If $G$ is a finite subgroup of $GL_n(\C)$, $S$ is the formal power series ring in $n$ variables and $R = S^G$ is the ring fixed under the action of $G$, then $R$ is a sirect summand of $S$ as $R$-modules.

\begin{proof}
Consider the map $\pi: S \to R$ given by
\begin{align*}
\pi(s) = \frac{1}{|G|} \sum_{g\in G} s^g
\end{align*}
It's clear that the image of $\pi$ is in $R$ becuase an action from $G$ wil just permute the order of teh sum. Further \begin{align*}
\pi(r) = \frac{1}{|G|} \sum_{g\in G} r^g = \frac{1}{|G|} \sum_{g\in G} r = r,
\end{align*}
so $\pi$ splits the inclusion $R \hookrightarrow S$ which shows that $R$ is a direct summand of $S$.
\end{proof}
\end{theorem}

\begin{appendices}
\section{Representation theory}
\label{appendix}
Define rep of ring

Define group representation
this is the same as $\C G$-rep

\begin{theorem}
\label{schur}
\end{theorem}
Schur's lemma

Projective module + exact hom

indec module + direct sum

indec proj = summand of ring = idempotent of ring

projective resolution

Ext
\end{appendices}

\section{Random Thoughts I need to figure out}
I R=S then they have the same depth meaning M is cohen macaulay iff it's projective dimension is 0 (Auslander-Buschsbauw), but that means its a direct summand of S as S(=R)-module, which make sense. If I can show that R = S/(f) fro some polynomial f, can I show R-direct summands of S have projective dimension 1 over S? Can I show R has depth depth(S)-1? Then I also need to prove Auslander-Buschsbauw... 

If P is indec projective then it is teh direct summand of a free module. Then there is a non-zero map to the ring. If P is not a summand of the ring then P decomposes, contradiction. Why do we need KRS???????

\nocite{*}
\bibliography{mybib}
\bibliographystyle{apalike}

\end{document}