\documentclass[11pt, a4paper, english]{article}
\usepackage[utf8]{inputenc}
\usepackage{babel, amsmath, amsthm, amssymb, amsfonts, enumitem, mathtools, centernot}
\usepackage{tikz-cd}
\usepackage{stmaryrd}
\newcommand\tab[1][1cm]{\hspace*{1}}
\DeclarePairedDelimiter{\ceil}{\lceil}{\rceil}

\newtheorem*{prop}{Proposition}
\newtheorem*{lemma}{Lemma}
\newtheorem*{theorem}{Theorem}
\newtheorem*{defin}{Definition}
\newtheorem*{example}{Example}

\newcommand{\C}{\mathbb{C}}
\DeclareMathOperator{\Hom}{Hom}

\begin{document}
\title{McKay correspondence}
\author{Jacob Fjeld Grevstad}
\maketitle

\begin{abstract}
The goal of this thesis is to establish a 1-1 correspondence between quivers created from the four following sets whenever $G$ is a finite subgroup of $SL(2,\C)$ and $S$ is the power series ring $\C \llbracket x, y \rrbracket$
\begin{itemize}
\item The Maximal Cohen-Macaulay modules of the fixed ring $S^G$.
\item The indecomposable projective modules of the skew group algerba $S\#G$.
\item The indecomposable projective modules of $End_{S^G}(S)$.
\item The irreducible representations of $G$ (indecomposable $\C G$-modules).
\end{itemize}
Much of the thesis will be used to define these four quivers and to develope tools to establish such a correspondence. A similar correspondence can be established for a general field $k$ and a finite subgroup of $GL(n, k)$ with order nonzero in $k$, but the case for $SL(2,\C)$ is the most interesting as the quivers will be extended Dynkin diagrams.
\end{abstract}

\section*{Finite subgroups of SL(2,C)}

\section*{characters and irreducible representations}
Recall that the trace of a matrix is defined to be the sum of its diagonal element and that the trace satisfies two important equations. Namely
$$tr(A+B)=tr(A)+tr(B) \text{  and  } tr(AB)=tr(BA)$$

For a given representation of $G$, $\rho: G \to GL_n(\mathbb{C})$ we define its characther by $\chi_\rho : G \to \mathbb{C}$, $\chi_\rho(g) = tr(\rho(g))$.

\begin{prop}
Conjugate elements in $G$ take the same value under a character.
\begin{proof}
Let $g$ and $g'$ be in the same conjugacy class. Then there exists an element $h$ such that $h^{-1}gh=g'$. Then we have
\begin{equation*}
\begin{split}
\chi(g')=\chi(h^{-1}gh) = tr(\rho(h)^{-1}\rho(g)\rho(h)) \stackrel{\mathclap{\normalfont\mbox{\tiny{*}}}}{=} tr(\rho(g)\rho(h)\rho(h)^{-1}) = tr(\rho(g))=\chi(g)
\end{split}
\end{equation*}
In (*) we use the fact that $tr(AB)=tr(BA)$.
\end{proof}
\end{prop}

\begin{lemma}
For a finite abelian group $G$ any irreducible representation must be 1-dimensional.
\begin{proof}
Let $\rho: G \to GL(V)$ be an irreducible representation. Since $G$ is abelian we have that $\rho(g)\rho(h)v = \rho(h)\rho(g)v$, and thus $\rho(g)$ is a homomorphism of $G$-representations. Then by Schur's lemma $\rho(g)$ must be a scalar multiplication. This implies that $\rho$ can be written as a direct sum of 1-dimensional representations, but since $\rho$ is irreducible $\rho$ must be 1-dimensional.
\end{proof}
\end{lemma}

\begin{prop}
If $\chi$ is the character of a representation, $\rho$, with dimension $n$ of a group $G$, and $g$ is an element of $G$ with order $m$, then the following holds
\begin{itemize}
 \item[(1)] $\chi(1) = n$
 \item[(2)] $\chi(g)$ is the sum of $m$-th roots of unity.
 \item[(3)] $chi(g^{-1}) = \overline{\chi(g)}$
\end{itemize}
\begin{proof}

\begin{itemize}
\item[]
\item[(1)]
The first result is immidiate.
$$\chi(1) = tr\left(\begin{bmatrix}
1 & \cdots & 0\\
\vdots & \ddots & \vdots\\
0 & \cdots & 1
\end{bmatrix}\right) = n$$
\item[(2)]
Since $\langle g \rangle$ is an abelian group, $rho$ decomposes into $n$ 1-dimensional $\langle g \rangle$-representations. Then there is a basis such that $\rho(g)$ is diagonal. Since $g$ has order $m$ it follows that the diagonal entries of $\rho(g)$ must be $m$-th roots of unity. Thus $\chi(g) = tr(\rho(g))$ must be the sum of $m$-th roots of unity.
\item[(3)]
Using the same basis as above and the fact that $\omega^{-1} = \overline{\omega}$ when $\omega$ is a root of unity we see that $\chi(g^{-1}) = tr(\rho(g)^{-1}) = \overline{tr(\rho(g))} = \overline{\chi(g)}$.
\end{itemize}
\end{proof}
\end{prop}

\section*{The McKay quiver}
\begin{defin}
Let $G$ be a finite subgroup of $GL(n, \C)$, and let $V$ be the cannonical representation (the one that sends $g$ to $g$). Then we define the \underline{McKay quiver} of $G$ to be the quiver with verticies the irreducible representations of $G$, denoted $V_i$. For two irreducible representations $V_i$ and $V_j$ we say there is an arrow from the former to the latter if and only if $V_j$ is a direct summand of $V \otimes V_i$. 
\end{defin}

\begin{example}
Let $G$ be the group generated by $g =\begin{bmatrix}
\omega^2 & 0\\
0 & \omega^{3}
\end{bmatrix}$, where $\omega$ is the primitive fifth root of unity. Then there are five different irreducible representations, the one sending $g$ to $\omega$, $\omega^2$, $\omega^3$, $\omega^4$ respectively, and the trivial representation. Denote the representation sending $g$ to $\omega^i$ by $V_i$, and let $V = V_2 \oplus V_3$ be the cannonical representation. Note that $V_i \otimes V_j = V_{i+j}$, where $i+j$ is understood to be modulo 5. Then we get the following McKay-quiver
$$
\begin{tikzcd}
& V_0 \arrow[<->]{ddr} \arrow[<->]{ddl} & \\
V_4 && V_1 \arrow[<->]{ll}\arrow[<->]{dll}\\
V_3 && V_2 \arrow[<->]{ull}
\end{tikzcd}  
$$
\end{example}

\section*{Skew algebra S\#G indecomposable projectives}
\begin{defin}
If $G$ is a subgroup of $GL_n(\C)$, we can extend the group action of $G$ to $\C\llbracket x_1, \cdots, x_n\rrbracket$. We then define the \underline{skew algebra $\C \llbracket x_1, \cdots, x_n \rrbracket \# G$} to be the algebra generated by elements of the form $f \cdot g$ with $f \in \C\llbracket x_1, \cdots, x_n\rrbracket$ and $g \in G$, and we define the multiplication by
$$ (f_1 \cdot g_1) \cdot (f_2 \cdot g_2) = (f_1 \cdot f_2^{g_1}) \cdot (g_1 \cdot g_2) $$
Where $f^g$ denotes the image of $f$ under the action of $g$.
\end{defin}

\begin{theorem}
We have an isomorphism of rings
$$ e \C\llbracket x, y \rrbracket \# G e \simeq \C\llbracket x, y \rrbracket^G $$
where $e = \frac{1}{|G|} \sum_{g \in G} g$.

\begin{proof}
Let $f^g$ denote the image of $f$ under the action of $g$. Then if we let $f(x,y)g$ be an element of the skew algebra we get that $e f(x,y)g e = f(x, y)^e \cdot ege = f(x, y)^e \cdot e$. It then follows that $  e \C\llbracket x, y\rrbracket \# G e$ is isomorphic to the image of $e$. Since $ge=g$ for all $g\in G$ it is clear that the image of $e$ is contained in the fixed ring. For the converse you just need to notice that the fixed ring is fixed under $e$ and thus is contained in the image.
\end{proof}
\end{theorem}

\begin{lemma}
Let $S = \C\llbracket x, y \rrbracket$. An $S\#G$-modulo is projective if it is projective as an $S$-module.

\begin{proof}
First we need to see that an $S\#G$-linear map is just an $S$-linear map such that $f(g(m))=g(f(m))$ for all $g \in G$. Equivalently $f(m) = g(f(g^{-1}(m)))$. This allows us to define a group action $f^g(m) = g(f(g^{-1}(m)))$. Then we can restate it as $$ \Hom_{S\#G}(M,N) = \Hom_S(M,N)^G$$
Clearly if $f$ is $S\#G$-linear then it's in $\Hom_S(M,N)^G$. To see the other inclusion, let $f$ be an $S$-linear map such that fixed under $G$. Then $f(s\cdot g m) = s f(g m) = s\cdot g(f(g^{-1} g m)) = s \cdot g f(m) $, and hence $f$ is $S\#G$-linear. Nextly I want to show that $-^G$ is an exact functor.
\\
If $K$ is the kernel of a map $f: M \to N$, then the kernel of the inuced map $f^G : M^G \to N^G$ is of course just $K \cap M^G$ which equals $K^G$. Assume $f$ is epi and let $n \in N^G$. Consider a preimage $m$ such that $f(m)=n$. Let $\theta = \frac{1}{|G|}\sum_{g \in G} g(m)$. Then $\theta$ is in $M^G$ and $f(\theta) = \frac{1}{|G|}\frac{1}{|G|}\sum_{g \in G} g(f(m)) = \frac{1}{|G|}\frac{1}{|G|}\sum_{g \in G} n = n$.
\\
This implies that if $\Hom_S(P, -)$ is exact then $\Hom_S(P, -)^G = \Hom_{S\#G}(P, -)$ is exact and our lemma follows.
\end{proof}
\end{lemma}

\begin{theorem}
Let $S = \C\llbracket x, y \rrbracket$ and let $\mathfrak{m} = \langle x, y \rangle_S$. Then there are bijections between the indecomposable projective $S\#G$-modules and the indecomposable $\C G$-modules given by
\begin{equation*}
\begin{split}
\mathcal{F}: P &\mapsto P/\mathfrak{m}P\\
\mathcal{G}: W &\mapsto S \otimes_\C W
\end{split}
\end{equation*}
Where the $S\#G$-module structure on $S \otimes_\C W$ is given by $(s \cdot g) \cdot f \otimes v = sf^g \otimes g(v)$.

\begin{proof}
To see that this are bijections we will show that they are mutuall inverses. First to see that $\mathcal{F}(\mathcal{G}(W)) \cong W$ we simply look at the definition
\begin{equation*}
\begin{split}
\frac{S \otimes_\C W}{\mathfrak{m}S \otimes_\C W} \cong S/\mathfrak{m} \otimes_\C W \cong \C \otimes_\C W \cong W
\end{split}
\end{equation*}
Next we consider $\mathcal{G}(\mathcal{F}(P)) = S \otimes_\C P/\mathfrak{m}P$. Notice that the top of $ S \otimes_\C P/\mathfrak{m}P$ is isomorphic to $P/\mathfrak{m}P$. Then by the uniquness of tops we have that $ S \otimes_\C P/\mathfrak{m}P \cong P$.
\\
The only thing that remains to show is that $\mathcal{F}$ and $\mathcal{G}$ are well-defined maps with the correct images. Namely that $\mathcal{F}(P)$ is an indecomposable $\C G$-module and that $\mathcal{G}(W)$ is an indecomposable projective $S\#G$-module.
\\
Since $P$ is an indecomposable projective we have that $\mathcal{F}(P)$ is a simple $S\#G$-module. By the natural inclusion $\C G \hookrightarrow S\#G$ $\mathcal{F}(P)$ becomes a $\C G$-module. Assume that $\mathcal{F}(P)$ decomposes as $P_1 \oplus P_2$ as a $\C G$-module. Then since the action of $x$ and $y$ are trivial on $\mathcal{F}(P)$, $P_1 \oplus P_2$ is a decomposition of $S\#G$-modules. This implies that $P_1 = 0$ or $P_2=0$, and we have that $\mathcal{F}(P)$ is indecomposable.
\\
Lastly we want to show that $\mathcal{G}(W)$ is projective and indecomposable.
\end{proof}

\end{theorem}

\section*{$End_{S^G}(S)$}

\section*{MCM modules}
\begin{defin}
If $R$ is a local ring with residual field $k$ we define the \underline{depth} of a module, $M$, to be the minimal $n$ such that the extension $Ext^n_R(k, M)$ is non-zero.
\end{defin}

\begin{defin}
If $M$ is a module over a local ring $R$ with Krull-dimension $d$ we say that $M$ is \underline{maximal Cohen Macaulay (MCM)} if the depth of $M$ equals $d$.
\end{defin}

\end{document}