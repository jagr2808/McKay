\documentclass[11pt, a4paper, english]{article}
\usepackage[utf8]{inputenc}
\usepackage{babel, amsmath, amsthm, amssymb, amsfonts, enumitem, mathtools, centernot}
\usepackage{tikz-cd}
\usepackage{stmaryrd}
\usepackage{cite}
\usepackage{hyperref}
\usepackage{cleveref}
\usepackage[toc,page]{appendix}
\newcommand\tab[1][1cm]{\hspace*{1}}
\DeclarePairedDelimiter{\ceil}{\lceil}{\rceil}

\newtheorem{prop}{Proposition}
\numberwithin{prop}{section}
\newtheorem{lemma}{Lemma}
\numberwithin{lemma}{section}
\newtheorem{theorem}{Theorem}
\numberwithin{theorem}{section}
\newtheorem{defin}{Definition}
\numberwithin{defin}{section}
\newtheorem{example}{Example}
\numberwithin{example}{section}

\newcommand{\C}{\mathbb{C}}
\newcommand{\Z}{\mathbb{Z}}
\DeclareMathOperator{\Hom}{Hom}
\DeclareMathOperator{\Ext}{Ext}
\DeclareMathOperator{\Tor}{Tor}
\DeclareMathOperator{\End}{End}
\DeclareMathOperator{\Aut}{Aut}
\DeclareMathOperator{\Image}{Im}
\DeclareMathOperator{\Ker}{Ker}
\DeclareMathOperator{\Cok}{Cok}
\DeclareMathOperator{\depth}{depth}
\DeclareMathOperator{\height}{height}

\begin{document}
\title{McKay correspondence}
\author{Jacob Fjeld Grevstad}
\maketitle

\begin{abstract}
The goal of this thesis is to establish a 1-1 correspondence between quivers created from the four following sets whenever $S$ is the power series ring $\C \llbracket x, y \rrbracket$ and $G$ is a finite subgroup of $SL(2,\C)$ acting on $S$.
\begin{itemize}
\item The Maximal Cohen-Macaulay modules of the fixed ring $S^G$.
\item The finitely generated indecomposable projective modules of the skew group algebra $S\#G$.
\item The finitely generated indecomposable projective modules of $\End_{S^G}(S)$.
\item The irreducible representations of $G$ (indecomposable $\C G$-modules).
\end{itemize}
Much of the thesis will be used to define these four quivers and to develope tools to establish such a correspondence. A similar correspondence can be established for a general field $k$ and a finite subgroup of $GL(n, k)$ with order nonzero in $k$, but in the general case we will only attain the MCM-modules that apear as $S^G$-direct summands of $S$. The finite subgroups of $SL(2, \C)$ are also especially interesting because the quivers are exactly the Dynkin diagrams.
\end{abstract}

\tableofcontents

\section*{Introduction}
The McKay correspondance arised in algebraic geometry with Klein's and Du Val's study of singularities {\color{red} refference}. Specifically they studied singularities of the form $\C^2 / G$ where $G$ is a finite subgroup of $SL_2(\C)$. The resolution graph of these singularities where exactly the Dynkin diagrams. McKay observed that the resolution graphs could be computed purely by looking at the representation theory of $G$. The correspondence can be understood from many different perspectives. For this thesis our focus will be on the algebraic perspective developed by Auslander.

The correspondence is between three quivers: the McKay quiver of irreducable $G$-representations, the Gabriel quiver of indecomposable finitely generated projective $S\#G$-modules, and the Auslander-Reiten quiver of MCM $R$-modules. We will not cover the AR quiver in this thesis, but more information on it can be found {\color{red} refference}. Another important part of the correspondence is the ring isomorphism between $S\#G$ and $\End_R(S)$. We will see that this takes significant use of rammification theory, commutative algebra and galois theory.

\section{The McKay quiver}
For a given group, $G$, and a representation of that group the McKay quiver uses that representation to establish relations between the irreducible representations of $G$. In the special case that $G$ is a linear group we have a natural choice of representation to use. This leads us to define the McKay quiver as below.
\begin{defin}
Let $G$ be a finite subgroup of $GL(n, \C)$, and let $V$ be the canonical representation (the one that sends $g$ to $g$). Then we define the \underline{McKay quiver} of $G$ to be the quiver with verticies the irreducible representations of $G$, denoted $V_i$. For two irreducible representations $V_i$ and $V_j$ there is an arrow from the former to the latter if and only if $V_j$ is a direct summand of $V \otimes V_i$. 
\end{defin}

\begin{example}
Let $G$ be the group generated by $g =\begin{pmatrix}
\omega^2 & 0\\
0 & \omega^{3}
\end{pmatrix}$, where $\omega$ is a primitive fifth root of unity. Then there are five different irreducible representations, the one sending $g$ to $\omega$, $\omega^2$, $\omega^3$, $\omega^4$ respectively, and the trivial representation. Denote the representation sending $g$ to $\omega^i$ by $V_i$, and let $V = V_2 \oplus V_3$ be the cannonical representation. Note that $V_i \otimes V_j = V_{i+j}$, where $i+j$ is understood to be modulo 5. Then we get the following McKay-quiver
$$
\begin{tikzcd}
& V_0 \arrow[<->]{ddr} \arrow[<->]{ddl} & \\
V_4 && V_1 \arrow[<->]{ll}\arrow[<->]{dll}\\
V_3 && V_2 \arrow[<->]{ull}
\end{tikzcd}  
$$
\end{example}


In this section we will use the fact that $S$ and $R$ have the same Krull dimension. This can be shown in general using some tools from algebraic geometry, but in the special case when $G$ is a finite subgroup of $SL_2(\C)$ we have that $R = \C \llbracket u, v, w \rrbracket / \langle f \rangle$ for some irreducible polynomial $f$. Therefore $R$ has dimension 2, just like $S$. The proof of this uses the fact that up to a change of basis there are only five families of finite subgroups of $SL_2(\C)$, a survey of which can be found in [\href{https://staff.fnwi.uva.nl/r.r.j.bocklandt/notes/kleinian.pdf}{kleinian}] and [\href{https://homepages.warwick.ac.uk/~masda/McKay/Carrasco_Project.pdf}{Carrasco project}]. Here I will simply list the groups and the formulas for $R$.

\begin{center}
{\renewcommand{\arraystretch}{1.6}
\begin{tabular}{|c|c|c|}
\hline
McKay quiver & $G$ & $R = S^G$\\
\hline\hline
$A_n$ & $\mathbb{Z}/n\mathbb{Z}$ & 
$\C \llbracket u, v, w \rrbracket/\langle uv - w^n \rangle$
\\
\hline
$D_n$ & $BD_{4n}$ & 
$\C \llbracket u, v, w \rrbracket/\langle u^{n+1} + v^2 - uw^2\rangle$
\\
\hline
$E_6$ & $BT_24$ & 
$\C \llbracket u, v, w \rrbracket/\langle u^4 + v^3 + w^2 \rangle$
\\
\hline
$E_7$ & $BO_{48}$ &
$\C \llbracket u, v, w \rrbracket/\langle u^3v + v^3 + w^2 \rangle$
\\
\hline
$E_8$ & $BI_{120}$ &
$\C \llbracket u, v, w \rrbracket/\langle u^5 + v^3 + w^2 \rangle$
\\
\hline
\end{tabular}
}
\end{center}

Below is a table of how the groups are realized in $SL_2(\C)$. We write $\zeta$ for the primitive fifth root of unity $\exp(2\pi i/5)$

{\renewcommand{\arraystretch}{2}
\begin{tabular}{|c|c|}
\hline
$G$ & generators in $SL_2(\C)$
\\
\hline
\hline
$\mathbb{Z}/n\mathbb{Z}$ & $\begin{pmatrix}
\exp(2\pi i/n) & 0\\
0 & \exp(-2\pi i/n)
\end{pmatrix} $\\
\hline
$BD_{4n}$ &  $ \begin{pmatrix}
\exp(\pi i/n) & 0\\
0 & \exp(-\pi i/n)
\end{pmatrix}, \begin{pmatrix}
0 & i\\
i & 0
\end{pmatrix} $\\
\hline
$BT_{24}$ & $ \begin{pmatrix}
\frac{i+1}{2} & -\frac{i+1}{2}\\
\frac{-i+1}{2} & \frac{-i+1}{2}
\end{pmatrix}, \begin{pmatrix}
\frac{i+1}{2} & \frac{i+1}{2}\\
-\frac{-i+1}{2} & \frac{-i+1}{2}
\end{pmatrix} $\\
\hline
$BO_{48}$ & $BT_{24}$, $\begin{pmatrix}
\frac{1+i}{\sqrt{2}} & 0\\
0 & \frac{1-i}{\sqrt{2}}
\end{pmatrix}$\\
\hline
$BI_{120}$ & $ \begin{pmatrix}
0 & 1\\
-1 & 0
\end{pmatrix}, \begin{pmatrix}
\zeta^3 & 0\\
0 & \zeta^2
\end{pmatrix}, \frac{1}{\sqrt{5}}\begin{pmatrix}
-\zeta + \zeta^4 & \zeta^2 - \zeta^3\\
\zeta^2 - \zeta^3 & \zeta - \zeta^4
\end{pmatrix}$
\\
\hline
\end{tabular}
}

\section{Skew group algebra S\#G indecomposable projectives}
This section is largely based on the book by \cite{CMR}. This section will use definitions and theorems from representation theory as taught in the courses MA3203 - Ring Theory and MA3204 - homological algebra. Since I do not assume knowledge of this I have created appendix \ref{appendix}. I will try to use footnotes to indicate where such theorems are used.

\begin{defin}
If $G$ is a subgroup of $GL_n(\C)$, we can extend the group action of $G$ on $\C^n$ to $\C\llbracket x_1, \cdots, x_n\rrbracket$. More explicitely $G$ acts on $x_i$ as it would the $i$th basis vector of $\C^n$, and acts on products and sums by acting on each component seperately. We then define the \underline{skew group algebra $\C \llbracket x_1, \cdots, x_n \rrbracket \# G$} to be the algebra generated by elements of the form $f \cdot g$ with $f \in \C\llbracket x_1, \cdots, x_n\rrbracket$ and $g \in G$, and we define the multiplication by
$$ (f_1 \cdot g_1) \cdot (f_2 \cdot g_2) = (f_1 \cdot f_2^{g_1}) \cdot (g_1 \cdot g_2) $$
Where $f^g$ denotes the image of $f$ under the action of $g$.
\end{defin}

The skew group algebra is also sometimes called the twisted algebra, because the multiplication is "twisted" by the action of $G$.

\begin{lemma}
\label{lem:S proj => SG proj}
Let $S = \C\llbracket x, y \rrbracket$. An $S\#G$-module is projective\footnote{The definition of projective can be found in \cref{def:projective} on page \pageref{def:projective}.} if and only if it is projective as an $S$-module.

\begin{proof}
Onlyifity follows from $S\#G$ being a free $S$-module, it is isomorphic to $\bigoplus_{g \in G} S$. Thus we need only show ifity.

First we need to see that an $S\#G$-linear map is just an $S$-linear map, $f: M \to N$ between $S\#G$-modules, such that $f(g(m))=g(f(m))$ for all $g \in G$ and all $m \in M$. Equivalently $f(m) = g(f(g^{-1}(m)))$. This allows us to define a group action on $S$-linear maps by $f^g(m) = g(f(g^{-1}(m)))$. Then we just need to show $$ \Hom_{S\#G}(M,N) = \Hom_S(M,N)^G.$$
Clearly if $f$ is $S\#G$-linear then it's in $\Hom_S(M,N)^G$. To see the other inclusion, let $f$ be an $S$-linear map that is fixed under $G$. Then $f(s\cdot g m) = s f(g m) = s\cdot g(f(g^{-1} g m)) = s \cdot g f(m) $, and hence $f$ is $S\#G$-linear. 

Nextly I want to show that $-^G$ is an exact functor\footnote{The definition of an exact functor can be found in \cref{def:exact_functor} on page \pageref{def:exact_functor}.}. If $K$ is the kernel of a map $f: M \to N$, then the kernel of the induced map $f^G : M^G \to N^G$ is of course just $K \cap M^G$ which equals $K^G$. Assume $f$ is epi and let $n \in N^G$. Consider a preimage $m$ such that $f(m)=n$. Let $\theta = \frac{1}{|G|}\sum_{g \in G} g(m)$. Then $\theta$ is in $M^G$ and $f(\theta) = \frac{1}{|G|}\sum_{g \in G} g(f(m)) = \frac{1}{|G|}\sum_{g \in G} n = n$.

Recall that a module being projective is equivalent to its covariant $\Hom$-functor being exact. So if $P$ is projective as an $S$-module then $\Hom_S(P, -)$ is exact. Using our above result we get $\Hom_S(P, -)^G = \Hom_{S\#G}(P, -)$ is exact and our lemma follows.
\end{proof}
\end{lemma}

\begin{lemma}
\label{lem:radical small}
Let $S$ be the complex power series ring in $n$ variables, and $\mathfrak{m} = \langle x_i \rangle_{i=1}^n$ the radical of $S$. Then for any free $S$-module $N$, $\mathfrak{m}N$ is \underline{small} in $N$. That is if $X$ is  a submodule of $N$ such that $X + \mathfrak{m}N = N$, then $X = N$.

\begin{proof}
Let $N$ be the free module $S^{(I)} := \bigoplus\limits_{i \in I} S_i$, where $S_i \cong S$. Assume that $X$ is a submodule such that $X + \mathfrak{m}N = N$. We denote by $1_i$ the elements that is 1 at index $i$ and 0 elsewhere. Since $\{ 1_i \}$ generate $N$, it is enough to show that $X$ contains all of them. Since $X + \mathfrak{m}N = N$, we know that there is an $m_i \in \mathfrak{m}N$ and an $x_i \in X$ such that $x_i + m_i = 1_i$. Then we have that $x_i = 1_i - m_i$. Since the power series at index $i$ of $x_i$ has constant coefficient 1 it is invertible. If we multiply $x_i$ by its inverse we get $\tilde{x}_i$ which is 1 at index $i$ and some element of $\mathfrak{m}$ at index $j \neq i$, say $m_{ij}$. Then $\tilde{x}_i - \sum\limits_{j \neq i} m_{ij}\tilde{x}_j$ has a unit in index $i$ and 0 at all other indicies. Thus $X$ contains $1_i$ for all $i$, and $X = N$.
\end{proof}
\end{lemma}

\begin{theorem}
\label{thm:indec_proj_SG=indec_CG}
Let $S = \C\llbracket x, y \rrbracket$ and let $\mathfrak{m} = \langle x, y \rangle_S$ be the radical of $S$. Then there are bijections between the indecomposable\footnote{The definition of indecomposable can be found in \cref{def:indecomposable} on page \pageref{def:indecomposable}.} finitely generated projective $S\#G$-modules and the indecomposable $\C G$-modules given by
\\
\\
\begin{tikzcd}
\left\lbrace \begin{matrix}
\text{indecomposable projective}\\
S\#G\text{-modules}
\end{matrix} \right\rbrace \ar[r, ->]
& 
\left\lbrace \begin{matrix}
\text{indecomposable}\\
\C G\text{-modules}
\end{matrix} \right\rbrace\\
\mathcal{F}: P \arrow[mapsto]{r} & P/\mathfrak{m}P\\
\mathcal{G}: S \otimes_\C W & W \arrow[mapsto]{l}
\end{tikzcd}

Where the $S\#G$-module structure on $S \otimes_\C W$ is given by $(s \cdot g) \cdot f \otimes v = sf^g \otimes v^g$.

\begin{proof}
First we should show that $S \otimes_\C W$ is an indecomposable projective $S\#G$-module and that $P/\mathfrak{m}P$ is in fact an indecomposable $\C G$-module. Since $S \otimes_\C W$ is a free $S$-module it follows from \cref{lem:S proj => SG proj} that it is projective. To see that it is indecomposable we will first study it as an $S$-module and exploit the fact that $\Hom_{S\#G}(M,N) \subseteq \Hom_S(M,N)$. 

Using \cref{lem:radical small} we get that $\mathfrak{m}S \otimes_\C W$ is small in $S\otimes_\C W$. This means that we get that 
\begin{align*}
\frac{S \otimes_\C W}{\mathfrak{m}S \otimes_\C W} \cong S/\mathfrak{m} \otimes_\C W \cong \C \otimes_\C W \cong W
\end{align*}
and therefore $S \otimes_\C W \to W$ is a projective cover\footnote{The definition of projective cover can be found in \cref{def:projective_cover} on page \pageref{def:projective_cover}.} of $W$ as $S$-modules. Further since the projection $S \otimes_\C W \to W$ is $S\#G$-linear we have that $S \otimes_\C W$ is the projective cover of $W$ also as $S\#G$-modules. Assume for the sake of contradiction that $S \otimes_\C W$ decomposes as $M \oplus N$ for non-zero $M$ and $N$. Then $W$ would equal $M/\mathfrak{m}M \oplus N/\mathfrak{m}N$ as an $S\#G/ \langle \mathfrak{m} \rangle$-module. Since $S\#G/ \langle \mathfrak{m} \rangle \cong \C G$ and $W$ is indecomposable we must have that either $M/\mathfrak{m}M$ or $N/\mathfrak{m}N$ is 0. This then gives a contradiction because $\mathfrak{m}M$ and $\mathfrak{m}N$ are small in $M$ and $N$. Hence we must have that $S \otimes_\C W$ is indecomposable.

It's clear that $P/\mathfrak{m}P$ is a $\C G$-module, because $\C G$ is a subring of $S\#G$. To see that it's indecomposable we will use a similar argument as above. Assume $P/\mathfrak{m}P$ decomposes as $V \oplus W$. Then both $P$ and $S\otimes_\C V \oplus S \otimes_\C W$ are projective covers of $P/\mathfrak{m}P = V\oplus W$ we get induced $S\#G$-linear epiomorphisms between them.

\begin{center}
\begin{tikzcd}
& S \otimes_\C P/\mathfrak{m}P \arrow[twoheadrightarrow]{d} \arrow[dashed, two heads, bend right=20]{dl}\\
P \arrow[->>]{r} \arrow[dashed, two heads, bend right=5]{ur} & P/\mathfrak{m}P
\end{tikzcd}
\end{center}

Now we use the fact that $P$ is finitely generated. Since there can only be an epimorphism from a module with more or equal amount of generators, $P$ and $S\otimes_\C \mathfrak{m}P$ must have the same amount of generators and the induced maps are in fact isomorphisms of $S$-modules. Since the maps are also $S\#G$-linear we have that $P$ decomposes as $S\otimes_\C V \oplus S \otimes_\C W$. Then since $P$ is indecomposable we must have that either $S \otimes_\C V$ or $S\otimes_\C W$ is 0. That means that either $V$ or $W$ is 0, and we have shown that $P/\mathfrak{m}P$ is an indecomposable $\C G$-module.

To see that the given maps are bijections we will show that they are mutual inverses. First to see that $\mathcal{F}(\mathcal{G}(W)) \cong W$ we simply look at the definition
\begin{equation*}
\begin{split}
\frac{S \otimes_\C W}{\mathfrak{m}S \otimes_\C W} \cong S/\mathfrak{m} \otimes_\C W \cong \C \otimes_\C W \cong W
\end{split}
\end{equation*}

Next we consider $\mathcal{G}(\mathcal{F}(P)) = S \otimes_\C P/\mathfrak{m}P$. We have already seen that the induced map
\begin{center}
\begin{tikzcd}
& S \otimes_\C P/\mathfrak{m}P \arrow[twoheadrightarrow]{d} \arrow[dashrightarrow]{dl}\\
P \arrow[->>]{r} & P/\mathfrak{m}P
\end{tikzcd}
\end{center}
is an isomorphism, and thus $P \cong \mathcal{G}(\mathcal{F}(P))$.
\end{proof}

\end{theorem}

\subsection{The Gabriel quiver}
Now that we have seen that the indecomposable $\C G$-modules and the indecomposable finitely generated projective $S\#G$-modules are in correspondence we will construct the quiver that corresponds to the McKay quiver.

\begin{defin}
For a skew group algebra $S\#G$ we define its \underline{Gabriel quiver} to be the quiver with verticies as the indecomposable projective modules of $S\#G$. The arrows are given by taking the minimal projective resolution\footnote{The definition of a minimal projective resolution can be found in \cref{def:projective_resolution} on page \pageref{def:projective_resolution}.} of $P/\mathfrak{m}P$, where $\mathfrak{m}$ is as defined above. If the minimal projective resolution of $P/\mathfrak{m}P$ is given by
\begin{center}
\begin{tikzcd}
\cdots \arrow{r} & Q_1 \arrow{r} & Q_0 \arrow{r} & 0
\end{tikzcd}
\end{center}
We say there is an arrow from $P$ to $P'$, if $P'$ appears as a direct summand of $Q_1$.
\end{defin}

\begin{defin}
Let $V$ be a vector space. We then define the exterior algebra $\bigwedge V$ as the associative unital graded algebra such that the multiplication is bilinear and satisfies $x \wedge y = -y \wedge x$ for any $x$ and $y$ in $V$. 
\end{defin}
Some key properties of the exterior algebra is that $x \wedge x = 0$, and more generally that $x_1 \wedge \cdots \wedge x_p = 0$ whenever $\{x_i\}_{i=1}^p$ are linearly dependent.

The $p$th exterior power of $V$, denoted $\bigwedge\limits^p V$ is the vector space of all elements that are the product of $p$ vectors in $V$. If $\{ x_i \}_{i=1}^n$ is a basis for $V$, then $x_{i_1} \wedge \cdots \wedge x_{i_p}$ where $i_1 < i_2 < \cdots < i_p$ and $1 \leq i_j \leq n$ forms a basis for $\bigwedge\limits^p V$, thus it is ${n \choose p}$-dimensional.  

\begin{prop}
If $S$ is the ring of formal power series over $\C$ in $n$ variables, and $G$ is a finite group acting on $S$, let $V=\mathfrak{m}/\mathfrak{m}^2$. Then the minimal projective resolution of $\C \cong S/\mathfrak{m}$ as an $S$-module is given by
\begin{center}
\begin{tikzcd}
0  \arrow{r} & S \otimes_\C \bigwedge\limits^n V  \arrow{r}{\partial_n} & \cdots \arrow{r}{\partial_2} & S \otimes_\C \bigwedge\limits^{1} V \arrow{r}{\partial_1} & S \arrow{r} & 0
\end{tikzcd}
\end{center}
Where $\partial_p$ is the $S\#G$-linear map defined by
\begin{align*}
\partial_p(s \otimes x_{i_1} \wedge x_{i_2} \wedge \cdots \wedge x_{i_p}) = \sum_{j=1}^{p} (-1)^{j+1} sx_{i_j} \otimes x_{i_1} \wedge \cdots \wedge \hat{x}_{i_{j}} \wedge \cdots \wedge x_{i_{p}} 
\end{align*}
Where $x_{i_1} \wedge x_{i_2} \wedge \cdots \wedge x_{i_p}$ is one of the standard basis vectors for $\bigwedge\limits^n V$, namely $i_1 < i_2 < \cdots < i_p$, and  $\hat{x}_j$ means that $x_j$ is ommited.

\begin{proof}
Firts we should show that this is a projective resolution of $\C$. In fact the complex described above is the Koszul complex of the regular sequence\footnote{Regular sequences are defined on page \pageref{def:regular_seq} in \cref{def:regular_seq}.} $(x_i)_{i=1}^n$. The Koszul complex of a regular sequence is a projective resolution of the ring modulo the ideal generated by the regular sequence, which in this case equals $S/\langle x_i \rangle_{i=1}^n = \C$. {\color{red} refference}

Secondly we want to show that the resolution is minimal. To do this it is enough to show that for each $k \geq 1$, $\partial_k$ is a projective cover of its image, and that $S \to \C$ is a projective cover of $\C$. In other words we have to show that the kernels of the maps are small. Since $\Image \partial_{k+1} = \Ker \partial_k$ and $\Image \partial_{k+1} \subseteq \mathfrak{m} \otimes_\C \bigwedge\limits^{k+1}V$ it follows from \cref{lem:radical small} that the resolution is minimal.
\end{proof}
\end{prop}

Since $V = \mathfrak{m}/\mathfrak{m}^2 = \langle x_1, x_2, \cdots, x_n \rangle_\C$ is exactly the cannonical representation of $G$ the relationship between the McKay quiver and the Gabriel quiver should be apparent. Now we move to the next theorem for a formal argument.

\begin{theorem}
If $S$ is the complex power series ring in $n$ variables and $G$ is a finite subgroup of $GL_n(\C)$, then the McKay quiver of $G$ and the Gabriel quiver of $S\#G$ are isomorphic.
\begin{proof}
We have already seen in \cref{thm:indec_proj_SG=indec_CG} that they have the same vertices, namely if $V_i$ are the irreducible representations of $G$, then $S \otimes_\C V_i$ are the indecomposable projectives of $S\#G$. To see that they have the same arrows consider as above the minimal resolution of $\C$:
\begin{center}
\begin{tikzcd}
0  \arrow{r} & S \otimes_\C \bigwedge\limits^n V  \arrow{r}{\partial_n} & \cdots \arrow{r}{\partial_2} & S \otimes_\C \bigwedge\limits^{1} V \arrow{r}{\partial_1} & S \arrow{r} & 0.
\end{tikzcd}
\end{center}
If we tensor with $V_i$ on the right we will get a minimal resolution of $V_i$:
\begin{center}
\begin{tikzcd}
\cdots \arrow{r}{\partial_2 \otimes_\C V_i} & S \otimes_\C \bigwedge\limits^{1} V \otimes_\C V_i \arrow{r}{\partial_1 \otimes_\C V_i} & S \otimes_\C V_i \arrow{r} & 0.
\end{tikzcd}
\end{center}
From here, since $\bigwedge\limits^{1} V = V$, we see that $P_j = S \otimes_\C V_j$ appears as a direct summand of $S \otimes_\C V \otimes_\C V_i$ exactly when $V_j$ appears as a direct summand of $V \otimes_\C V_i$.
\end{proof}
\end{theorem}

\section{The endomorphism ring of $S$ as an $S^G$-module}
This section is largely based on the article by \cite{IyTa} and the book by \cite{CMR}.

In this section we will show that $S\#G$ is isomorphic to $\End_R(S)$ as rings, where $S$ is the complex power series ring in 2 variables, $G$ is a finite subgroup of $SL_2(\C)$, and $R = S^G$ is the fixed ring of $S$ by $G$. This will be the longest proof of this thesis and I have therefore decided to split it up into several steps. The proof will be done by constructing an explicit isomorphism.
\begin{center}
\begin{tikzcd}
S\#G \ar[r] & \End_R(S)\\
s \cdot g \ar[r, mapsto] & (t \mapsto s \cdot t^g)
\end{tikzcd}
\end{center}
We can easily show that this is an injective ring-homomorphism. The meat of the proof is to consider the map as a morphism of $R$-modules, and then using ramification theory to show that it is an epimorphism. To do this we will show that for every height one prime ideal $\mathfrak{p}$ of $S$ if we localize at $\mathfrak{p}$ we get a socalled unramified extension of rings.
\begin{center}
\begin{tikzcd}
R_{\mathfrak{p} \cap R} \ar[hook, r] & S_\mathfrak{p}
\end{tikzcd}
\end{center}
We will use this to show that the short exact sequence
\begin{center}
\begin{tikzcd}
I \ar[hook, r] & S_\mathfrak{p} \otimes_{R_{\mathfrak{p} \cap R}} S_\mathfrak{p} \ar[two heads, r]{}{\mu} & S_\mathfrak{p}
\end{tikzcd}
\end{center}
where $\mu$ is the multiplication map and $I$ is the kernel, has a splitting. Whenever this happens we say the extension is seperable. Now writing $\mathfrak{q} = \mathfrak{p} \cap R$, and $S_\mathfrak{q}$ for $R_\mathfrak{q} \otimes_R S$, what we really want is a split exact sequence
\begin{center}
\begin{tikzcd}
I \ar[hook, r] & S_\mathfrak{q} \otimes_{R_{\mathfrak{q}}} S_\mathfrak{q} \ar[two heads, r]{}{\mu} & S_\mathfrak{q}
\end{tikzcd}
\end{center}
We will use this splitting to construct an inverse for $S_\mathfrak{q}\#G \to \End_{R_{\mathfrak{q}}}(S_\mathfrak{q})$. Finally we will show that since we get an isomorphism whenever we localize at a height one prime ideal this means that the original map is an isomorphism.

Let us first begin with some definitions
\begin{defin}
If $A$ is a local ring with residual field $k$ we define the \underline{depth} of a module, $M$, to be the minimal $n$ such that $\Ext^n_A(k, M)$ is non-zero\footnote{The extension group is defined in \cref{def:Ext} on page \pageref{def:Ext}.}. We write $\depth_A(M)$ for this or simply $\depth(M)$ when which ring we are using is clear.
\end{defin}

\begin{defin}
\label{def:regular_seq}
If $R$ is a commutative ring and $M$ is an $R$-module, an \underline{$R$-regular sequence on $M$} is a sequence of elements of $R$, $r_1, r_2, \cdots r_n$ such that $M/\langle r_1, \cdots, r_i \rangle M$ is non-zero and multiplication by $r_i$ is injective on $M/\langle r_1, \cdots, r_{i-1} \rangle M$.
\end{defin}

\begin{prop}
For a module over a local ring all maximal regular sequences have the same length. That length is the depth of the module.
\begin{proof}
{\color{red} reffrence}
\end{proof}
\end{prop}

\begin{defin}
If $R$ is a ring, we say that $\mathfrak{p}$ is  a \underline{prime ideal} in $R$ if
\begin{enumerate}
\item $\mathfrak{p}$ is a proper ideal of $R$.
\item For any two elements $a,b \in R$ such that $ab \in \mathfrak{p}$ we must have that either $a$ is in $\mathfrak{p}$ or $b$ is.
\end{enumerate}
The \underline{height} of $\mathfrak{p}$ is the length of the longest chain of prime ideals contained in $\mathfrak{p}$:
\begin{align*}
\mathfrak{p}_0 \subset \mathfrak{p}_1 \subset \cdots \subset \mathfrak{p}_n = \mathfrak{p}.
\end{align*}
\end{defin}

\begin{defin}
Let $A$ and $B$ be two local commutative rings with maximal ideal $\mathfrak{n}$ and $\mathfrak{m}$ respectively, and let 
\begin{tikzcd}
A \ar[hook, r] & B
\end{tikzcd}
be an extension of rings. We say that the extension is \underline{unramified} if the following conditions hold:
\begin{itemize}
\item $B$ is a finitely generated $A$-module.
\item \begin{tikzcd}
A/\mathfrak{n} \ar[hook, r] & B/\mathfrak{m}
\end{tikzcd}
is a seperable field extension.
\item $\mathfrak{n}B = \mathfrak{m}$
\end{itemize}
If the two first conditions are met, and there is a positive integer $e$ such that $\mathfrak{n} B = \mathfrak{m}^e B$, we say the extension has \underline{ramification index} $e$ when $e$ is the smallest such number. Note that being unramified is then equivalent to having ramification index 1. 
\end{defin}

In order to show that unramified implies seperable we must first take a small detour.

\begin{defin}
Let $A \to B$ be an extension of rings. We then define the \underline{derivation module} $\Omega_{B | A}$ as the $B$-module with formal generators $db$ for all $b \in B$ and with the following relations:
\begin{itemize}
\item[$A$-linearity:] $d(ab + a'b') = adb + a'db'$ for all $a, a' \in A$ and $b, b' \in B$.
\item[Leibniz rule:] $d(bc) = bdc + cdb$ for all $b,c \in B$.
\end{itemize} 
\end{defin}
Note that for any polynomial expression $f(b)$ we have that $df(b) = f'(b)db$ where $f'$ is the formal derivative of $f$. Now we will show how the derivation module make a link between unramified extensions and the splitting of our sequence.

\begin{prop}
Let $A \to B$ be an unramified extension of local rings. Then $\Omega_{B|A}$ is $0$.

\begin{proof}
Keeping with the notation above we let $\mathfrak{n}$ be the maximal ideal of $A$ and $\mathfrak{m}$ the maximal ideal of $B$. Furthermore let $l$ denote $B/\mathfrak{m}$ and $k$ denote $A/\mathfrak{n}$. Then I claim there is an exact sequence
\begin{center}
\begin{tikzcd}
\mathfrak{m}/\mathfrak{m}^2 \ar[r]{}{\alpha} & \Omega_{B|A} \otimes_B B/\mathfrak{m} \ar[r] & \Omega_{l|A} \ar[r] & 0
\end{tikzcd}
\end{center}
where $\alpha(\overline{m}) = d_{B|A}m \otimes 1$ for any $m$ in $\mathfrak{m}$. Let's first show that $\alpha$ is well defined. Let $m_1 \cdot m_2$ be in $\mathfrak{m}^2$. Then we need to show that $\alpha(\overline{m_1 \cdot m_2})$ is 0. 
\begin{align*}
\alpha(\overline{m_1 \cdot m_2}) &= d_{B|A}(m_1 \cdot m_2) \otimes 1 =\\ 
m_1 d_{B|A}m_2 \otimes 1 &+ m_2d_{B|A}m_1 \otimes 1 =\\ 
d_{B|A}m_2 \otimes (m_1 \cdot 1) &+ d_{B|A}m_1 \otimes (m_2 \cdot 1)
\end{align*}
Since $l = B/\mathfrak{m}$ we have that $m_1 \cdot 1$ and $m_2 \cdot 1$ is 0 in $l$, thus the right hand side is 0, and $\alpha$ is well defined.

The map $\Omega_{B|A} \otimes_B B/\mathfrak{m} \to \Omega_{l|A}$ is just the natural projection sending $db \otimes 1$ to $d\overline{b}$, where $\overline{b}$ is the projection of $b$ onto $l$. We want to show that this is the cokernel of $\alpha$. The kernel of $\Omega_{B|A} \otimes_B B/\mathfrak{m} \to \Omega_{l|A}$ is generated by $dm \otimes 1$ for $m \in \mathfrak{m}$, but this is exactly the image of $\alpha$, thus the sequence is exact.

Nextly we want to show that $\Omega_{l|A} = 0$. Since $\mathfrak{n} \subseteq \mathfrak{m}$ and $l$ is annihilated by $\mathfrak{m}$ we have that $\Omega_{l|A} = \Omega_{l|k}$. Let $x$ be an element of $l$, and let $p$ be its irreducible polynomial over $k$. Now we want to use the fact that $k \subset l$ is a separable field extension. Remember that $k \subset l$ being seperable means that the formal derivative of $p$ is non-zero. Now we have that 
\begin{align*}
0 = d(p(x)) = p'(x)dx.
\end{align*}
Since $p'$ is a non-zero polynomial of lower degree than $p$, and $p$ is the smallest polynomial with root $x$, we must have that $p'(x)$ is non-zero. This implies that $dx=0$, and since this holds for all $x$ it must be that $\Omega_{l|k}=0$.

Since $\Omega_{l|k}=0$ we have that $\alpha$ is surjective. We will now use that since $A \to B$ is unramified $\mathfrak{n}B = \mathfrak{m}$. More specifically the map $\beta: \mathfrak{n}/\mathfrak{n}^2 \otimes_A B \to \mathfrak{m}/\mathfrak{m}^2$ is surjective. Since both $\alpha$ and $\beta$ are surjective we have that $\alpha\beta$ is also surjective, but 
\begin{align*}
\alpha\beta(\overline{n} \otimes b) = \alpha(\overline{nb}) = d(nb) \otimes 1 = ndb \otimes 1 = db \otimes n \cdot 1 = 0
\end{align*}
for all $n \in \mathfrak{n}$ and $b \in B$. Thus the only conclusion is that $\Omega_{B|A} \otimes_B l = 0$.

Since $\Omega_{B|A} \otimes_B l = \Omega_{B|A} \otimes_B B/\mathfrak{m} = \Omega_{B|A}/\mathfrak{m}\Omega_{B|A}$ it follows from Nakayama's lemma\footnote{The statement of Nakayama's lemma can be found on \href{https://stacks.math.columbia.edu/tag/07RC}{https://stacks.math.columbia.edu/tag/07RC}.} that $\Omega_{B|A}=0$.
\end{proof}
\end{prop}

\begin{theorem}
\label{thm:unramified_implies_seperable}
Let $A \to B$ be an unramified extension of local rings. Then the sequence
\begin{center}
\begin{tikzcd}
0 \ar[r] & I \ar[r] & B \otimes_A B \ar{r}{\mu} & B \ar[r] & 0
\end{tikzcd}
\end{center}
splits as a short exact sequence of $B\otimes_AB$-modules. Here $\mu$ is given by $\mu(b \otimes b') = bb'$, and the $B \otimes_A B$-module structure on $B$ is given by $b \otimes b' \cdot b'' = bb'b''$, and $I = \Ker \mu$. We say that the extension is separable

\begin{proof}
Firstly note that $I$ is generated by elements on the form $b \otimes 1 - 1 \otimes b$, and that since $B$ is fintiely generated as an $A$-module, $I$ is finitely generated.

Next we want to show that $I/I^2 = \Omega_{B|A}$, which we have already seen equals 0. Since $\Omega_{B|A}$ is a $B$-module we need a $B$-module structure on $I/I^2$. Since $(b \otimes 1)i - (1\otimes b)i$ is in $I^2$ for $i \in I$, we have that $(b \otimes 1)i = (1 \otimes b)i \mod I^2$. Then $I/I^2$ is generated by $(c \otimes 1 - 1 \otimes c)$ as a $B$-module with the $B$-module action given by $b \cdot i := (1 \otimes b)i$.

Now to see that $I/I^2 = \Omega_{B|A}$ we will show that the relations on $(b \otimes 1 - 1 \otimes b)$ in $I/I^2$ are exactly the same as those for $db$ in $\Omega_{B|A}$, thus that $(db \mapsto (b \otimes 1 - 1 \otimes b))$ is an isomorphism.

$A$-linearity follows from the fact that we are tensoring over $A$, that is
\begin{align*}
(ab \otimes 1 - 1 \otimes ab) &= (b \otimes a - 1 \otimes ab) =\\ 
(1 \otimes a)(b \otimes 1 - 1 \otimes b) &= a \cdot 
(b \otimes 1 - 1 \otimes b)
\end{align*}

The Leibniz rule $dbc - bdc - cdb = 0$ follows from a similar computation.
\begin{align*}
& \;(b \otimes 1 - 1 \otimes b)(c \otimes 1 - 1 \otimes c) \\
=&\; bc \otimes 1 - b \otimes c - c \otimes b + 1 \otimes bc \\
=&\; (bc \otimes 1 - 1 \otimes bc) - (c \otimes b - 1 \otimes bc) - (b \otimes c - 1 \otimes bc)\\
=&\; (bc \otimes 1 - 1 \otimes bc) - (1 \otimes b)(c \otimes 1 - 1 \otimes c) - (1 \otimes c)(b \otimes 1 - 1 \otimes b)\\
\end{align*}
and we see that $(bc \otimes 1 - 1 \otimes bc) - b \cdot (c \otimes 1 - 1 \otimes c) - c \cdot (b \otimes 1 - 1 \otimes b)$ generates $I^2$.

Now that we have shown that $I/I^2 = \Omega_{B|A} = 0$, or rather that $I = I^2$. Nakayama's lemma gives that there is an $i \in I$ such that $ji = j$ for all $j \in I$. Then we can define the splitting map $B \otimes_A B \to I$ by $b \otimes b' \mapsto b \otimes b' \cdot i$. Thus the sequence 
\begin{center}
\begin{tikzcd}
0 \ar[r] & I \ar[r] & B \otimes_A B \ar{r}{\mu} & B \ar[r] & 0
\end{tikzcd}
\end{center}
splits.
\end{proof}
\end{theorem}

\begin{theorem}
\label{thm:separable_implies_ringiso}
Let $B$ be a local $k$-algebra domain, and $G$ a finite subgroup of $\Aut_k(B)$ with order relatively prime to the characteristic of $k$, and denote by $A$ the fixed ring $B^G$. If the short exact sequence
\begin{center}
\begin{tikzcd}
0 \ar[r] & I \ar[r] & B \otimes_A B \ar{r}{\mu} & B \ar[r] & 0
\end{tikzcd}
\end{center}
splits, then the map
\begin{center}
\begin{tikzcd}
B\#G \ar[r]{}{\gamma} & \End_A(B)\\
b \cdot g \ar[mapsto, r] & (a \mapsto b \cdot a^g)
\end{tikzcd}
\end{center}
is an isomorphism of $A$-modules, and isomorphism of rings.

\begin{proof}
First in order to see that the map is injective, assume $b \cdot g$ and $b' \cdot g'$ map to the same endomorphism. Then $b \cdot t^g = b' \cdot t^{g'}$ for all $t \in B$. Choosing $t=1$ we see that $b = b'$. Then since $B$ is a domain this means that $t^g = t^{g'}$ for all $t$, that is to say $g = g'$.

To see that the map is surjective we will construct a splitting. The splitting will be constructed using the following diagram:
\begin{center}
\begin{tikzcd}
B\#G \ar[r]{}{\gamma} & \End_A(B) \ar[d]{}{f \mapsto f \otimes \rho}\\
B \otimes_A B\#G \ar{u}{\tilde{\mu}} & \Hom_B(B \otimes_A B, B \otimes_A B\#G) \ar[l]{}{ev_\epsilon}
\end{tikzcd}
\end{center}
where $\rho$ is the modified Reinolds-operator
\begin{center}
\begin{tikzcd}
\rho(b) = \sum\limits_{g \in G} b^g \cdot g.
\end{tikzcd}
\end{center}
Since we assumed the extension is unramified we have that
\begin{center}
\begin{tikzcd}
0 \ar[r] & I  \ar[r]{}{\iota} & B \otimes_A B \ar[r]{}{\mu} \ar[l, dotted, bend left = 30]{}{\psi} & B \ar[r] & 0
\end{tikzcd}
\end{center}
splits. As indicated we denote the left splitting by $\psi$. Then let $\epsilon = 1 \otimes 1 - \iota\psi(1 \otimes 1)$ in $B \otimes_A B$. Then $\mu(\epsilon) = 1$, and $(b \otimes 1 - 1 \otimes b)\epsilon = 0$. Then we define the evaluation map at $\epsilon$ by
\begin{center}
\begin{tikzcd}
ev_\epsilon: & \Hom_B(B \otimes_A B, B \otimes_A B\#G) \ar[r] & B \otimes_A B\#G\\
& f \ar[r, mapsto] & f(\epsilon)
\end{tikzcd}
\end{center}
Lastly $\tilde{\mu}: B \otimes_A B\#G \to B\#G$ is simply the map $b \otimes c \cdot g \mapsto bc \cdot g$. We have now defined all the maps in the square 
\begin{center}
\begin{tikzcd}
B\#G \ar[r]{}{\gamma} & \End_A(B) \ar[d]{}{f \mapsto f \otimes \rho}\\
B \otimes_A B\#G \ar{u}{\tilde{\mu}} & \Hom_B(B \otimes_A B, B \otimes_A B\#G) \ar[l]{}{ev_\epsilon}
\end{tikzcd}
\end{center}
Now we want to show that the composition of the three bottom maps forms a splitting. That is for any $f \in \End_A(B)$ we have that $\gamma(\tilde{\mu}(ev_\epsilon(f \otimes \rho))) = f$.

Write $\epsilon = \sum\limits_i x_i \otimes y_i$. Then I claim that 
\begin{align*}
\sum_i x_i y_i^g = \begin{cases}
1 & g = 1_G\\
0 & \text{otherwise}
\end{cases}
\end{align*}
We know that 
\begin{align*}
(b \otimes 1)\sum_i x_i \otimes y_i = (1 \otimes b)\sum_i x_i \otimes y_i
\end{align*}
holds for all $b$. Then applying the map $1 \otimes g$ on both sides we get
\begin{align*}
\sum_i bx_i \otimes y_i^g = \sum_i x_i \otimes b^gy_i^g
\end{align*}
Then by applying $\mu$ we get 
\begin{align*}
b\sum_i x_i y_i^g = b^g\sum_i x_i  y_i^g
\end{align*}
Then since $B$ is a domain we get that either $b = b^g$ or $\sum_i x_i  y_i^g = 0$. If we assume that $\sum_i x_i  y_i^g \neq 0$ then we must have that $b = b^g$ for all $b \in B$ and we then get that $g = 1_G$. Then since
\begin{align*}
\sum_i x_i  y_i = \mu(\epsilon) = 1
\end{align*}
we see that my claim holds. We can now calculate $\gamma(\tilde{\mu}(ev_\epsilon(f \otimes \rho)))$:
\begin{align*}
\gamma \left[ \tilde{\mu} \left[ (f \otimes \rho)(\epsilon) \right] \right] (b) &=\\ 
\gamma \left[ \tilde{\mu} \left[ (f \otimes \rho)(\sum_i x_i \otimes y_i) \right] \right] (b) &=\\
\gamma \left[ \tilde{\mu} \left[ \sum_i f(x_i) \otimes \rho(y_i) \right] \right] (b) &=\\
\gamma \left[ \sum_i f(x_i) \sum_g y_i^g \cdot g \right] (b) &=\\
\gamma \left[ \sum_g \sum_i f(x_i) y_i^g \cdot g \right] (b) &=\\
\sum_g \left(\sum_i f(x_i) y_i^g \cdot b^g \right) &\stackrel{\mathclap{\normalfont\mbox{*}}}{=}\\
f \left( \sum_g \left(\sum_i x_i y_i^g \right) \cdot b^g \right) &\stackrel{\mathclap{\normalfont\mbox{**}}}{=}\\
f(b) &
\end{align*}
In (*) we use the fact that $f$ is $A$-linear and that $\sum_g y_i^g b^g$ is in $A$. In (**) we use the claim from above that 
\begin{align*}
\sum_i x_i y_i^g = \begin{cases}
1 & g = 1_G\\
0 & \text{otherwise}
\end{cases}
\end{align*}
This means that $\gamma$ is an epimorphism and then also an isomorphism.
\end{proof}
\end{theorem}

\begin{defin}
Let $S$ be a commutative ring, $G$ a subgroup of $\Aut(S)$, and $\mathfrak{p}$ a prime ideal. The \underline{inertia group of $\mathfrak{p}$} is defined as
\begin{align*}
T(\mathfrak{p}) = \{ g \in G | s^g - s \in \mathfrak{p} \;\; \forall s \in S \}.
\end{align*}
\end{defin}

\begin{defin}
Let $S$ be a commutative ring, $G$ a subgroup of $\Aut(S)$, and $\mathfrak{p}$ a prime ideal. The \underline{decomposition group of $\mathfrak{p}$} is defined as
\begin{align*}
D(\mathfrak{p}) = \{ g \in G | g(\mathfrak{p}) = \mathfrak{p} \}.
\end{align*}
\end{defin}

\begin{lemma}
Let $S$ be the complex power series ring in $n$ variables, let $G$ be  a finite subgroup of $GL_n(\C)$ acting on $S$, and let $\mathfrak{p}$ be a height one prime ideal of $S$. Denote by $R$ the fixed ring $S^G$ and let $\mathfrak{q} = R \cap \mathfrak{p}$. Let $e$ be the rammification index of $R_\mathfrak{q} \subset S_\mathfrak{p}$, and let $f$ be the degree of the field extension $R_\mathfrak{q}/\mathfrak{q} \subset S_\mathfrak{p}/\mathfrak{p}$. Then the order of the decomposition group $|D(\mathfrak{p})|$ is $ef$.

\begin{proof}
Let $\{ \mathfrak{p}_i \}$ be the set of prime ideals in $S$ lying over $\mathfrak{q}$. The group $G$ acts on the set by permuting the ideals. We will show that this group action is transitive. Assume for the sake of contradiction the it is not and that there is a prime ideal $\mathfrak{p}_t$ such that $g(\mathfrak{p}) \neq \mathfrak{p}_t$ for all $g \in G$. Then by the {\color{red} approximation lemma?} we have that there is an $a \in \mathfrak{p}_t$ such that $a^g \not\in \mathfrak{p}$ for any $g \in G$. Now consider $x = \prod_{g \in G} a^g$. Clearly $x$ is in $R$ and thus in $\mathfrak{q}$, but since none of the factors of $x$ are in $\mathfrak{p}$ we must have that $x \not\in \mathfrak{p}$. This is a contradiction, thus the action of $G$ is transitive on $\{ \mathfrak{p}_i \}$.

The orbit-stabilizer theorem states that the size of an orbit is the same as the index of the stabilizer group. Note that $D(\mathfrak{p})$ is exactly the stabilzer of $\mathfrak{p}$. Then since $G$ acts transitivey we have that $|\{ \mathfrak{p}_i \}| = |G|/|D(\mathfrak{p})|$. In particular this set is finite, say $r := |\{ \mathfrak{p}_i \}|$, and $|G| = |D(\mathfrak{p})|\cdot r$.

Since the order of $G$ is $|G|=efr$ {\color{red} ??} it follows that $|D(\mathfrak{p})| = ef$.
\end{proof}
\end{lemma}

\begin{lemma}
Let $S$ be the complex power series ring in $n$ variables, let $G$ be  a finite subgroup of $GL_n(\C)$ acting on $S$, and let $\mathfrak{p}$ be a height one prime ideal of $S$. Denote by $R$ the fixed ring $S^G$ and let $\mathfrak{q} = R \cap \mathfrak{p}$. Then the ramification index of $R_\mathfrak{q} \subset S_\mathfrak{p}$, denotet by $e$, divides order of the inertia group $|T(\mathfrak{p})|$.

\begin{proof}
Since $\mathfrak{p}$ is invariant under the action of $D(\mathfrak{p})$, also its compliment will be. This means the group action on $S_\mathfrak{p}$ given by $g(\frac{s}{t}) = \frac{s^g}{t^g}$ is well defined whenever $g$ is in $D(\mathfrak{p})$. This also gives a well defined group action on $S_\mathfrak{p}/\mathfrak{p}S_\mathfrak{p}$. Since this action fixes $R_\mathfrak{q}/\mathfrak{q}R_\mathfrak{q}$ we get a map from $D(\mathfrak{p})$ to the galois group of the field extension $R_\mathfrak{q}/\mathfrak{q}R_\mathfrak{q} \subset S_\mathfrak{p}/\mathfrak{p}S_\mathfrak{p}$. If we can show that the kernel of this map is $T(\mathfrak{p})$, then we get that $|D(\mathfrak{p})/T(\mathfrak{p})|$ divides the order of the galois group which equals the order of the field extension. Then since $|D(\mathfrak{p})| = ef$ we get that $ef \Big| |T(\mathfrak{p})|f$, and that $e$ divides $T(\mathfrak{p})$.

First we see that $T(\mathfrak{p})$ is contained in the kernel. Since $\frac{s}{t} = \frac{\prod_{g \neq 1}t^g s}{\prod_g t^g}$ where $g$ ranges over the elements of $T(\mathfrak{p})$, we have that all fractions in $S_\mathfrak{p}$ can be written with a denominator invariant under $T(\mathfrak{p})$. Then since $s^g - s \in \mathfrak{p}$ whenever $s$ is in $S$ and $g$ is in $T(\mathfrak{p})$ we get that $\left(\frac{s}{t}\right)^g - \frac{s}{t} \in \mathfrak{p}S_\mathfrak{p}$ for all $\frac{s}{t} \in S_\mathfrak{p}$ and $g \in T(\mathfrak{p})$. Thus the action of $T(\mathfrak{p})$ on $S_\mathfrak{p}/\mathfrak{p}S_\mathfrak{p}$ is trivial.

To see the converse assume $g in D(\mathfrak{p})$ acts trivially on $S_\mathfrak{p}/\mathfrak{p}S_\mathfrak{p}$. Then in particular we have that $\frac{s^g}{1} - \frac{s}{1} \in \mathfrak{p}S_\mathfrak{p}$ for all $s \in S$. This means that $s^g - s \in \mathfrak{p}$, which is exactly the condition for $g$ to be in $T(\mathfrak{p})$.

This shows that $T(\mathfrak{p})$ is the kernel of the map, and thus that the rammification index divides the order of $T(\mathfrak{p})$. 
\end{proof}
\end{lemma}

\begin{theorem}
Let $S$ be the complex power series ring in $n$ variables, let $G$ be  a finite subgroup of $GL_n(\C)$ acting on $S$, and let $\mathfrak{p}$ be a height one prime ideal of $S$. Denote by $R$ the fixed ring $S^G$ and let $\mathfrak{q} = R \cap \mathfrak{p}$. Then the ramification index of $R_\mathfrak{q} \subset S_\mathfrak{p}$ equals the order of the inertia group $|T(\mathfrak{p})|$.

\begin{proof}
We write $\mathfrak{m}$ for the maximal ideal of $S$. Since $\mathfrak{p}$ is height one and $S$ is a UFD we have that $\mathfrak{p} = \langle z \rangle$ for some $z \in \mathfrak{m}$. We define an inner product on $V := \mathfrak{m}/\mathfrak{m}^2$ by
\begin{align*}
\langle x, y \rangle_G = \frac{1}{|G|} \sum_{g \in G} \langle x^g,  y^g \rangle
\end{align*}
\end{proof}
where $\langle -,- \rangle$ is the standard inner product. Note that the action of $G$ is orthogonal with respect to this inner product.

We write $\overline{z}$ for the representative for $z$ in $V$. Since the action of $G$ preserves degrees and that $\overline{z}^g - \overline{z} \in \langle \overline{z} \rangle$ we must have that $\overline{z}^g = a_g \cdot \overline{z}$ for some scalar $a_g \in \C$. Further since $x^g = x + \lambda_{g,x}\overline{z}$ for all $x \in V$ and $g in T(\mathfrak{p})$, and $g$ is an orthogonal operator we have that $g$ fixes the $\langle -,- \rangle_G$-orthogonal complement to $\overline{z}$. This means we can choose a basis such that all elements of $T(\mathfrak{p})$ are on the form:
\begin{align*}
\begin{pmatrix}
1\\
& 1\\
&& \ddots\\
&&&1\\
&&&& a_g
\end{pmatrix}
\end{align*}

This means $T(\mathfrak{p})$ is isomorphic to $\{ a_g \}_{g \in T(\mathfrak{p})} \leq \C^*$ which is a subgroup of $\C^*$. Since all finite subgroups of $\C^*$ are cyclic this implies that $T(\mathfrak{p})$ is cyclic. Let $s$ be the order of $T(\mathfrak{p})$. Then
\begin{align*}
\sigma :=
\begin{pmatrix}
1\\
& 1\\
&& \ddots\\
&&&1\\
&&&& \exp(2\pi i/s)
\end{pmatrix}
\end{align*}
generates $T(\mathfrak{p})$. Consider the ring $S^{T(\mathfrak{p})}$. We have that $R \subset S^{T(\mathfrak{p})}$, and $\mathfrak{q} \subset S^{T(\mathfrak{p})} \cap \mathfrak{p}$. Then we have that $R_\mathfrak{q} \subset S^{T(\mathfrak{p})}_{S^{T(\mathfrak{p})} \cap \mathfrak{p}}$, and the ramification index of $R_\mathfrak{q} \subset S_\mathfrak{p}$ is the product of the ramification index of $R_\mathfrak{q} \subset S^{T(\mathfrak{p})}_{S^{T(\mathfrak{p})} \cap \mathfrak{p}}$ and of $S^{T(\mathfrak{p})}_{S^{T(\mathfrak{p})} \cap \mathfrak{p}} \subset S_\mathfrak{p}$. Then since $(S^{T(\mathfrak{p})} \cap \mathfrak{p})S = z^s S = \langle z \rangle^s S$, we have that the ramification index of $R_\mathfrak{q} \subset S_\mathfrak{p}$ is divisable by the order of $T(\mathfrak{p})$. Since we have already seen that the ramification index divides $|T(\mathfrak{p})|$ this implies that $e = |T(\mathfrak{p})|$.
\end{theorem}

\begin{theorem}
\label{thm:unramified_pseudoreflections}
$R_\mathfrak{q} \subset S_\mathfrak{p}$ is unramified for all height one primes $\mathfrak{p}$ if and only if $G$ contains no pseudoreflections, that is a non-trivial element that fixes a codimension 1 subspace.
\begin{proof}
Firstly since we are working in characteristic 0, all field extensions are seperable, thus $R_\mathfrak{q}/\mathfrak{q} \subset S_\mathfrak{p}/\mathfrak{p}$ is seperable. Since $S$ is a rank $|G|$ $R$-module, $S_\mathfrak{p}$ will be a finitely generated $R_\mathfrak{q}$-module.

We know that elements of $T(\mathfrak{p})$ can be written on the form
\begin{align*}
\begin{pmatrix}
1\\
& 1\\
&& \ddots\\
&&&1\\
&&&& a_g
\end{pmatrix}.
\end{align*}
Since $G$ does not contain any pseudoreflections we must have that $a_g = 1$ and therefore $T(\mathfrak{p})$ is trivial and $|T(\mathfrak{p})| = 1$. That means that the ramification index of $R_\mathfrak{q} \subset S_\mathfrak{p}$ is 1, and the extension is unramified.
\end{proof}
\end{theorem}
Note that no finite subgroup of $SL_n(\C)$ contains pseduoreflections. In particular $R_\mathfrak{q} \subset S_\mathfrak{p}$ is unramified when $G$ is a finite subgroup of $SL_2(\C)$.

Now the last piece of the puzzle is to show that this implies that 
\begin{center}
\begin{tikzcd}
S \# G \ar[r]{}{\gamma} & \End_R(S)
\end{tikzcd}
\end{center}
is an isomorphism when $S = \C\llbracket x, y \rrbracket$, and $G$ is a finite subgroup of $SL_2(\C)$.

\begin{lemma}
\label{lem:height_one_iso}
Let $S$ be a local ring and let $M$ and $N$ be $S$-modules such that $\depth M_\mathfrak{p} \geq \min \{ 2, \height(\mathfrak{p}) \}$ and $\depth M_\mathfrak{p} \geq \min \{ 1, \height(\mathfrak{p}) \}$ for all prime ideals $\mathfrak{p}$\footnote{This is called Serre's criterion}. Let $f: M \to N$ be a monomorphism such that $f_\mathfrak{p}: M_\mathfrak{p} \to N_\mathfrak{p}$ is an epimorphism for all height one prime ideals. Then $f$ is an isomorphism.
\begin{proof}
Assume $f$ is not an epimorphism. Then $f$ has a cokernel $C \neq 0$, and we have a short exact sequence
\begin{center}
\begin{tikzcd}
0 \ar[r] & M \ar[r]{}{f} & N \ar[r] & C \ar[r] & 0
\end{tikzcd}
\end{center}
Now we choose $\mathfrak{p}$ to be the annihilator of a submodule $\langle c \rangle$ for some non-zero $c \in C$. We want to show that $\mathfrak{p}$ has height at least 2. If $\mathfrak{p}$ had height one then since $f_\mathfrak{p}$ is epi we would have that $C_\mathfrak{p} = 0$. This equivalent to saying that for every $c \in C$ there is some element $s \not\in \mathfrak{p}$ such that $sc = 0$. This is impossible since $\mathfrak{p}$ is the anniholator for some $c$, thus if $sc=0$ then $s$ is in $\mathfrak{p}$. The same argument works for a height 0 prime ideal since they are contained in height one prime ideals.

Thus $\mathfrak{p}$ has height at least 2 and $\depth M_\mathfrak{p} \geq 2$, $\depth N_\mathfrak{p} \geq 1$. Now we want to show that $C_\mathfrak{p}$ has depth 0, using regular sequences. Recall that the depth of a module is the length of the longest regular sequence. Since $\mathfrak{p}$ annihilates some $c \in C$ multiplication by $p \in \mathfrak{p}$ cannot be injective on $C_\mathfrak{p}$, because $\frac{c}{1}$ will be in the kernel. Multiplication by any element not in $\mathfrak{p}$ will be epimorphic since $s \cdot \frac{c}{s \cdot t} = \frac{c}{t}$, thus no regular sequence exist on $C_\mathfrak{p}$.

Now we consider the short exact sequence
\begin{center}
\begin{tikzcd}
0 \ar[r] & M_\mathfrak{p} \ar[r]{}{f_\mathfrak{p}} & N_\mathfrak{p} \ar[r] & C_\mathfrak{p} \ar[r] & 0
\end{tikzcd}
\end{center}
and take its long exact sequence of $\Ext_S(k, -)$ where $k$ is the residual field of $S$.
\begin{center}
\begin{tikzcd}
\cdots \ar[r] & \Hom_S(k, N_\mathfrak{p}) \ar[r] & \Hom_S(k, C_\mathfrak{p}) \ar[r] & \Ext^1_S(k, M_\mathfrak{p}) \ar[r] & \cdots
\end{tikzcd}
\end{center}
Since $\depth N_\mathfrak{p} \geq 1$ and $\depth M_\mathfrak{p} \geq 2$ we have that $\Hom_S(k, N_\mathfrak{p})$ and $\Ext^1_S(k, M_\mathfrak{p})$ is 0. Then by exactness we get that $\Hom_S(k, C_\mathfrak{p}) = 0$. This contradicts the fact that $\depth C_\mathfrak{p} = 0$, and thus our assumption that $C \neq 0$ is wrong. Therefore $f$ is an epimorphism and therefore also an isomorphism.
\end{proof}
\end{lemma}

\begin{theorem}
\label{thm:skew_group_algerba_equals_endomorphsim_ring}
Let $S = \C \llbracket x, y \rrbracket$ be the complex power series ring in two variables, let $G$ be  a fintie subgroup of $SL_2(\C)$ acting on $S$, and let $R = S^G$ be the fixed ring. Then the map
\begin{center}
\begin{tikzcd}
S \# G \ar[r]{}{\gamma} & \End_R(S)
\end{tikzcd}
\end{center}
is an isomorphism of rings.
\begin{proof}
Let $\mathfrak{q}$ be a height one prime ideal of $R$ and let $\mathfrak{p}$ be a prime ideal in $S$ lying over $\mathfrak{q}$. Then since $G$ is in $SL_2(\C)$ it can't contain any pseudoreflections, thus by \cref{thm:unramified_pseudoreflections} the extension $R_\mathfrak{q} \subset S_\mathfrak{p}$ is unramified. Then by \cref{thm:unramified_implies_seperable} we have that 
\begin{center}
\begin{tikzcd}
0 \ar[r] & I  \ar[r] & S_\mathfrak{p} \otimes_{R_\mathfrak{q}} S_\mathfrak{p}  \ar[r]{}{\mu} & S_\mathfrak{p}  \ar[r] & 0
\end{tikzcd}
\end{center}
is a split exact sequence. Since $S_\mathfrak{q} := R_\mathfrak{q} \otimes_R S$ lies between $R_\mathfrak{q}$ and $S_\mathfrak{p}$ we then have that 
\begin{center}
\begin{tikzcd}
0 \ar[r] & I  \ar[r] & S_\mathfrak{q} \otimes_{R_\mathfrak{q}} S_\mathfrak{q}  \ar[r]{}{\mu} & S_\mathfrak{q}  \ar[r] & 0
\end{tikzcd}
\end{center}
also splits {\color{red} ?????}. Then by \cref{thm:separable_implies_ringiso} we have that the map
\begin{equation}
\label{eq:localized_gamma}
\begin{tikzcd}
S_\mathfrak{q} \# G \ar[r]{}{\gamma} & \End_{R_\mathfrak{q}}(S_\mathfrak{q})
\end{tikzcd}
\end{equation}
is an isomorphism of rings. Now we want to show that (\ref{eq:localized_gamma}) is the localization of 
\begin{center}
\begin{tikzcd}
S \# G \ar[r]{}{\gamma} & \End_R(S).
\end{tikzcd}
\end{center}

That $S_\mathfrak{q} \# G$ is the localization of $S\#G$ is clear to see. What we need to show is that $\End_{R_\mathfrak{q}}(S_\mathfrak{q})$ is the localization of $\End_R(S)$, that is $R_\mathfrak{q} \otimes_R \End_R(S) \cong \End_{R_\mathfrak{q}}(S_\mathfrak{q})$. We can construct an explicit isomorphsim by
\begin{center}
\begin{tikzcd}
R_\mathfrak{q} \otimes_R \End_R(S) \ar[r] & \End_{R_\mathfrak{q}}(S_\mathfrak{q})\\
\frac{1}{r} \otimes \varphi \ar[r, mapsto] & (\frac{s}{t} \mapsto \frac{\varphi(s)}{rt})
\end{tikzcd}
\end{center}
It should be clear that the map is injective. To see surjectivity let $s_1, s_2, \cdots, s_n$ be generators of $S$ as an $R$-module, and let $\psi$ be in $\End_{R_\mathfrak{q}}(S_\mathfrak{q})$. Write $\frac{t_i}{r_i}$ for $\psi(s_i)$, and let $\varphi$ be the map in $\End_R(S)$ sending $s_i$ to $t_i\prod_{j \neq i} r_j$. Then $\frac{1}{\prod r_i} \otimes \varphi$ is a pre-image of $\psi$, and thus $R_\mathfrak{q} \otimes_R \End_R(S) \cong \End_{R_\mathfrak{q}}(S_\mathfrak{q})$.

This means that for each height one prime of $R$, the localization of $\gamma$ is an isomorphsim. Then \cref{lem:height_one_iso} reduces the problem of showing that $\gamma$ is an isomorphsim to showing that $\depth_R S_\mathfrak{q}\#G \geq \min \{ 2, \height(\mathfrak{q}) \}$, and that $\depth_R \End_{R_\mathfrak{q}}(S_\mathfrak{q}) \geq \min \{ 1, \height(\mathfrak{p}) \}$.
{\color{red} not sure how to show this}

\iffalse
Firstly let's show that for a prime ideal $\mathfrak{p}$ we have that $S_\mathfrak{p}^G = R_\mathfrak{q}$ where $\mathfrak{q} = R \cap \mathfrak{p}$. Assume $\frac{s}{p} \in S_\mathfrak{p}^G$ is fixed by $G$. Consider the fraction
\begin{align*}
\frac{ \left( \prod_{g \neq 1} p^g \right) s}{\prod_g p^g}
\end{align*}
Since we have just multiplied by $\prod_{g \neq 1} p^g$ in the nominator and the denominator it still equals $\frac{s}{p}$. The bottom is obviously fixed by $g$, but why is it not in $\mathfrak{q}$??? 
{\color{red} How do I know $p^g$ is not in $\mathfrak{p}$ ???? You localize using the complement of the prime ideal right??} Then since the denominator is fixed and the fraction as a whole is fixed this implies that the nominator is fixed as well.

Secondly I want to show that $\End_R(S)_\mathfrak{p} = \End_{R_\mathfrak{q}}(S_\mathfrak{p})$. {\color{red} ???}

From here we just need to wrap everything together. Since $G$ does not contain any pseudoreflections we get from \cref{thm:unramified_pseudoreflections} that the map is an isomoprhism when localizing at any height one prime ideal. Then \cref{lem:height_one_iso} gives us that it's an isomophism {\color{red} need to show depth}.
\fi
\end{proof}
\end{theorem}

\section{Maximal Cohen-Macaulay modules of $S^G$}
\iffalse
\begin{defin}
If $R$ is a local ring with residual field $k$ we define the \underline{depth} of a module, $M$, to be the minimal $n$ such that $\Ext^n_R(k, M)$ is non-zero. We write $\depth_R(M)$ fro this or simply $\depth(M)$ when which ring we are using is clear.
\end{defin}

\begin{defin}
\label{def:regular_seq}
If $R$ is a commutative ring and $M$ is an $R$-module, an \underline{$R$-regular sequence on $M$} is a sequence of elements of $R$, $r_1, r_2, \cdots r_n$ such that $M/\langle r_1, \cdots, r_i \rangle M$ is non-zero and multiplication by $r_i$ is injective on $M/\langle r_1, \cdots, r_{i-1} \rangle M$.
\end{defin}

\begin{prop}
The depth of a module equals the lenth of the longest regular sequence on that module.
\begin{proof}
{\color{red} reffrence}
\end{proof}
\end{prop}

\begin{defin}
If $R$ is a ring, we say that $\mathfrak{p}$ is  a \underline{prime ideal} in $R$ if
\begin{enumerate}
\item $\mathfrak{p}$ is a proper ideal of $R$.
\item For any two elements $a,b \in R$ such that $ab \in \mathfrak{p}$ we must have that either $a$ is in $\mathfrak{p}$ or $b$ is.
\end{enumerate}
\end{defin}
\fi

\begin{defin}
If $R$ is a ring we define its \underline{Krull dimension} to be the maximum length of a chain of prime ideals in $R$. 
\end{defin}
\begin{example}
For example the power series ring $\C\llbracket x_1, \cdots, x_n \rrbracket$ has Krull dimension $n$ given by the chain
\begin{equation*}
\begin{tikzcd}
0 \subseteq \langle x_1 \rangle \subseteq \langle x_1, x_2 \rangle \subseteq \cdots \subseteq \langle x_1, \cdots, x_n \rangle
\end{tikzcd}
\end{equation*}
\end{example}

\begin{defin}
If $M$ is a module over a local ring $R$ with Krull dimension $d$ we say that $M$ is \underline{maximal Cohen Macaulay (MCM)} if the depth of $M$ equals $d$. If a ring is MCM as a module over itself we say that it's a \underline{Cohen Macaulay ring (CM)}. Note that the ring must be CM for it to have any MCM modules. 
\end{defin}

\begin{theorem}
If $G$ is a finite subgroup of $GL_n(\C)$, $S$ is the complex power series ring in $n$ variables and $R = S^G$ is the ring fixed under the action of $G$, then $R$ is a direct summand of $S$ as $R$-modules.

\begin{proof}
Consider the map $\pi: S \to R$ given by
\begin{align*}
\pi(s) = \frac{1}{|G|} \sum_{g\in G} s^g
\end{align*}
It's clear that the image of $\pi$ is in $R$ because an action from $G$ will just permute the order of the sum. Further \begin{align*}
\pi(r) = \frac{1}{|G|} \sum_{g\in G} r^g = \frac{1}{|G|} \sum_{g\in G} r = r,
\end{align*}
so $\pi$ splits the inclusion $R \hookrightarrow S$ which shows that $R$ is a direct summand of $S$.
\end{proof}
\end{theorem}

Since we are interested in the MCM modules that appear as $R$-direct summands of $S$, it is nice to see that $R$ itself is such a module.

\begin{prop}
\label{prop:direct_summand_MCM}
Let $R$ be a local ring with depth of $R$ equaling it's Krull dimension, that is $R$ is CM. If $M$ is an MCM $R$-module, and $N$ is a direct summand of $M$ then $N$ is also MCM.
\begin{proof}
We write $M$ as $N \oplus X$. Since $M$ is MCM we have that $0 = \Ext^i_R(M) = \Ext^i_R(N) \oplus \Ext^i_R(X)$ for all $i$ less than the Krull dimension of $R$. This means the depth of $N$ is greater than or equal to the Krull dimension of $R$. Since the depth of a module cannot exceed the Krull dimension of the ring {\color{red} refference} we have that $N$ is MCM.
\end{proof}
\end{prop}

\begin{lemma}
\label{lem:depth_of_S_less_than_R}
If $S$ is a commutative local ring with finite depth, $G$ a finite subgroup of $\Aut(S)$, and $R=S^G$ is the fixed ring, then $\depth_SS \leq \depth_RR$, and $\depth_SS = \depth_RS$.
\begin{proof}
Let $N(s) = \prod_{g\in G} s^g$ be a map from $S$ to $S$. Note that for any $s \in S$ we have that $N(s) \in R$. Further if $s$ is a non-unit and not a zero divisor, then $N(s)$ will also be a non-unit and not a zero divisor. Another property of $N(s)$ we will use is that $\left(S/N(s)S\right)^G = R/N(s)R$ whenever $N(s)$ is a not a zero divisor. To see this note that since $(N(s)S)^G \subset N(s)S$, we have that $(s + N(s)S)^g = s^g + N(s)S$, and thus $\left(S/N(s)S\right)^G = S/\left(N(s)S \cap R \right)$. Then what's left to show is that $N(s)S \cap R = N(s)R$. Assume $N(s)t$ is in $N(s)S\cap R$. Then $(N(s)t)^g = N(s)t$ for all $g \in G$. Since $N(s)$ is in $R$ we have that $(N(s)t)^g = N(s)t^g$. Now using the fact that $N(s)$ is not a zero divisor we get that $t^g = t$ for all $g \in G$, which means that $t$ is in $R$. Now the rest of the proof will be induction on $\depth_SS$.

For the base case let $\depth_SS = 0$. Since depth is alwasy bigger than or equal to 0 we have that $\depth_SS \leq \depth_RR$. Also since $\depth_RS \leq \depth_SS$ we have that $\depth_SS = \depth_RS=0$.

Assume the statement holds whenever $\depth_SS < n$. Then we want to show that it holds when $\depth_SS = n$. Since the $\depth_SS > 0$ there exists a non-unit which is not a zero divisor $s \in S$. Then $N(s)$ is also a non-unit and not a zero divisor. This means $N(s)$ is the first element of some maximal regular sequence on $S$. Since all maximal regular sequences on $S$ has the same length this means that $\depth_S S/N(S)S = \depth_SS -1$. Further since $N(s)$ is in $R$ the same argument gives that $\depth_R S/N(s)S = \depth_RS -1$. Since every element in $N(s)S$ annihilates $S/N(s)S$ we can consider $S$-regular sequecnes on $S/N(s)S$ as $S/N(s)$-regular sequences and vice versa. Thus $\depth_S S/N(s)S = \depth_{S/N(s)S} S/N(s)S$. Using that $\left( S/N(s)S \right)^G = R/N(s)R$, the same argument gives that $\depth_R S/N(s)S = \depth_{R/N(s)R} S/N(s)S$. Now we apply the induction hypothesis. Sicne $S/N(s)S$ is a local ring with depth less than $n$ we have that $\depth_{S/N(s)S} S/N(s)S \leq \depth_{R/N(s)R} R/N(s)R$ and that $\depth_{S/N(s)S} S/N(s)S = \depth_{R/N(s)R} S/N(s)S$. This means that $\depth_S S -1 \leq \depth_RR -1$ and that $\depth_S S - 1 = \depth_R S -1$. Thus we can conclude that $\depth_SS \leq \depth_RR$ and $\depth_SS = \depth_RS$. By induction the statement holds for all $n$.
\end{proof}
\end{lemma}

Now that we have shown a general relationship between the depths of local rings fixed under group actions we can apply it to our special case of $S = \C\llbracket x, y \rrbracket$ and $G$ being in $SL_2(\C)$.

\begin{theorem}
\label{thm:S_MCM}
Let $S$ be the complex power series ring in two variables, $G$ a finite subgroup of $SL_2(\C)$ acting on $S$ by linear change of variables, and $R=S^G$ the fixed ring. Then $R$ is a CM and $S$ is an MCM $R$-module.
\begin{proof}
Since $S$ is the complex power series ring in two variables we have that $\dim S = \depth_SS=2$. By \cref{lem:depth_of_S_less_than_R} we have that $\depth_SS \leq \depth_RR$. Since the depth of a module never exceeds the Krull dimension of the ring {\color{red}refference} we have that $\depth_SR \leq \dim S$. Lastly we saw in the first section that $R$ always equals $\C\llbracket u, v, w \rrbracket/\langle f \rangle$ for an irreducible polynomial $f$. Thus we have that $\dim R = 2 = \dim S$. Putting this together we get
\begin{align*}
\dim R = \dim S = \depth_SS \leq \depth_RR \leq \dim_R
\end{align*}
and thus $dim R = \depth_RR$, which means $R$ is CM. \cref{lem:depth_of_S_less_than_R} also gives us that $\depth_SS = \depth_RS$, which means that $\depth_RS = \dim S = \dim R$, so $S$ is MCM.
\end{proof}
\end{theorem}

Note that \cref{prop:direct_summand_MCM} gives that all the $R$-direct summands of $S$ are MCM $R$-modules. This holds true if we let $G$ be any finite subgroup of $GL_n(\C)$ which does not contain pseudoreflections. An interesting fact that I will not prove here is that when $G$ is a finite subgroup of $SL_2(\C)$ the converse holds. That is, all MCM $R$-modules can be written as the sum of $R$-direct summands of $S$.

\begin{theorem}
Let $S$ be the complex power series ring in two variables, $G$ a finite subgroup of $SL_2(\C)$ acting on $S$ by linear change of variables, and $R=S^G$ the fixed ring. Then there is a one-to-one correspondance between the finitely generated indecomposable projective $S\#G$-modules and the indecomposable MCM $R$-modules appearing as $R$-direct summands of $S$.

\begin{proof}
Let $\mathfrak{m}$ denote the maximal ideal of $S$. We saw in \cref{thm:indec_proj_SG=indec_CG} that theres a one-to-one correspondance between the finitely generated projective $S\#G$-modules and the finitely generated $\C G$-modules. In particular we get that $\S\#G \cong S \otimes_\C S\#G/\mathfrak{m}S\#G$. Since $S\#G/\mathfrak{m}S\#G \cong \C G$ and all indecomposable $\C G$ modules appear as a direct summand of $\C G$ we have that all finitely generated projectvie $S\#G$-modules appear as a direct summand of $S\#G$. The direct summands of $S\#G$ are exactly the primitive idempotents of $\End_{S\#G}(S\#G) \cong S\#G^{op}$. Since $S\#G$ is ismorphic to $S\#G^{op}$ by the isomorphism $(s\cdot g \mapsto s^{g^{-1}} \cdot g^{-1})$, we have that the finitely generated project $S\#G$-modules correspond to primitive idempotents of $S\#G$.

In \cref{thm:skew_group_algerba_equals_endomorphsim_ring} we showed that $S\#G \cong \End_R(S)$. Thus the primitive idempotents of $S\#G$ corresponds to the primitve idempotents of $\End_R(S)$ which are exactly the projections onto the indecomposable $R$-direct summands of $S$. Thus the finitely generated indecomposable projective $S\#G$-modules are in one-to-one correpsondance with the indecomposable $R$-direct summands of $S$. Since we showed in \cref{thm:S_MCM} that $S$ is MCM it follows from \cref{prop:direct_summand_MCM} that the direct summands of $S$ also are MCM.
\end{proof}
\end{theorem}

This last theorem concludes the part of the McKay correspondance that I will prove in this thesis. Another interesting part of the thesis I have not shown is that the Auslander-Reiten quiver of the indecomposable MCM modules appearing as an $R$-direct summand of $S$ is isomorphic to the Gabriel quiver. If the reader is interested they can read further in {\color{red} refferences and further readings}.

\begin{appendices}
\section{Representation theory}
\label{appendix}

\begin{defin}
If $R$ is a ring and $M$ is an abelian group, we define a \underline{representation of $R$} to be a ring-map, $\varphi$, from $R$ to $\End(M)$. Then we say that $M$ is a (left) \underline{$R$-module}, and we write $rm$ with $r \in R$ and $m \in M$ to mean $\varphi(r)(m)$. Similarly we define a right $R$-module if $\varphi$ goes from $R$ to $\End(M)^{op}$ and we write $mr$ for $\varphi(r)(m)$.
\end{defin}

\begin{defin}
If $G$ is a group and $V$ a complex vector space, we define a \underline{representation of $G$} to be a group-map, $\rho$, from $G$ to $\Aut_\C(V)$. When $\rho$ is infered we say that $V$ is a representation of $G$ and we write $gv$ to mean $\rho(g)(v)$. Note that representations of $G$ exactly corresponds to representations of the ring $\C G$ of formal linear combinations of elements of $G$ with multiplication given by $\lambda g \cdot \lambda' g' = (\lambda \cdot \lambda')gg'$.
\end{defin}

\begin{defin}
\label{def:indecomposable}
If $R$ is a ring and $M_1$ and $M_2$ are two modules we define their \underline{direct sum}, $M_1 \oplus M_2$ to be the module consisting of all pairs $(m_1, m_2)$ (usually written $m_1 + m_2$), where addition and scalar multiplication is pointwise. If a non-zero module cannot be written as the direct sum of two non-zero modules we call it \underline{indecomposable}.
\end{defin}

\begin{defin}
\label{def:simple}
A \underline{submodule} is a subset of a module which is also a module. A non-zero module with no non-trivial proper submodules is called \underline{simple} or \underline{irreducible}\footnote{The word simple is used for representations of rings while irreducible is used for representations of groups. Note that for finite groups irreducible and indecomposable are equivalent.}.
\end{defin}

\begin{theorem}
\label{schur}
(Schur's Lemma) Let $G$ be a group and $V$ and $W$ be two irreducible representations of $G$. If $f:V \to W$ is a $G$-linear map then $f$ is a 0 if $V$ and $W$ are not isomorphic, and a scaling of identity (up to change of basis) if they are isomorphic.
\begin{proof}
Start by assuming $f$ is non-zero. Then we will show that $V$ and $W$ are isomorphic. Since the image of $f$ is a non-zero subrepresentation of $W$ and $W$ is irreducible, we have that $\Image f = W$ and $f$ is surjective. Since the kernel of $f$ is a proper subrepresentation of $V$ we must have that the kernel is 0, and that $f$ is injective. Thus $f$ is an isomorphism.
Now assume $f: V \to V$ is a $G$-linear map, then we want to show that $f$ is simply a scaling of identity. Since $f$ is a linear map on a complex vector space it must have at least one eigen value, say $\lambda \in \C$. Let $v$ be in the eigenspace $\lambda$. Sicne $f(gv) = g f(v) = \lambda gv$ for all $g$ in $G$ we have that $gv$ is also in the eigenspace. This means the eigenspace is a subrepresentation, and since $V$ is irreducible it must equal all of $V$. This means that $f$ is just scaling by $\lambda$.
\end{proof}
\end{theorem}

\begin{defin}
\label{def:exact_functor}
We call a functor \underline{left exact} if for any short exact sequence
\begin{center}
\begin{tikzcd}
0 \arrow{r} & A \arrow{r}{f} & B \arrow{r}{g} & C
\end{tikzcd}
\end{center}
the image of the sequence under the functor is also exact. For example for any module $M$ the functor $\Hom(M, -)$ is left exact. That is the sequnce
\begin{center}
\begin{tikzcd}
0 \arrow{r} & \Hom(M,A) \arrow{r}{f \circ -} & \Hom(M,B) \arrow{r}{g \circ -} & C
\end{tikzcd}
\end{center}
is exact. Dually we call a functor \underline{right exact} if short exact sequences of the form
\begin{center}
\begin{tikzcd}
A \arrow{r} & B \arrow{r} & C \arrow{r} & 0
\end{tikzcd}
\end{center}
is mapped to an exact sequence. A functor that is both left exact and right exact is called \underline{exact}.
\end{defin}

\begin{defin}
\label{def:projective}
We say that a module, $P$, is \underline{projective} if for any epimorphism $f: M \twoheadrightarrow N$, and any map $g: P \to N$, there is a map $\varphi: P \to M$ such that $f \varphi = g$. Said another way, the diagram below induces the dotted arrow making the diagram commute
\begin{center}
\begin{tikzcd}
& P \arrow{d}{g} \arrow[dl, dotted, swap]{}{\varphi} \\
M \arrow[twoheadrightarrow, swap]{r}{f} & N
\end{tikzcd}
\end{center} 
Note that $P$ being projective is equivalent to $\Hom(P, -)$ being right exact (i.e. exact).
\end{defin}

\begin{defin}
\label{def:projective_cover}
If $M$ is a module and $f: P \to M$ is a homomorphism we say that $f$ is a \underline{projective cover} of $M$ if
\begin{itemize}
\item $P$ is projective
\item $f$ is an epimorphism
\item For any homomorphism $g: X \to P$, if $f \circ g$ is an epimorphsim then $g$ is an epimorphsim.
\end{itemize}
\end{defin}
The last condition for a projective cover is equivalent to the kernel of $f$ being \underline{small}. That is for any submodule $X$ of $P$, if $X + \Ker f = P$ then $X=P$. For a module $M$ the choice of $P$ in the projective cover is unique (but the choice of $f$ might not be). Therefore it is normal to refer to $P$ as \textit{the} projective cover of $M$. 

%indec proj = summand of ring = idempotent of ring

\begin{defin}
\label{def:projective_resolution}
If $M$ is a module we say that a \underline{projective resolution} of $M$ is a sequence
\begin{center}
\begin{tikzcd}
\cdots \ar{r}{\partial_2} & P_2 \ar{r}{\partial_1} & P_1 \ar{r}{\partial_0} & P_0 \ar[r] & 0
\end{tikzcd}
\end{center}
such that $P_i$ is projective for all $i$, the sequence is exact around every $P_i$ for $i \geq 1$, and that $\Cok \partial_0 = M$.

We call a projective resolution \underline{minimal} if each map $\partial_i$ as well as the cokernel map $P_0 \to M$ is a projective cover of its image. In a minimal resolution the objects $P_i$ are uniquely determined.
\end{defin}

\begin{defin}
\label{def:Ext}
If $M$ and $N$ are modules then the $i$th extension group, $\Ext^i(M, N)$, is constructed in the following way.
\begin{itemize}
\item Take a projective resolution of $M$
\begin{center}
\begin{tikzcd}
\cdots \ar{r}{\partial_2} & P_2 \ar{r}{\partial_1} & P_1 \ar{r}{\partial_0} & P_0 \ar[r] & 0
\end{tikzcd}
\end{center}
\item Apply $\Hom(-,N)$
\begin{center}
\begin{tikzcd}
0 \ar[r] & \Hom(P_0, N) \ar{r}{- \circ \partial_0} & \Hom(P_1, N) \ar{r}{- \circ \partial_1} & \Hom(P_2, N) \ar{r}{- \circ \partial_2} & \cdots 
\end{tikzcd}
\end{center}
\item Now $\Ext^i(M, N)$ is the homology at position $-i$, that is $\Ext^i(M,N) = \Ker (- \circ \partial_{i}) / \Image (- \circ \partial_{i-1})$.
\end{itemize}
\end{defin}

Note that the defintion of $\Ext^i(M, N)$ is independent of choice of projective resolution. An important property of $\Ext$ that is used in this thesis is the long exact sequence in $\Ext$-groups.

\begin{theorem}
For any module $M$ and short exact sequence of modules:
\begin{center}
\begin{tikzcd}
0 \ar[r] & A \ar[r] & B \ar[r] & C \ar[r] & 0
\end{tikzcd}
\end{center}
there are long exact sequences:
\begin{center}
\begin{tikzcd}
0 \ar[r] & \Hom(M, A) \ar[r] & \Hom(M, B) \ar[r] & \Hom(M, C) \ar[llld, overlay, out=-10, in=170] \\
\Ext^1(M, A) \ar[r] & \Ext^1(M, B) \ar[r] &\Ext^1(M, C) \ar[r] & \Ext^2(M, A) \ar[r] & \cdots
\end{tikzcd}
\end{center}
and
\begin{center}
\begin{tikzcd}
0 \ar[r, <-] & \Hom(A, M) \ar[r, <-] & \Hom(B, M) \ar[r, <-] & \Hom(C, M) \ar[llld, <-, overlay, out=-10, in=170] \\
\Ext^1(A, M) \ar[r, <-] & \Ext^1(B, M) \ar[r, <-] &\Ext^1(C, M) \ar[r, <-] & \Ext^2(A, M) \ar[r, <-] & \cdots
\end{tikzcd}
\end{center}
\end{theorem}

\end{appendices}

\section{Disposisjon}
Define McKay quiver [check]

Define $S\#G$ [check]

Correspondance with projectives [put in finitely generated to fix argument]

Gabriel Quiver [Make Koszul complex a refference]

$\End_R(S) \cong S\#G$ [understand the proof]
$q \in S$ height one prime implies $q = (f)$ for a homogenous polynomial? Why homogenous?
If $T(q)$ is nontrivial then it acts non-trivially on $S/qS$ if $f$ has degree bigger than 1, then all degree 1 polynomials survive in $S/qS$ and are acted upon trivialy by $T(q)$. Therefore $T(q)$ would be trivial, so $f$ is homogenous of degree 1.
Since the group operations preseve degree $\sigma(f) = a_\sigma f$ for a nonzero constant $a_\sigma$. All finite matrix groups diagonalizeable implies $\sigma = diag(1,1, \cdots a_\sigma)$. Therefore $T(q)$ is iso to finite subgroup of $\C^*$, hence cyclic.
Then $p=q \cap R = (f^n)$ where $n$ is the order of $T(q)$. Thus $q = pS_q$ if and only if $T(q)$ is trivial.
$I/I^2 = \Omega_{S|R}$ is 0 iff $pS = q$, $I/I^2 = 0$ implies idempotent implies spliting.
Why is $\End_R(S)$ reflexive? or rather why does height one iso imply iso.

MCM $R$-summands of $S$ 
[$S$ is MCM using dimension argument, summands are MCM using depth $\leq$ dim]

\nocite{*}
\bibliography{mybib}
\bibliographystyle{apalike}

\end{document}