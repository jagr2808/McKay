\documentclass[11pt, a4paper, english]{article}
\usepackage[utf8]{inputenc}
\usepackage{babel, amsmath, amsthm, amssymb, amsfonts, enumitem, mathtools, centernot}
\usepackage{tikz-cd}
\usepackage{stmaryrd}
\usepackage{cite}
\usepackage{hyperref}
\usepackage{cleveref}
\usepackage[toc,page]{appendix}
\newcommand\tab[1][1cm]{\hspace*{1}}
\DeclarePairedDelimiter{\ceil}{\lceil}{\rceil}

\newtheorem{prop}{Proposition}
\numberwithin{prop}{section}
\newtheorem{lemma}{Lemma}
\numberwithin{lemma}{section}
\newtheorem{theorem}{Theorem}
\numberwithin{theorem}{section}
\newtheorem{defin}{Definition}
\numberwithin{defin}{section}
\newtheorem{example}{Example}
\numberwithin{example}{section}

\newcommand{\C}{\mathbb{C}}
\newcommand{\Z}{\mathbb{Z}}
\DeclareMathOperator{\Hom}{Hom}
\DeclareMathOperator{\Ext}{Ext}
\DeclareMathOperator{\Tor}{Tor}
\DeclareMathOperator{\End}{End}
\DeclareMathOperator{\Aut}{Aut}
\DeclareMathOperator{\Image}{Im}
\DeclareMathOperator{\Ker}{Ker}
\DeclareMathOperator{\Cok}{Cok}

\begin{document}
\title{McKay correspondence}
\author{Jacob Fjeld Grevstad}
\maketitle

\begin{abstract}
The goal of this thesis is to establish a 1-1 correspondence between quivers created from the four following sets whenever $S$ is the power series ring $\C \llbracket x, y \rrbracket$ and $G$ is a finite subgroup of $SL(2,\C)$ acting on $S$
\begin{itemize}
\item The Maximal Cohen-Macaulay modules of the fixed ring $S^G$.
\item The indecomposable projective modules of the skew group algebra $S\#G$.
\item The indecomposable projective modules of $\End_{S^G}(S)$.
\item The irreducible representations of $G$ (indecomposable $\C G$-modules).
\end{itemize}
Much of the thesis will be used to define these four quivers and to develope tools to establish such a correspondence. A similar correspondence can be established for a general field $k$ and a finite subgroup of $GL(n, k)$ with order nonzero in $k$, but in the general case we will only attain the MCM-modules that apear as $S^G$-direct summands of $S$. $SL(2, \C)$ is also especially interesting because the quivers are exactly the Dynkin diagrams.
\end{abstract}

\tableofcontents

\iffalse
\section{Finite subgroups of SL(2,C)}

\section{Characters and irreducible representations}
This section is largely based on the book by \cite{RCG}.

Recall that the trace of a matrix is defined to be the sum of its diagonal elements and that the trace satisfies two important equations. Namely
$$tr(A+B)=tr(A)+tr(B) \text{  and  } tr(AB)=tr(BA)$$
For a given representation of $G$, $\rho: G \to GL_n(\mathbb{C})$, we define its character by $\chi_\rho : G \to \mathbb{C}$, $\chi_\rho(g) = tr(\rho(g))$.

\begin{prop}
Conjugate elements in $G$ take the same value under a character.
\begin{proof}
Let $g$ and $g'$ be in the same conjugacy class. Then there exists an element $h$ such that $h^{-1}gh=g'$. Then we have
\begin{equation*}
\begin{split}
\chi(g')=\chi(h^{-1}gh) = tr(\rho(h)^{-1}\rho(g)\rho(h)) \stackrel{\mathclap{\normalfont\mbox{\tiny{*}}}}{=} tr(\rho(g)\rho(h)\rho(h)^{-1}) = tr(\rho(g))=\chi(g)
\end{split}
\end{equation*}
In (*) we use the fact that $tr(AB)=tr(BA)$.
\end{proof}
\end{prop}

\begin{lemma}
For a finite abelian group $G$ any irreducible representation must be 1-dimensional.
\begin{proof}
Let $\rho: G \to GL(V)$ be an irreducible representation. Since $G$ is abelian we have that $\rho(g)\rho(h)v = \rho(h)\rho(g)v$. Thus multiplication by $\rho(g)$ respects the action of $G$ and we have that $\rho(g)$ is a homomorphism of $G$-representations between $\rho$ and itself. Then by Schur's lemma\footnote{Statement and proof of Schur's lemma can be found in the appendix on page \pageref{schur} as \cref{schur}.} $\rho(g)$ must be a scalar multiplication. In other words every matrix $\rho(g)$ for $g \in G$ is diagonal (it is a scaling of identity). This implies that $\rho$ can be written as a direct sum of 1-dimensional representations, but since $\rho$ is irreducible $\rho$ must be 1-dimensional
\end{proof}
\end{lemma}

\begin{prop}
If $\chi$ is the character of a representation, $\rho$, with dimension $n$ of a group $G$, and $g$ is an element of $G$ with order $m$, then the following holds
\begin{itemize}
 \item[(1)] $\chi(1) = n$
 \item[(2)] $\chi(g)$ is the sum of $m$-th roots of unity.
 \item[(3)] $\chi(g^{-1}) = \overline{\chi(g)}$
\end{itemize}
\begin{proof}

\begin{itemize}
\item[]
\item[(1)]
The first result is immediate.
$$\chi(1) = tr\left(\begin{pmatrix}
1 & \cdots & 0\\
\vdots & \ddots & \vdots\\
0 & \cdots & 1
\end{pmatrix}\right) = n$$
\item[(2)]
Since $\langle g \rangle$ is an abelian group, $\rho$ decomposes into $n$ 1-dimensional $\langle g \rangle$-representations. Then there is a basis such that $\rho(g)$ is diagonal. Since $g$ has order $m$ it follows that the diagonal entries of $\rho(g)$ must be $m$-th roots of unity. Thus $\chi(g) = tr(\rho(g))$ must be the sum of $m$-th roots of unity.
\item[(3)]
Using the same basis as above and the fact that $\omega^{-1} = \overline{\omega}$ when $\omega$ is a root of unity we see that $\chi(g^{-1}) = tr(\rho(g)^{-1}) = \overline{tr(\rho(g))} = \overline{\chi(g)}$.
\end{itemize}
\end{proof}
\end{prop}

\fi

\section{The McKay quiver}
\begin{defin}
Let $G$ be a finite subgroup of $GL(n, \C)$, and let $V$ be the canonical representation (the one that sends $g$ to $g$). Then we define the \underline{McKay quiver} of $G$ to be the quiver with verticies the irreducible representations of $G$, denoted $V_i$. For two irreducible representations $V_i$ and $V_j$ there is an arrow from the former to the latter if and only if $V_j$ is a direct summand of $V \otimes V_i$. 
\end{defin}

\begin{example}
Let $G$ be the group generated by $g =\begin{pmatrix}
\omega^2 & 0\\
0 & \omega^{3}
\end{pmatrix}$, where $\omega$ is a primitive fifth root of unity. Then there are five different irreducible representations, the one sending $g$ to $\omega$, $\omega^2$, $\omega^3$, $\omega^4$ respectively, and the trivial representation. Denote the representation sending $g$ to $\omega^i$ by $V_i$, and let $V = V_2 \oplus V_3$ be the cannonical representation. Note that $V_i \otimes V_j = V_{i+j}$, where $i+j$ is understood to be modulo 5. Then we get the following McKay-quiver
$$
\begin{tikzcd}
& V_0 \arrow[<->]{ddr} \arrow[<->]{ddl} & \\
V_4 && V_1 \arrow[<->]{ll}\arrow[<->]{dll}\\
V_3 && V_2 \arrow[<->]{ull}
\end{tikzcd}  
$$
\end{example}

\iffalse

\section{Krull-Remack-Schmidt}
This section is largely based on the book by \cite{CMR}.
Here we will prove the Krull-Remack-Schmidt theorem for complete local noetherian rings.

We say a ring satisfies Krull-Remack-Schmidt if the following condition holds:
\begin{itemize}
	\item[(i)] Any finitely generated module can be written as the finite direct sum of indecomposable modules.
	\item[(ii)] If $$\bigoplus_{i=1}^m M_i \cong \bigoplus_{j=1}^n N_j$$
	for indecomposable $M_i$'s and $N_j$'s, then $m=n$ and there is a permutation, $\sigma \in S_n$, such that $M_i \cong N_{\sigma(i)}$ for all $i=1,2, \cdots, n$.  
\end{itemize}

It's clear that (i) holds for any noetherian ring, since any decomposition of a noetherian module must eventually reach an indecomposable. In this chapter we will focus on proving (ii).

\fi

\section{Skew group algebra S\#G indecomposable projectives}
This section is largely based on the book by \cite{CMR}. This section will use definitions and theorems from representation theory as taught in the courses MA3203 - Ring Theory and MA3204 - homological algebra. Since I do not assume knowledge of this I have created appendix \ref{appendix}. I will try to use footnotes to indicate where such theorems are used.

\begin{defin}
If $G$ is a subgroup of $GL_n(\C)$, we can extend the group action of $G$ to $\C\llbracket x_1, \cdots, x_n\rrbracket$. More explicitely $G$ acts on $x_i$ as it would the $i$th basis vector of $\C^n$, and acts on products and sums by acting on each component seperatley. We then define the \underline{skew group algebra $\C \llbracket x_1, \cdots, x_n \rrbracket \# G$} to be the algebra generated by elements of the form $f \cdot g$ with $f \in \C\llbracket x_1, \cdots, x_n\rrbracket$ and $g \in G$, and we define the multiplication by
$$ (f_1 \cdot g_1) \cdot (f_2 \cdot g_2) = (f_1 \cdot f_2^{g_1}) \cdot (g_1 \cdot g_2) $$
Where $f^g$ denotes the image of $f$ under the action of $g$.
\end{defin}

\begin{theorem}
We have an isomorphism of rings
$$ e \C\llbracket x, y \rrbracket \# G e \simeq \C\llbracket x, y \rrbracket^G $$
where $e = \frac{1}{|G|} \sum_{g \in G} g$.

\begin{proof}
Let $f^g$ denote the image of $f$ under the action of $g$. Then if we let $f(x,y)g$ be an element of the skew algebra we get that $e f(x,y)g e = f(x, y)^e \cdot ege = f(x, y)^e \cdot e = e \cdot f(x, y)$. It then follows that $  e \C\llbracket x, y\rrbracket \# G e$ is isomorphic to the image of $\C\llbracket x, y \rrbracket$ under the action of $e$. Since $ge=g$ for all $g\in G$ it is clear that the image of $e$ is contained in the fixed ring. For the converse you just need to notice that the fixed ring is fixed under $e$ and thus is contained in the image.
\end{proof}
\end{theorem}

\begin{lemma}
\label{lem:S proj => SG proj}
Let $S = \C\llbracket x, y \rrbracket$. An $S\#G$-module is projective if and only if it is projective as an $S$-module.

\begin{proof}
Onlyifity follows from $S\#G$ being a free $S$-module, it is isomorphic to $\bigoplus_{g \in G} S$. Thus we need only show ifity.

First we need to see that an $S\#G$-linear map is just an $S$-linear map such that $f(g(m))=g(f(m))$ for all $g \in G$. Equivalently $f(m) = g(f(g^{-1}(m)))$. This allows us to define a group action on $S$-linear maps by $f^g(m) = g(f(g^{-1}(m)))$. Then we can restate it as $$ \Hom_{S\#G}(M,N) = \Hom_S(M,N)^G$$
Clearly if $f$ is $S\#G$-linear then it's in $\Hom_S(M,N)^G$. To see the other inclusion, let $f$ be an $S$-linear map that is fixed under $G$. Then $f(s\cdot g m) = s f(g m) = s\cdot g(f(g^{-1} g m)) = s \cdot g f(m) $, and hence $f$ is $S\#G$-linear. Nextly I want to show that $-^G$ is an exact functor.

If $K$ is the kernel of a map $f: M \to N$, then the kernel of the induced map $f^G : M^G \to N^G$ is of course just $K \cap M^G$ which equals $K^G$. Assume $f$ is epi and let $n \in N^G$. Consider a preimage $m$ such that $f(m)=n$. Let $\theta = \frac{1}{|G|}\sum_{g \in G} g(m)$. Then $\theta$ is in $M^G$ and $f(\theta) = \frac{1}{|G|}\sum_{g \in G} g(f(m)) = \frac{1}{|G|}\sum_{g \in G} n = n$.

This implies that if $\Hom_S(P, -)$ is exact then $\Hom_S(P, -)^G = \Hom_{S\#G}(P, -)$ is exact and our lemma follows.
\end{proof}
\end{lemma}

\begin{lemma}
\label{lem:radical small}
Let $S$ be the complex power series ring in $n$ variables, and $\mathfrak{m} = \langle x_i \rangle_{i=1}^n$ the radical of $S$. Then for any free $S$-module $N$, $\mathfrak{m}N$ is \underline{small} in $N$. That is if $X$ is  a submodule of $N$ such that $X + \mathfrak{m}N = N$, then $X = N$.

\begin{proof}
Let $N$ be the free module $S^{(I)} := \bigoplus\limits_{i \in I} S_i$, where $S_i \cong S$. Assume that $X$ is a submodule such that $X + \mathfrak{m}N = N$. We denote by $1_i$ the elements that is 1 at index $i$ and 0 elsewhere. Since $\{ 1_i \}$ generate $N$, it is enough to show that $X$ contains all of them. Since $X + \mathfrak{m}N = N$, we know that there is an $m_i \in \mathfrak{m}N$ and an $x_i \in X$ such that $x_i + m_i = 1_i$. Then we have that $x_i = 1_i - m_i$. Since the power series at index $i$ of $x_i$ has constant coefficient 1 it is invertible. If we multiply $x_i$ by its inverse we get $\tilde{x}_i$ which is 1 at index $i$ and some elemeent of $\mathfrak{m}$ at index $j \neq i$, say $m_{ij}$. Then $\tilde{x}_i - \sum\limits_{j \neq i} m_{ij}\tilde{x}_j$ has a unit in index $i$ and 0 at all other indicies. Thus $X$ contains $1_i$ for all $i$, and $X = N$.
\end{proof}
\end{lemma}

\begin{theorem}
Let $S = \C\llbracket x, y \rrbracket$ and let $\mathfrak{m} = \langle x, y \rangle_S$ be the radical of $S$. Then there are bijections between the indecomposable finitely generated projective $S\#G$-modules and the indecomposable $\C G$-modules given by

\begin{tikzcd}
\left\lbrace \begin{matrix}
\text{indecomposable projective}\\
S\#G\text{-modules}
\end{matrix} \right\rbrace 
& 
\left\lbrace \begin{matrix}
\text{indecomposable}\\
\C G\text{-modules}
\end{matrix} \right\rbrace\\
\mathcal{F}: P \arrow[mapsto]{r} & P/\mathfrak{m}P\\
\mathcal{G}: S \otimes_\C W & W \arrow[mapsto]{l}
\end{tikzcd}
\\
Where the $S\#G$-module structure on $S \otimes_\C W$ is given by $(s \cdot g) \cdot f \otimes v = sf^g \otimes v^g$.

\begin{proof}
First we should show that $S \otimes_\C W$ is an indecomposable projective $S\#G$-module and that $P/\mathfrak{m}P$ is infact an indecomposable $\C G$-module. Since $S \otimes_\C W$ is a free $S$-module it follows from \cref{lem:S proj => SG proj} that it is projective. To see that it is indecomposable we will first study it as an $S$-module and exploit the fact that $\Hom_{S\#G}(M,N) \subseteq \Hom_S(M,N)$. 

Using \cref{lem:radical small} we get that $\mathfrak{m}S \otimes_\C W$ is small in $S\otimes_\C W$. This means that we get that 
\begin{align*}
\frac{S \otimes_\C W}{\mathfrak{m}S \otimes_\C W} \cong S/\mathfrak{m} \otimes_\C W \cong \C \otimes_\C W \cong W
\end{align*}
is the top of $S \otimes_\C W$ as an $S$-module. Further since the projection $S \otimes_\C W \to W$ is $S\#G$-linear we have that $S \otimes_\C W$ is the projective cover of $W$ also as $S\#G$-modules. Assume for the sake of contradiction that $S \otimes_\C W$ decomposes as $M \oplus N$ for non-zero $M$ and $N$. Then $W$ would equal $M/\mathfrak{m}M \oplus N/\mathfrak{m}N$ as an $S\#G/ \langle \mathfrak{m} \rangle$-module. Since $S\#G/ \langle \mathfrak{m} \rangle \cong \C G$ and $W$ is indecomposable we must have that either $M/\mathfrak{m}M$ or $N/\mathfrak{m}N$ is 0. This then gives a contradiction because $\mathfrak{m}M$ and $\mathfrak{m}N$ are small in $M$ and $N$. Hence we must have that $S \otimes_\C W$ is indecomposable.

It's clear that $P/\mathfrak{m}P$ is a $\C G$-module, because $\C G$ is a subring of $S\#G$. To see that it's indecomposable we will first show that it's indecomposable we will use a similar argument as above. Assume $P/\mathfrak{m}P$ decomposes as $V \oplus W$. Then both $P$ and $S\otimes_\C V \oplus S \otimes_\C W$ are projective covers of $\mathfrak{m}P = V\oplus W$ we get induced $S\#G$-linear epiomorphisms between them.

\begin{center}
\begin{tikzcd}
& S \otimes_\C P/\mathfrak{m}P \arrow[twoheadrightarrow]{d} \arrow[dashed, two heads, bend right=20]{dl}\\
P \arrow[->>]{r} \arrow[dashed, two heads, bend right=5]{ur} & P/\mathfrak{m}P
\end{tikzcd}
\end{center}

Now we use the fact that $P$ is finitely generated. Since there can only be an epimorphism from a module with more or equal amount of generators $P$ and $S\otimes_\C \mathfrak{m}P$ must have the same amount of generators and teh induced maps are in fact isomorphisms of $S$-modules. Since the maps are also $S\#G$-linear we have that $P$ decomposes as $S\otimes_\C V \oplus S \otimes_\C W$. Then since $P$ is indecomposable we must have that either $S \otimes_\C V$ or $S\otimes\C W$ is 0. That means that either $V$ or $W$ is 0, and we have shown that $P/\mathfrak{m}P$ is an indecomposable $\C G$-module.

To see that the given maps are bijections we will show that they are mutual inverses. First to see that $\mathcal{F}(\mathcal{G}(W)) \cong W$ we simply look at the definition
\begin{equation*}
\begin{split}
\frac{S \otimes_\C W}{\mathfrak{m}S \otimes_\C W} \cong S/\mathfrak{m} \otimes_\C W \cong \C \otimes_\C W \cong W
\end{split}
\end{equation*}

Next we consider $\mathcal{G}(\mathcal{F}(P)) = S \otimes_\C P/\mathfrak{m}P$. We have already seen that the induced map
\begin{center}
\begin{tikzcd}
& S \otimes_\C P/\mathfrak{m}P \arrow[twoheadrightarrow]{d} \arrow[dashrightarrow]{dl}\\
P \arrow[->>]{r} & P/\mathfrak{m}P
\end{tikzcd}
\end{center}
is an isomorphism, and thus $P \cong \mathcal{G}(\mathcal{F}(P))$.
\end{proof}

\iffalse
Next we consider $\mathcal{G}(\mathcal{F}(P)) = S \otimes_\C P/\mathfrak{m}P$. We have already seen that it's projective. Both $P$ and $S \otimes_\C P/\mathfrak{m}P$ have a natural projection onto $P/\mathfrak{m}P$, and by projectivity we get an induced $S\#G$-linear map from $S \otimes_\C P/\mathfrak{m}P$ to $P$:
\begin{center}
\begin{tikzcd}
& S \otimes_\C P/\mathfrak{m}P \arrow[twoheadrightarrow]{d} \arrow[dashrightarrow]{dl}\\
P \arrow[->>]{r} & P/\mathfrak{m}P
\end{tikzcd}
\end{center}
Now by \cref{lem:radical small} we have that $\mathfrak{m}P$ is small and that therefor the induced map is an epimorphism. Similarly we get an epimorphism in the other direction. Since $S \otimes_\C P/\mathfrak{m}P$ has finite $S$-rank ($P$ finitely generated {\color{red}{Why is P finitely generated? This would be true if P/mP is indec CG-module, so if thats clear this should be clear}}), it follows that the map is an isomorphism of $S$-modules. Since the map is $S\#G$-linear it is tehrefor also an isomorphism of $S\#G$-modules.
\end{proof}


\begin{proof}
To see that this are bijections we will show that they are mutuall inverses. First to see that $\mathcal{F}(\mathcal{G}(W)) \cong W$ we simply look at the definition
\begin{equation*}
\begin{split}
\frac{S \otimes_\C W}{\mathfrak{m}S \otimes_\C W} \cong S/\mathfrak{m} \otimes_\C W \cong \C \otimes_\C W \cong W
\end{split}
\end{equation*}
Next we consider $\mathcal{G}(\mathcal{F}(P)) = S \otimes_\C P/\mathfrak{m}P$. Notice that the top of $ S \otimes_\C P/\mathfrak{m}P$ is isomorphic to $P/\mathfrak{m}P$. Then by the uniquness of tops we have that $ S \otimes_\C P/\mathfrak{m}P \cong P$.

The only thing that remains to show is that $\mathcal{F}$ and $\mathcal{G}$ are well-defined maps with the correct images. Namely that $\mathcal{F}(P)$ is an indecomposable $\C G$-module and that $\mathcal{G}(W)$ is an indecomposable projective $S\#G$-module.

Since $P$ is an indecomposable projective we have that $\mathcal{F}(P)$ is a simple $S\#G$-module. By the natural inclusion $\C G \hookrightarrow S\#G$ $\mathcal{F}(P)$ becomes a $\C G$-module. Assume that $\mathcal{F}(P)$ decomposes as $P_1 \oplus P_2$ as a $\C G$-module. Then since the action of $x$ and $y$ are trivial on $\mathcal{F}(P)$, $P_1 \oplus P_2$ is a decomposition of $S\#G$-modules. This implies that $P_1 = 0$ or $P_2=0$, and we have that $\mathcal{F}(P)$ is indecomposable.

Lastly we want to show that $\mathcal{G}(W)$ is projective and indecomposable. Since $S \otimes_\C W$ is free as an $S$-module it follows from \cref{lem:S proj => SG proj}, that it is a projective $S\#G$-module. To see that it is indecomposable, we just need to notice that its top, $S/\mathfrak{m} \otimes W \cong W$, is simple.
\end{proof}
\fi

\end{theorem}

\subsection{The Gabriel quiver}

\begin{defin}
For a skew group algebra $S\#G$ we define its \underline{Gabriel quiver} to be the quiver with verticies as the indecomposable projective modules of $S\#G$. The arrows are given by taking the minimal projective resolution of $P/\mathfrak{m}P$, where $\mathfrak{m}$ is as defined above. If the minimal projective resolution of $P/\mathfrak{m}P$ is given by
\begin{center}
\begin{tikzcd}
\cdots \arrow{r} & Q_1 \arrow{r} & Q_0 \arrow{r} & 0
\end{tikzcd}
\end{center}
We say there is an arrow from $P$ to $P'$ if $P'$ appears as a direct summand of $Q_1$.
\end{defin}

\begin{defin}
Let $V$ be a vector space. We then define the exterior algebra $\bigwedge V$ as the associative unital graded algebra such that the multiplication is bilinear and satisfies $x \wedge y = -y \wedge x$ for any $x$ and $y$ in $V$. 
\end{defin}
Some key properties of the exterior algebra is that $x \wedge x = 0$, and more generally that $x_1 \wedge \cdots \wedge x_p = 0$ whenever $\{x_i\}_{i=1}^p$ are linearly dependent.

The $p$th exterior power of $V$, denoted $\bigwedge\limits^p V$ is the vector space of all elements that are the product of $p$ vectors in $V$. If $\{ x_i \}_{i=1}^n$ is a basis for $V$, then $x_{i_1} \wedge \cdots \wedge x_{i_p}$ where $i_1 < i_2 < \cdots < i_p$ and $1 \leq i_j \leq n$ forms a basis for $\bigwedge\limits^p V$, thus it is ${n \choose p}$-dimensional.  

\begin{prop}
If $S$ is the ring of formal power series over $\C$ in $n$ variables, and $G$ is a finite group acting on $S$, let $V=\mathfrak{m}/\mathfrak{m}^2$. Then the minimal projective resolution of $\C \cong S/\mathfrak{m}$ is given by
\begin{center}
\begin{tikzcd}
0  \arrow{r} & S \otimes_\C \bigwedge\limits^n V  \arrow{r}{\partial_n} & \cdots \arrow{r}{\partial_2} & S \otimes_\C \bigwedge\limits^{1} V \arrow{r}{\partial_1} & S \arrow{r} & 0
\end{tikzcd}
\end{center}
Where $\partial_p$ is the $S\#G$-linear map defined by
\begin{align*}
\partial_p(s \otimes x_{i_1} \wedge x_{i_2} \wedge \cdots \wedge x_{i_p}) = \sum_{j=1}^{p} (-1)^{j+1} sx_{i_j} \otimes x_{i_1} \wedge \cdots \wedge \hat{x}_{i_{j}} \wedge \cdots \wedge x_{i_{p}} 
\end{align*}
Where $x_{i_1} \wedge x_{i_2} \wedge \cdots \wedge x_{i_p}$ is one of the standard basis vectors for $\bigwedge\limits^n V$, namely $i_1 < i_2 < \cdots < i_p$, and  $\hat{x}_j$ means that $x_j$ is ommited.

\begin{proof}
Firts we should show that this is a projective resolution of $\C$. In fact the complex described above is the Koszul complex of the regular sequence\footnote{Regular sequences are defined on page \pageref{def:regular_seq} in \cref{def:regular_seq}.} $(x_i)_{i=1}^n$. It's a general fact of homological algebra that the Koszul complex of a regular sequence is a projective resolution of the ring modulo the ideal generated by the regular sequence, which in this case equals $S/\langle x_i \rangle_{i=1}^n = \C$. {\color{red} refference yes/no?}
\iffalse
First we should show that this is a projective resolution. Note that since the maps are $S\#G$-linear, showing that it's a minimal free resolution as an $S$-module implies it is a minimal projective resolution as an $S\#G$-module. Then what we need to show is 
\begin{itemize}
\item[(i)] $\Cok \partial_1 = \C$
\item[(ii)] $\partial_{p-1} \circ \partial_{p} = 0$ for all $p$
\item[(iii)] $\Image \partial_{p+1} = \Ker \partial_p$ for $p \geq 2$
\end{itemize}
(i) is clear since the image of $\partial_1$ is $\mathfrak{m}$. (ii) can be shown through a quick computation
\begin{align*}
\partial_{p-1} \circ \partial_{p}(s \otimes x_{i_1} \wedge \cdots \wedge x_{i_p}) &=\\ 
\partial_{p-1} \left(\sum_{j=1}^p (-1)^{j+1} sx_{i_j} \otimes x_{i_1} \wedge \cdots \hat{x}_j \wedge \cdots \wedge x_{i_p} \right)&=\\
\sum_{j=1}^p (-1)^{j+1} \Bigg(\sum_{k=1}^{j-1} (-1)^{k+1} sx_{i_j}x_{i_k} \otimes x_{i_1}\wedge \cdots \hat{x}_{i_k} \wedge \cdots \wedge \cdots \hat{x}_j \wedge \cdots \wedge x_{i_p}& +\\
\sum_{k=j+1}^{p} (-1)^k sx_{i_j}x_{i_k} \otimes x_{i_1}\wedge \cdots \hat{x}_{i_j} \wedge \cdots \wedge \cdots \hat{x}_k \wedge \cdots \wedge x_{i_p}\Bigg) &
\end{align*}
From here we notice that the term with $j < k$ is canceled by the term where $k < j$, because they are the negatives of each other. Thus the composition is 0. This would then imply that $\Image \partial_{p+1} \subseteq \Ker \partial_p$, so for part (iii) we need only show that $\Ker \partial_p \subseteq \Image \partial_{p+1}$.

First some notation: let $\mathfrak{I}_p$ be the set of all tuples $(i_1, i_2, \cdots, i_p)$ with $i_1 < i_2 < \cdots < i_p$ and $1 \leq i_j \leq n$, and let $x_I$ denote $x_{i_1} \wedge \cdots x_{i_p}$ when $I=(i_1, \cdots, i_p)$. Then assume $$ \sum_{I \in \mathfrak{I}_p} s_I \otimes x_I$$ is in the kernel of $\partial_p$. 


Maybe just prove n=2, its simpler....
proof by induction on n on wikipedia %https://en.wikipedia.org/wiki/Koszul_complex#Properties_of_a_Koszul_homology
{\color{red}Maybe refferencing the proof will be better... or make an appendix on homological algebra}
\fi

Secondly we want to show that the resolution is minimal. To do this it is enough to show that for each $k \geq 1$ $\partial_k$ is a projective cover of its image, and that $S \to \C$ is a projective cover. In other words we have to show is that the kernels of the maps are small. Since $\Image \partial_{k+1} = \Ker \partial_k$ and $\Image \partial_{k+1} \subseteq \mathfrak{m} \otimes_\C \bigwedge\limits^{k+1}V$ it follows from \cref{lem:radical small} that the resolution is minimal.
\end{proof}
\end{prop}

\begin{theorem}
If $S$ is the complex power series ring in $n$ variables and $G$ is a fintie subgroup of $GL_n(\C)$, then the McKay quiver of $G$ and the Gabriel quiver of $S\#G$ are isomorphic.
\begin{proof}
We have already seen that they have the same vertices, namely if $V_i$ are the irreducible representations of $G$, then $S \otimes_\C V_i$ are the indecomposable projectives of $S\#G$. To see that they have the same arrows consider as above the minimal resolution of $\C$.
\begin{center}
\begin{tikzcd}
0  \arrow{r} & S \otimes_\C \bigwedge\limits^n V  \arrow{r}{\partial_n} & \cdots \arrow{r}{\partial_2} & S \otimes_\C \bigwedge\limits^{1} V \arrow{r}{\partial_1} & S \arrow{r} & 0
\end{tikzcd}
\end{center}
If we tensor with $V_i$ on the right we will get a minimal resolution of $V_i$ (you can see that is minimal by using the exact same argumant as above).
\begin{center}
\begin{tikzcd}
\cdots \arrow{r}{\partial_2 \otimes_\C V_i} & S \otimes_\C \bigwedge\limits^{1} V \otimes_\C V_i \arrow{r}{\partial_1 \otimes_\C V_i} & S \otimes_\C V_i \arrow{r} & 0
\end{tikzcd}
\end{center}
From here, since $\bigwedge\limits^{1} V = V$, we see that $P_j = S \otimes_\C V_j$ appears as a direct summand of $S \otimes_\C V \otimes_\C V_i$ exactly when $V_j$ appears as a direct summand of $V \otimes_\C V_i$.
\end{proof}
\end{theorem}

\section{The endomorphism ring of $S$ as an $S^G$-module}
This section is largely based on the article by \cite{IyTa} and the book by \cite{CMR}.

In this section we will show that $S\#G$ is isomorphic to $\End_R(S)$ as rings, where $R = S^G$ is the fixed ring of $S$ by $G$ (and we have some additional assumptions on $S$ and $G$). This will be the longest proof of this thesis and I have therefor decided to split it up into several steps. The proof will be done by constructing an explicit isomorphism.
\begin{center}
\begin{tikzcd}
S\#G \ar[r] & \End_R(S)\\
s \cdot g \ar[r, mapsto] & (t \mapsto s \cdot t^g)
\end{tikzcd}
\end{center}
We can easily show that this is an injective ring-homomorphism. The meat of the proof is to consider the map as a morphism of $S$-modules, and then using ramification theory to show that it is an epimorphism. To do this we will show that for every height one prime ideal $\mathfrak{p}$ of $S$ if we localize at $\mathfrak{p}$ we get a socalled unramified extension of rings.
\begin{center}
\begin{tikzcd}
R_{\mathfrak{p} \cap R} \ar[hook, r] & S_\mathfrak{p}
\end{tikzcd}
\end{center}
We will use this to show that the short exact sequence
\begin{center}
\begin{tikzcd}
I \ar[hook, r] & S_\mathfrak{p} \otimes_{R_{\mathfrak{p} \cap R}} S_\mathfrak{p} \ar[two heads, r]{}{\mu} & S_\mathfrak{p}
\end{tikzcd}
\end{center}
where $m$ is the multiplication map and $I$ is the kernel, has a splitting. Whenever this happens we say the extension is seperable. We will use this splitting to construct an inverse for $S_\mathfrak{p}\#G \to \End_{R_{\mathfrak{p} \cap R}}(S_\mathfrak{p})$. Finally we will show that since we get an isomorphism whenever we localize at a height one prime ideal this means that the original map is an isomorphism.

Let us first begin with some definitions
\begin{defin}
Let $A$ and $B$ be two local commutative rings with maximal ideal $\mathfrak{n}$ and $\mathfrak{m}$ respectively, and let 
\begin{tikzcd}
A \ar[hook, r] & B
\end{tikzcd}
be an extension of rings. We say that the extension is \underline{unramified} if the following conditions hold:
\begin{itemize}
\item $B$ is a finitely generated $A$-module.
\item \begin{tikzcd}
A/\mathfrak{n} \ar[hook, r] & B/\mathfrak{m}
\end{tikzcd}
is a seperable field extension.
\item $\mathfrak{n}B = \mathfrak{m}$
\end{itemize}
If the two first conditions are met, and there is a positive integer $e$ such that $\mathfrak{n} B = \mathfrak{m}^e B$, we say the extension has \underline{ramification index} $e$ when $e$ is the smallest such number. Note that being unramified is then equivalent to having ramification index 1. 
\end{defin}

In order to show that unramified implies seperable we must first take a small detour.

\begin{defin}
Let $A \to B$ be an extension of rings. We then define the \underline{derivation module} $\Omega_{B | A}$ as the $B$-module with forrmal generators $db$ for all $b \in B$ and with the following relations:
\begin{itemize}
\item[$A$-linearity:] $d(ab + a'b') = adb + a'db'$ for all $a, a' \in A$ and $b, b' \in B$.
\item[Leibniz rule:] $d(bc) = bdc + cdb$ for all $b,c \in B$.
\end{itemize} 
\end{defin}
Note that for any polynomial expression $f(b)$ we have that $df(b) = f'(b)db$ where $f'$ is the formal derivative of $f$. Now we will show how the derivation module make a link between unramified and seperable extensions.

\begin{prop}
Let $A \to B$ be an unramified extension of local rings. Then $\Omega_{B|A}$ is 0.

\begin{proof}
Keeping with the notation above we let $\mathfrak{n}$ be the maximal ideal of $A$ and $\mathfrak{m}$ the maximal ideal of $B$. Furthermore let $l$ denote $B/\mathfrak{m}$ and $k$ denote $A/\mathfrak{n}$. Then I claim there is an exact sequence
\begin{center}
\begin{tikzcd}
\mathfrak{m}/\mathfrak{m}^2 \ar[r]{}{\alpha} & \Omega_{B|A} \otimes_B B/\mathfrak{m} \ar[r] & \Omega_{l|A} \ar[r] & 0
\end{tikzcd}
\end{center}
where $\alpha(\overline{m}) = d_{B|A}m \otimes 1$ for any $m$ in $\mathfrak{m}$. Let's first show that $\alpha$ is well defined. Let $m_1 \cdot m_2$ be in $\mathfrak{m}^2$. Then we need to show that $\alpha(\overline{m_1 \cdot m_2})$ is 0. 
\begin{align*}
\alpha(\overline{m_1 \cdot m_2}) &= d_{B|A}(m_1 \cdot m_2) \otimes 1 =\\ 
m_1 d_{B|A}m_2 \otimes 1 &+ m_2d_{B|A}m_1 \otimes 1 =\\ 
d_{B|A}m_2 \otimes (m_1 \cdot 1) &+ d_{B|A}m_1 \otimes (m_2 \cdot 1)
\end{align*}
Since $l = B/\mathfrak{m}$ we have that $m_1 \cdot 1$ and $m_2 \cdot 1$ is 0 in $l$, thus the right hand side is 0, and $\alpha$ is well defined.

The map $\Omega_{B|A} \otimes_B B/\mathfrak{m} \to \Omega_{l|A}$ is just the natural projection sending $db \otimes 1$ to $d\overline{b}$, where $\overline{b}$ is teh projection of $b$ onto $l$. We want to show that this is the cokernel of $\alpha$. The kernel of $\Omega_{B|A} \otimes_B B/\mathfrak{m} \to \Omega_{l|A}$ is generated by $dm \otimes 1$ for $m \in \mathfrak{m}$, but this is exactly the image of $\alpha$, thus the sequence is exact.

Nextly we want to show that $\Omega{l|A} = 0$. Since $\mathfrak{n} \subseteq \mathfrak{m}$ and $l$ is annihalated by $\mathfrak{m}$ we have that $\Omega{l|A} = \Omega_{l|k}$. Let $x$ be an element of $l$, and let $p$ be its irreducible polynomial over $k$. Now we want to use the fact that $k \subset l$ is a seperable field extension. Remember that $k \subset l$ being seperable means that the formal derivative of $p$ is non-zero. Now we have that 
\begin{align*}
0 = d(p(x)) = p'(x)dx.
\end{align*}
Since $p'$ is a non-zero polynomial of lower degree than $p$, and $p$ is the smallest polynomial with root $x$, we must have that $p'(x)$ is non-zero. This implies that $dx=0$, and since this holds for all $x$ it must be that $\Omega{l|k}=0$.

Since $\Omega{l|k}=0$ we have that $\alpha$ is surjective. We will now use that since $A \to B$ is unramified $\mathfrak{n}B = \mathfrak{m}$. More specifically the map $\beta: \mathfrak{n}/\mathfrak{n}^2 \otimes_A B \to \mathfrak{m}/\mathfrak{m}^2$ is surjective. Since both $\alpha$ and $\beta$ are surjective we have that $\alpha\beta$ is also surjective, but 
\begin{align*}
\alpha\beta(\overline{n} \otimes b) = \alpha(\overline{nb}) = d(nb) \otimes 1 = ndb \otimes 1 = db \otimes n \cdot 1 = 0
\end{align*}
for all $n \in \mathfrak{n}$ and $b \in B$. Thus the only conclusion is that $\Omega_{B|A} \otimes_B l = 0$.

Since $\Omega_{B|A} \otimes_B l = \Omega_{B|A} \otimes_B B/\mathfrak{m} = \Omega_{B|A}/\mathfrak{m}\Omega_{B|A}$ it follows from Nakayamas lemma {\color{red} refference or something} that $\Omega_{B|A}=0$.
\end{proof}
\end{prop}

\begin{theorem}
Let $A \to B$ be an unramified extension of local rings. Then the sequence
\begin{center}
\begin{tikzcd}
0 \ar[r] & I \ar[r] & B \otimes_A B \ar{r}{\mu} & B \ar[r] & 0
\end{tikzcd}
\end{center}
splits as a short exact sequence of $B\otimes_AB$-modules. With $\mu(b \otimes b') = bb'$, and the $B \otimes_A B$-module structure on $B$ is given by $b \otimes b' \cdot b'' = bb'b''$, and $I = \Ker \mu$.

\begin{proof}
Firstly note that $I$ is generated by elements on the form $b \otimes 1 - 1 \otimes b$, and that since $B$ is fintiely generated as an $A$-module, $I$ is finitely generated.

Next we want to show that $I/I^2 = \Omega_{B|A}$, which we have already seen equals 0. Since $\Omega_{B|A}$ is a $B$-module we need a $B$-module structure on $I/I^2$. Since $(b \otimes 1)i - (1\otimes b)i$ is in $I^2$ for $i \in I$, we have that $(b \otimes 1)i = (1 \otimes b)i \mod I^2$. Then $I/I^2$ is generated by $(c \otimes 1 - 1 \otimes c)$ as a $B$-module with the $B$-module action given by $b \cdot i := (1 \otimes b)i$.

Now to see that $I/I^2 = \Omega_{B|A}$ we will show that the relations on $(b \otimes 1 - 1 \otimes b)$ in $I/I^2$ are exactly the same as those for $db$ in $\Omega_{B|A}$, such that $(db \mapsto (b \otimes 1 - 1 \otimes b)$ is an isomorphism.

$A$-linearity follows from the fact that we are tensoring over $A$, that is
\begin{align*}
(ab \otimes 1 - 1 \otimes ab) &= (b \otimes a - 1 \otimes ab) =\\ 
(1 \otimes a)(b \otimes 1 - 1 \otimes b) &= a \cdot 
(b \otimes 1 - 1 \otimes b)
\end{align*}

The Leibniz rule $dbc - bdc - cdb = 0$ follows from a similar computation.
\begin{align*}
& \;(b \otimes 1 - 1 \otimes b)(c \otimes 1 - 1 \otimes c) \\
=&\; bc \otimes 1 - b \otimes c - c \otimes b + 1 \otimes bc \\
=&\; (bc \otimes 1 - 1 \otimes bc) - (c \otimes b - 1 \otimes bc) - (b \otimes c - 1 \otimes bc)\\
=&\; (bc \otimes 1 - 1 \otimes bc) - (1 \otimes b)(c \otimes 1 - 1 \otimes c) - (1 \otimes c)(b \otimes 1 - 1 \otimes b)\\
\end{align*}
and we see that $(bc \otimes 1 - 1 \otimes bc) - b \cdot (c \otimes 1 - 1 \otimes c) - c \cdot (b \otimes 1 - 1 \otimes b)$ generates $I^2$.

Now that we have shown that $I/I^2 = \Omega_{B|A} = 0$, or rather that $I = I^2$ Nakaymas lemma {\color{red} refference} gives that there is an $i \in I$ such that $ji = j$ for all $j \in I$. Then we can define the splitting map $B \otimes_A B \to I$ by $b \otimes b' \mapsto b \otimes b' \cdot i$. Thus the sequence 
\begin{center}
\begin{tikzcd}
0 \ar[r] & I \ar[r] & B \otimes_A B \ar{r}{\mu} & B \ar[r] & 0
\end{tikzcd}
\end{center}
splits.
\end{proof}
\end{theorem}

\begin{theorem}
Let $B$ be a local $k$-algebra domain, and $G$ a finite subgroup of $\Aut_k(B)$ with order relatively prime to the characteristic of $k$, such that the extension $B^G =:A \to B$ is unramified. Then the map
\begin{center}
\begin{tikzcd}
B\#G \ar[r]{}{\gamma} & \End_A(B)\\
b \cdot g \ar[mapsto, r] & (a \mapsto b \cdot a^g)
\end{tikzcd}
\end{center}
is an isomorphism of $B$-modules, and isomorphism of rings.

\begin{proof}
First we to see that the map is injective, assume $b \cdot g$ and $b' \cdot g'$ map to the same endomorphism. Then $b \cdot t^g = b' \cdot t^{g'}$ for all $t \in B$. Choosing $t=1$ we see that $b = b'$. Then since $B$ is a domain this menas that $t^g = t^{g'}$ for all $t$, that is to say $g = g'$.

To see that the map is surjective we will construct a splitting. The splitting will be constructed using the following diagram:
\begin{center}
\begin{tikzcd}
B\#G \ar[r]{}{\gamma} & \End_A(B) \ar[d]{}{f \mapsto f \otimes \rho}\\
B \otimes_A B\#G \ar{u}{\tilde{\mu}} & \Hom_B(B \otimes_A B, B \otimes_A B\#G) \ar[l]{}{ev_\epsilon}
\end{tikzcd}
\end{center}
where $\rho$ is the modified Reinolds-opertaor
\begin{center}
\begin{tikzcd}
\rho(b) = \sum_{g \in G} b^g \cdot g.
\end{tikzcd}
\end{center}
Since we assumed the extension is unramified we have that
\begin{center}
\begin{tikzcd}
0 \ar[r] & I  \ar[r]{}{\iota} & B \otimes_A B \ar[r]{}{\mu} \ar[l, dotted, bend left = 30]{}{\psi} & B \ar[r] & 0
\end{tikzcd}
\end{center}
splits. As indicated we denote the left splitting by $\psi$. Then let $\epsilon = 1 \otimes 1 - \iota\psi(1 \otimes 1)$. Then $\mu(\epsilon) = 1$, and $(b \otimes 1 - 1 \otimes b)\epsilon = 0$. Then we define the evaluation map at $\epsilon$ by
\begin{center}
\begin{tikzcd}
ev_\epsilon: & \Hom_B(B \otimes_A B, B \otimes_A B\#G) \ar[r] & B \otimes_A B\#G\\
& f \ar[r, mapsto] & f(\epsilon)
\end{tikzcd}
\end{center}
Lastly $\tilde{\mu}: B \otimes_A B\#G \to B\#G$ is simply the map $b \otimes c \cdot g \mapsto bc \cdot g$. We have now defined all the maps in the square 
\begin{center}
\begin{tikzcd}
B\#G \ar[r]{}{\gamma} & \End_A(B) \ar[d]{}{f \mapsto f \otimes \rho}\\
B \otimes_A B\#G \ar{u}{\tilde{\mu}} & \Hom_B(B \otimes_A B, B \otimes_A B\#G) \ar[l]{}{ev_\epsilon}
\end{tikzcd}
\end{center}
Now we want to show that the composition of the three bottom maps forms a splitting. That is for any $f \in \End_A(B)$ we have that $\gamma(\tilde{\mu}(ev_\epsilon(f \otimes \rho))) = f$.

Write $\epsilon = \sum\limits_i x_i \otimes y_i$. Then I claim that 
\begin{align*}
\sum_i x_i \cdot y_i^g = \begin{cases}
1 & g = 1_G\\
0 & \text{otherwise}
\end{cases}
\end{align*}
We know that 
\begin{align*}
(b \otimes 1)\sum_i x_i \otimes y_i = (1 \otimes b)\sum_i x_i \otimes y_i
\end{align*}
holds for all $b$. Then applying the map $1 \otimes g$ on both sides we get
\begin{align*}
\sum_i bx_i \otimes y_i^g = \sum_i x_i \otimes b^gy_i^g
\end{align*}
Then by applying $\mu$ we get 
\begin{align*}
b\sum_i x_i y_i^g = b^g\sum_i x_i  y_i^g
\end{align*}
Then since $B$ is a domain we get that either $b = b^g$ or $\sum_i x_i  y_i^g = 0$. If we assume that $\sum_i x_i  y_i^g \neq 0$ we then get that $g = 1_G$. Then since
\begin{align*}
\sum_i x_i  y_i = \mu(\epsilon) = 1
\end{align*}
we see that my claim holds. We can now calculate $\gamma(\tilde{\mu}(ev_\epsilon(f \otimes \rho)))$:
\begin{align*}
\gamma \left[ \tilde{\mu} \left[ (f \otimes \rho)(\epsilon) \right] \right] (b) &=\\ 
\gamma \left[ \tilde{\mu} \left[ (f \otimes \rho)(\sum_i x_i \otimes y_i) \right] \right] (b) &=\\
\gamma \left[ \tilde{\mu} \left[ \sum_i f(x_i) \otimes \rho(y_i) \right] \right] (b) &=\\
\gamma \left[ \sum_i f(x_i) \sum_g y_i^g \cdot g \right] (b) &=\\
\gamma \left[ \sum_g \sum_i f(x_i) y_i^g \cdot g \right] (b) &=\\
\sum_g \left(\sum_i f(x_i) y_i^g \cdot b^g \right) &\stackrel{\mathclap{\normalfont\mbox{*}}}{=}\\
f \left( \sum_g \left(\sum_i x_i y_i^g \right) \cdot b^g \right) &\stackrel{\mathclap{\normalfont\mbox{**}}}{=}\\
f(b) &
\end{align*}
In (*) we use the fact that $f$ is $A$-linear and that $\sum_g \sum_i y_i^g b^g$ is in $A$. In (**) we use the claim from above that 
\begin{align*}
\sum_i x_i \cdot y_i^g = \begin{cases}
1 & g = 1_G\\
0 & \text{otherwise}
\end{cases}
\end{align*}
This means that $\gamma$ is an epimorphism and then also an isomorphism.
\end{proof}
\end{theorem}

\begin{defin}
Let $S$ be a commutative ring, $G$ a subgroup of $\Aut(S)$, and $\mathfrak{p}$ a prime ideal. The \underline{inertia group of $\mathfrak{p}$} is defined as
\begin{align*}
T(\mathfrak{p}) = \{ g \in G | s^g - s \in \mathfrak{p} \;\; \forall s \in S \}
\end{align*}
\end{defin}

\begin{theorem}
Let $S$ be the complex power series ring in $n$ variables, let $G$ be  a finite subgroup of $GL_n(\C)$ acting on $S$, and let $\mathfrak{p}$ be a height one prime ideal of $S$. Denote by $R$ the fixed ring $S^G$ and let $\mathfrak{q} = R \cap \mathfrak{p}$. Then the ramification index of $R_\mathfrak{q} \subset S_\mathfrak{p}$ equals the order of the inertia group $|T(\mathfrak{p})|$.

\begin{proof}
We write $\mathfrak{m}$ for the maximal ideal of $S$. Since $\mathfrak{p}$ is height one and $S$ is a UFD we have that $\mathfrak{p} = \langle z \rangle$ for some $z \in \mathfrak{m}$. We define an inner product on $V := \mathfrak{m}/\mathfrak{m}^2$ by
\begin{align*}
\langle x, y \rangle_G = \frac{1}{|G|} \sum_{g \in G} \langle x^g,  y^g \rangle
\end{align*}
\end{proof}
where $\langle -,- \rangle$ is the standard inner product. Note that the action of $G$ is orthogonal with respect to this inner product.

We write $\overline{z}$ for the representative for $z$ in $V$. Since the action of $G$ preserves degrees and that $\overline{z}^g - \overline{z} \in \langle \overline{z} \rangle$ we must have that $\overline{z}^g = a_g \cdot \overline{z}$ for some scalar $a_g \in \C$. Further since $x^g = x + \lambda_{g,x}\overline{z}$ for all $x \in V$ we have that $g$ fixes the $\langle -,- \rangle_G$-orthogonal complement to $\overline{z}$ for all $g \in T(\mathfrak{p})$. This means we can choose a basis such that all elements of $T(\mathfrak{p})$ are on the form:
\begin{align*}
\begin{pmatrix}
1\\
& 1\\
&& \ddots\\
&&&1\\
&&&& a_g
\end{pmatrix}
\end{align*}

This means $T(\mathfrak{p})$ is isomorphic to $\{ a_g \}_{g \in T(\mathfrak{p})} \leq \C^*$ which is a subgroup of $\C^*$. Since all finite subgroups of $\C^*$ are cyclic this implies that $T(\mathfrak{p})$ is cyclic. 

{\color{red} wait this proof only shows that $|T(\mathfrak{p})|$ is the ramification index of $S^{T(\mathfrak{p})}_\mathfrak{p} \subset S$...????? I want the ramification index for $S^G$, why are these the same???}

\end{theorem}

\begin{theorem}
\label{thm:unramified_pseudoreflections}
$R_\mathfrak{q} \subset S_\mathfrak{p}$ is unramified for all height one primes $\mathfrak{p}$ if and only if $G$ contains no pseudoreflections, that is a non-trivial element that fixes a codimension 1 subspace.
\begin{proof}
Firstly since we are working in characteristic 0, all field extensions are seperable, thus $R_\mathfrak{q}/\mathfrak{q} \subset S_\mathfrak{p}/\mathfrak{p}$ is seperable. Since $S$ is a rank $|G|$ $R$-module, $S_\mathfrak{p}$ will be a finitely generated $R_\mathfrak{q}$-module.

We know that elements of $T(\mathfrak{p})$ can be written on the form
\begin{align*}
\begin{pmatrix}
1\\
& 1\\
&& \ddots\\
&&&1\\
&&&& a_g
\end{pmatrix}.
\end{align*}
Since $G$ does not contain any pseudoreflections we must have that $a_g = 1$ and therefor $T(\mathfrak{p})$ is trivial and $|T(\mathfrak{p})| = 1$. That means that the ramification index of $R_\mathfrak{q} \subset S_\mathfrak{p}$ is 1, and the extension is unramified.
\end{proof}
\end{theorem}
Note that no finite subgroup of $SL_n(\C)$ contains pseduoreflections. In particular $R_\mathfrak{q} \subset S_\mathfrak{p}$ is unramified when $G$ is a finite subgroup of $SL_2(\C)$.

Now the last piece of the puzzle is to show that this implies that 
\begin{center}
\begin{tikzcd}
S \# G \ar[r]{}{\gamma} & \End_R(S)
\end{tikzcd}
\end{center}
is an isomorphism when $S = \C\llbracket x, y \rrbracket$, and $G$ is a finite subgroup of $SL_2(\C)$.

\begin{lemma}
\label{lem:height_one_iso}
Let $S$ be a local ring and let $M$ and $N$ be $S$-modules such that $depth M_\mathfrak{p} \geq \min \{ 2, height(\mathfrak{p}) \}$ and $depth M_\mathfrak{p} \geq \min \{ 1, height(\mathfrak{p}) \}$ for all prime ideals $\mathfrak{p}$\footnote{This is called Serre's criterion}. Let $f: M \to N$ be a monomorphism such that $f_\mathfrak{p}: M_\mathfrak{p} \to N_\mathfrak{p}$ is an epimorphism for all height one prime ideals. Then $f$ is an isomorphism.
\begin{proof}
Assume $f$ is not an epimorphism. Then $f$ has a cokernel $C \neq 0$, and we have a short exact sequence
\begin{center}
\begin{tikzcd}
0 \ar[r] & M \ar[r]{}{f} & N \ar[r] & C \ar[r] & 0
\end{tikzcd}
\end{center}
Now we choose $\mathfrak{p}$ to be the annihalator of a submodule $\langle c \rangle$ for some non-zero $c \in C$. We want to show that $\mathfrak{p}$ has height at least 2. If $\mathfrak{p}$ had height one then since $f_\mathfrak{p}$ is epi we would have that $C_\mathfrak{p} = 0$. This equivalent to saying that for every $c \in C$ there is some element $s \not\in \mathfrak{p}$ such that $sc = 0$. This is impossible since $\mathfrak{p}$ is the anniholator for some $c$, thus if $sc=0$ then $s$ is in $\mathfrak{p}$. The same argument works for a height 0 prime ideal since they are contained in height one prime ideals.

Thus $\mathfrak{p}$ has heigth at least 2 and $depth M_\mathfrak{p} \geq 2$, $depth N_\mathfrak{p} \geq 1$. Now we want to show that $C_\mathfrak{p}$ has depth 0, using regular sequences. Recall that the depth of a module is the length of the longest regular sequence {\color{red} I need some reorganizing of when I define depth/how}. Since $\mathfrak{p}$ annihalates some $c \in C$ multiplication by $p \in \mathfrak{p}$ cannot be injective on $C_\mathfrak{p}$, because $\frac{c}{1}$ will be in the kernel. Multiplication by any element not in $\mathfrak{p}$ will be epimorphic since $s \cdot \frac{c}{s \cdot t} = \frac{c}{t}$, thus no regular sequence exist on $C_\mathfrak{p}$.

Now we consider the short exact sequence
\begin{center}
\begin{tikzcd}
0 \ar[r] & M_\mathfrak{p} \ar[r]{}{f_\mathfrak{p}} & N_\mathfrak{p} \ar[r] & C_\mathfrak{p} \ar[r] & 0
\end{tikzcd}
\end{center}
and take its long exact sequence of $\Ext_S(k, -)$ where $k$ is the residual field of $S$.
\begin{center}
\begin{tikzcd}
\cdots \ar[r] & \Hom_S(k, N_\mathfrak{p}) \ar[r] & \Hom_S(k, C_\mathfrak{p}) \ar[r] & \Ext^1_S(k, M_\mathfrak{p}) \ar[r] & \cdots
\end{tikzcd}
\end{center}
Since $depth N_\mathfrak{p} \geq 1$ and $depth M_\mathfrak{p} \geq 2$ we have that $\Hom_S(k, N_\mathfrak{p})$ and $\Ext^1_S(k, M_\mathfrak{p})$ is 0. Then by exactness we get that $\Hom_S(k, C_\mathfrak{p}) = 0$. This contradicts the fact that $depth C_\mathfrak{p} = 0$, and thus our assumption that $C \neq 0$ is wrong. Therefor $f$ is an epimorphism and therefor also an isomorphism.
\end{proof}
\end{lemma}

\begin{theorem}
Let $S = \C \llbracket x, y \rrbracket$ be the complex power series ring in two variables, let $G$ be  a fintie subgroup of $SL_2(\C)$ acting on $S$, and let $R = S^G$ be the fixed ring. Then the map
\begin{center}
\begin{tikzcd}
S \# G \ar[r]{}{\gamma} & \End_R(S)
\end{tikzcd}
\end{center}
is an isomorphism of rings.
\begin{proof}
Firstly let's show that for a prime ideal $\mathfrak{p}$ we have that $S_\mathfrak{p}^G = R_\mathfrak{q}$ where $\mathfrak{q} = R \cap \mathfrak{p}$. Assume $\frac{s}{p} \in S_\mathfrak{p}^G$ is fixed by $G$. Consider the fraction
\begin{align*}
\frac{ \left( \prod_{g \neq 1} p^g \right) s}{\prod_g p^g}
\end{align*}
Since we have just multiplied by $\prod_{g \neq 1} p^g$ in the nominator and the denominator it still equals $\frac{s}{p}$. The bottom is obviously fixed by $g$, but why is it not in $\mathfrak{q}$??? 
{\color{red} How do I know $p^g$ is not in $\mathfrak{p}$ ???? You localize using the complement of the prime ideal right??} Then since the denominator is fixed and the fraction as a whole is fixed this implies that the nominator is fixed as well.

Secondly I want to show that $\End_R(S)_\mathfrak{p} = \End_{R_\mathfrak{q}}(S_\mathfrak{p})$. {\color{red} ???}

From here we just need to wrap everything together. Since $G$ does not contain any pseudoreflections we get from \cref{thm:unramified_pseudoreflections} that the map is an isomoprhism when localizing at any height one prime ideal. Then \cref{lem:height_one_iso} gives us that it's an isomophism {\color{red} need to show depth}.
\end{proof}
\end{theorem}

\begin{theorem}
Let $S$ be the complex power series ring in two variables, $G$ be a finite subgroup of $GL_2(\C)$, $R = S^G$ the fixed ring of $S$ under the action of $G$, and $(S\#G)^G$ be the fixed ring of $S\#G$ under left multiplication by $G$. Then $S$ is isomorphic to $(S\#G)^G$ as $R$-modules.

\begin{proof}
To see this we will define an injective $R$-linear map from $S$ to $S\#G$ and show that it's image is $(S\#G)^G$. Let $\rho: S \to S\#G$ be given by
\begin{align*}
\rho(s) = \sum_{g \in G} s^g \cdot g.
\end{align*}
It's clear that it's injective and it is $R$-linear because $$\rho(rs) = \sum_{g \in G} r^gs^g \cdot g = r\sum_{g \in G} s^g \cdot g.$$ It should also be clear that the image is contained in $(S\#G)^G$ because
\begin{align*}
h \cdot \rho(s) = \sum_{g \in G} h \cdot s^g \cdot g = \sum_{g \in G} s^{hg} \cdot hg = \rho(s).
\end{align*}
To see that the image is all of $(S\#G)^G$ consider an arbitrary element in $(S\#G)^G$, $\psi = \sum_{g\in G} s_g \cdot g$. Since $\psi$ is fixed under left multiplication by $G$ we must have that
\begin{align*}
\sum_{g\in G} s_g^h \cdot hg = \sum_{g\in G} s_g \cdot g,
\end{align*}
in particular $s_h$ must equal $s_1^h$ and it follows that $\psi = \rho(s_1)$.
\end{proof}
\end{theorem}

\section{Maximal Cohen-Macaulay modules of $S^G$}
\begin{defin}
If $R$ is a local ring with residual field $k$ we define the \underline{depth} of a module, $M$, to be the minimal $n$ such that $\Ext^n_R(k, M)$ is non-zero.
\end{defin}

\begin{defin}
\label{def:regular_seq}
If $R$ is a commutative ring and $M$ is an $R$-module, an \underline{$R$-regular sequence on $M$} is a sequence of elements of $R$, $r_1, r_2, \cdots r_n$ such that $M/\langle r_1, \cdots, r_i \rangle M$ is non-zero and multiplication by $r_i$ is injective on $M/\langle r_1, \cdots, r_{i-1} \rangle M$.
\end{defin}

\begin{prop}
The depth of a module equals the lenth of the longest regular sequence on that module.
\begin{proof}
{\color{red} reffrence}
\end{proof}
\end{prop}

\begin{defin}
If $R$ is a ring, we say that $\mathfrak{p}$ is  a \underline{prime ideal} in $R$ if
\begin{enumerate}
\item $\mathfrak{p}$ is a proper ideal of $R$.
\item For any two elements $a,b \in R$ such that $ab \in \mathfrak{p}$ we must have that either $a$ is in $\mathfrak{p}$ or $b$ is.
\end{enumerate}
\end{defin}

\begin{defin}
If $R$ is a ring we define its \underline{Krull-dimesnion} to be the maximum length of a chain of prime ideals in $R$. For example the polynomial ring $\C[x_1, \cdots, x_n]$ has Krull-dimension $n$ given by the chain
\begin{equation*}
\begin{tikzcd}
0 \subseteq \langle x_1 \rangle \subseteq \langle x_1, x_2 \rangle \subseteq \cdots \subseteq \langle x_1, \cdots, x_n \rangle
\end{tikzcd}
\end{equation*}
\end{defin}

\begin{defin}
If $M$ is a module over a local ring $R$ with Krull-dimension $d$ we say that $M$ is \underline{maximal Cohen Macaulay (MCM)} if the depth of $M$ equals $d$.
\end{defin}

\begin{theorem}
If $G$ is a finite subgroup of $GL_n(\C)$, $S$ is the complex power series ring in $n$ variables and $R = S^G$ is the ring fixed under the action of $G$, then $R$ is a direct summand of $S$ as $R$-modules.

\begin{proof}
Consider the map $\pi: S \to R$ given by
\begin{align*}
\pi(s) = \frac{1}{|G|} \sum_{g\in G} s^g
\end{align*}
It's clear that the image of $\pi$ is in $R$ because an action from $G$ will just permute the order of the sum. Further \begin{align*}
\pi(r) = \frac{1}{|G|} \sum_{g\in G} r^g = \frac{1}{|G|} \sum_{g\in G} r = r,
\end{align*}
so $\pi$ splits the inclusion $R \hookrightarrow S$ which shows that $R$ is a direct summand of $S$.
\end{proof}
\end{theorem}

\begin{defin}
Let $R$ be a {\color{red}local?} ring, $M$ an $R$-module, and $(x_i)_{i=1}^n$ an $R$-regular sequence on $M$. Let $V$ denote the free abelian group with formal generators $\{ x_i \}_{i=1}^n$. The \underline{Koszul complex} of the sequence is then defined to be
\begin{equation*}
\begin{tikzcd}
0 \arrow[r] & M \otimes_\Z \bigwedge^n V \arrow[r]{}{\partial_n} & M \otimes_\Z \bigwedge^{n-1} V \arrow[r]{}{\partial_{n-1}} & \cdots \\ 
\cdots \arrow[r]{}{\partial_2} & M \otimes_\Z \bigwedge^1 V \arrow[r]{}{\partial_1} & M \arrow[r] & 0.
\end{tikzcd}
\end{equation*}
\end{defin}

\iffalse
\begin{theorem}
The Koszul complex of a regular sequence is exact everywhere except in degree 0.
\begin{proof}
The proof will be inductive on the length of the regular sequence. Let's first show that it holds when the length of the sequence is 1. Let $M$ be our module, $(x_1)$ our regular sequence, and $V = \Z x_1$. Then we get the following complex
\begin{equation*}
\begin{tikzcd}
0 \arrow[r] & M \otimes_\Z \bigwedge^1 V \arrow{r}{\partial_1} & M \arrow[r] & 0
\end{tikzcd}
\end{equation*}
This is isomorphic to 
\begin{equation*}
\begin{tikzcd}
0 \arrow[r] & M\arrow{r}{x_1 \cdot -} & M \arrow[r] & 0
\end{tikzcd}
\end{equation*}
and since multiplication by $x_1$ is injective on $M$, this is exact in degree 1. Since it is 0 in all degrees except 0 and 1, it is exact there as well. 

For the inductive step we will show a relationship between regular sequences of different lengths. Let $(x_i)_{i=1}^{n}$ be a regular sequence on $M$, and let $K(s)$ be the Koszul complex of the sequnece $(x_i)_{i=1}^s$. Then there is an exact sequence of complexes
\begin{equation*}
\begin{tikzcd}
0 \arrow[r] & K(s) \arrow{r}{\iota} & K(s+1) \arrow{r}{\pi} & K(s)[-1] \arrow{r} & 0
\end{tikzcd}
\end{equation*}
where $K(s)[-1]$ is $K(s)$ shifted one spot to the left. The short exacts equence is given by
\begin{equation*}
\begin{tikzcd}
\cdots \arrow{r}{\partial_3^s} & 
M \otimes_\Z \bigwedge^2 V_s \arrow{r}{\partial_2^s} \arrow{d}{\iota_2} & 
M \otimes_\Z \bigwedge^1 V_s \arrow{r}{\partial_1^s} \arrow{d}{\iota_1} & 
M \arrow[r] \arrow[d, equal] & 0 \arrow[d]\\
\cdots \arrow{r}{\partial_3^{s+1}} & 
M \otimes_\Z \bigwedge^2 V_{s+1}\arrow{r}{\partial_2^{s+1}} \arrow{d}{\pi_2} & 
M \otimes_\Z \bigwedge^1 V_{s+1} \arrow{r}{\partial_1^{s+1}} \arrow{d}{\pi_1} & 
M \arrow[r] \arrow[d] &  0 \arrow[d]\\
\cdots \arrow{r}{\partial_2^s} & 
M \otimes_\Z \bigwedge^1 V_s \arrow{r}{\partial_1^s} & 
M \arrow[r] & 0 \arrow[r] & 0
\end{tikzcd}
\end{equation*}
where $\iota$ is just the inclusion and $\pi_{j+1}(m \otimes x_{i_1} \wedge \cdots \wedge x_{i_{j}} \wedge x_{s+1}) = m \otimes x_{i_1} \wedge \cdots \wedge x_{i_{j}}$, and 0 if $x_{s+1}$ does not appear in the wedge product.

This short exact sequence induces a longe exact sequence on homology
\begin{center}
\begin{tikzcd}
\cdots \ar[r] & H_2(K(s+1)) \ar[r] & H_1(K(s)) \ar[r] & H_1(K(s)) \ar[dlll, overlay, in=170, out=-10] \\
H_1(K(s+1)) \ar[r] & H_0(K(s)) \ar[r] & H_0(K(s)) \ar[r] & H_0(K(s+1)) \ar[r] & 0
\end{tikzcd}
\end{center}
Now we assume by induction that $K(s)$ is exact in all non-zero degrees. Then we get the following long exact sequence
\begin{center}
\begin{tikzcd}
\cdots \ar[r] & H_2(K(s+1)) \ar[r] & 0 \ar[r] & 0 \ar[dlll, overlay, in=170, out=-10] \\
H_1(K(s+1)) \ar[r] & M/I_sM \ar[r]{}{x_{s+1} \cdot -} & M/I_s \ar[r] & M/I_{s+1}M \ar[r] & 0
\end{tikzcd}
\end{center}
where $I_k = \langle x_i \rangle_{i=1}^k$. Then since the sequence is exact we get that $H_i(K(s+1)) = 0$ whenever $i \geq 2$, and that $H_1(K(s+1))$ is isomorphic to the kernel of multiplication by $x_{s+1}$. Here we use the fact that the sequence is regular, which means that multiplication by $x_{s+1}$ is injective on $M/I_sM$ and therefor $H_1(K(s+1)) = 0$.

Then by induction we have shown that $H_i(K(n)) = 0$ for all $i \neq 0$, or in other words that the Koszul complex is exact in all degrees except 0.
\end{proof}
\end{theorem}
\fi

\begin{prop}
{\color{red} Ext on direct sum + depth < dim}
Let $R$ be a local ring with depth of $R$ equaling it's Krull-dimension. If $M$ is an MCM $R$-module, and $N$ is a direct summand of $M$ then $N$ is also MCM.
\begin{proof}
We write $M$ as $N \oplus X$. Since $M$ is MCM we have that $0 = \Ext^i_R(M) = \Ext^i_R(N) \oplus \Ext^i_R(X)$ for all $i$ less than the Krull-dimension of $R$. This means the depth of $N$ is greater than or equal to the Krull-dimension of $R$. Since the depth of a module cannot exceed the krull-dimension of the ring {\color{red} refference} we have that $N$ is MCM.
\end{proof}
\end{prop}

In this section we will use the fact that $S$ and $R$ have the same krull dimension. This can be shown in general using some tools from algebraic geometry, but in the special case when $G$ is a finite subgroup of $SL_2(\C)$ we have that $R = \C \llbracket u, v, w \rrbracket / \langle f \rangle$ for some irreducible polynomial $f$. Therefor $R$ has dimension 2, just like $S$. The proof of this is uses that up to a change of basis there are only five families of finite subgroups of $SL_2(\C)$, a survey of which can be found in [\href{https://staff.fnwi.uva.nl/r.r.j.bocklandt/notes/kleinian.pdf}{kleinian}] and [Carrasco project]. Here I will simply list the groups and the formulas for $R$.

\begin{center}
{\renewcommand{\arraystretch}{1.6}
\begin{tabular}{|c|c|c|}
\hline
McKay quiver & $G$ & $R = S^G$\\
\hline\hline
$A_n$ & $\mathbb{Z}/n\mathbb{Z}$ & 
$\C \llbracket u, v, w \rrbracket/\langle uv - w^n \rangle$
\\
\hline
$D_n$ & $BD_{4n}$ & 
$\C \llbracket u, v, w \rrbracket/\langle u^{n+1} + v^2 - uw^2\rangle$
\\
\hline
$E_6$ & $BT_24$ & 
$\C \llbracket u, v, w \rrbracket/\langle u^4 + v^3 + w^2 \rangle$
\\
\hline
$E_7$ & $BO_{48}$ &
$\C \llbracket u, v, w \rrbracket/\langle u^3v + v^3 + w^2 \rangle$
\\
\hline
$E_8$ & $BI_{120}$ &
$\C \llbracket u, v, w \rrbracket/\langle u^5 + v^3 + w^2 \rangle$
\\
\hline
\end{tabular}
}
\end{center}

Below is a table of how the groups are realized in $SL_2(\C)$. We write $\zeta$ for the primitive fifth root of unity $\exp(2\pi i/5)$

{\renewcommand{\arraystretch}{2}
\begin{tabular}{|c|c|}
\hline
$G$ & generators in $SL_2(\C)$
\\
\hline
\hline
$\mathbb{Z}/n\mathbb{Z}$ & $\begin{pmatrix}
\exp(2\pi i/n) & 0\\
0 & \exp(-2\pi i/n)
\end{pmatrix} $\\
\hline
$BD_{4n}$ &  $ \begin{pmatrix}
\exp(\pi i/n) & 0\\
0 & \exp(-\pi i/n)
\end{pmatrix}, \begin{pmatrix}
0 & i\\
i & 0
\end{pmatrix} $\\
\hline
$BT_{24}$ & $ \begin{pmatrix}
\frac{i+1}{2} & -\frac{i+1}{2}\\
\frac{-i+1}{2} & \frac{-i+1}{2}
\end{pmatrix}, \begin{pmatrix}
\frac{i+1}{2} & \frac{i+1}{2}\\
-\frac{-i+1}{2} & \frac{-i+1}{2}
\end{pmatrix} $\\
\hline
$BO_{48}$ & $BT_{24}$, $\begin{pmatrix}
\frac{1+i}{\sqrt{2}} & 0\\
0 & \frac{1-i}{\sqrt{2}}
\end{pmatrix}$\\
\hline
$BI_{120}$ & $ \begin{pmatrix}
0 & 1\\
-1 & 0
\end{pmatrix}, \begin{pmatrix}
\zeta^3 & 0\\
0 & \zeta^2
\end{pmatrix}, \frac{1}{\sqrt{5}}\begin{pmatrix}
-\zeta + \zeta^4 & \zeta^2 - \zeta^3\\
\zeta^2 - \zeta^3 & \zeta - \zeta^4
\end{pmatrix}$
\\
\hline
\end{tabular}
}

\begin{lemma}
\label{lem:depth_of_S_less_than_R}
If $S$ is a local ring, $G$ a finite subgroup of $\Aut(S)$, and $R=S^G$ is the fixed ring, then $depth_SS \leq depth_RR$.
\begin{proof}
{\color{red} Exercise 5.30 page 78, I have to remember how we did this.}
\end{proof}
\end{lemma}

\begin{theorem}
Let $S$ be the complex power series ring in two variables, $G$ a finite subgroup of $SL_2(\C)$ acting on $S$ by linear change of variables, and $R=S^G$ the fixed ring. Then $S$ is an MCM $R$-module.
\begin{proof}
Since $S$ is the complex power series ring in two variables we have that $dim S = depth_SS=2$. By \cref{lem:depth_of_S_less_than_R} we have that $depth_SS \leq depth_RR$. Since $R \subset S$, any $R$-regular sequence on $R$ is also an $S$-regular sequence. Therefor we have that $depth_RR \leq depth_SR$. Since the depth of a module never exceeds the krull dimension of the ring we have that $depth_SR \leq dim S$. Lastly we have seen that $dim R = dim S$. Chaining this all together we get
\begin{align*}
dim R = dim S = depth_SS \leq depth_RR \leq depth_SR \leq dim S
\end{align*}
and thus $R$ is CM, but why is S MCM {\color{red} ????}
\end{proof}
\end{theorem}

\begin{appendices}
\section{Representation theory}
\label{appendix}

\begin{defin}
If $R$ is a ring and $M$ is an abelian group, we define a \underline{representation of $R$} to be a ring-map, $\varphi$, from $R$ to $\End(M)$. We say that $M$ is a (left) $R$-module, and we write $rm$ with $r \in R$ and $m \in M$ to mean $\varphi(r)(m)$. Similarly we define a right $R$-module if $\varphi$ goes from $R$ to $\End(M)^{op}$ and we write $mr$ for $\varphi(r)(m)$.
\end{defin}

\begin{defin}
If $G$ is a group and $V$ a complex vectorspace, we define a \underline{representation of $G$} to be a group-map, $\rho$, from $G$ to $\Aut_\C(V)$. When $\rho$ is infered we say that $V$ is a representation of $G$ and we write $gv$ to mean $\rho(g)(v)$. Note that representations of $G$ exactly corresponds to representations of the ring $\C G$ of formal linear combinations of elements of $G$ with multiplication given by $\lambda g \cdot \lambda' g' = (\lambda \cdot \lambda')gg'$.
\end{defin}

\begin{defin}
If $R$ is a ring and $M_1$ and $M_2$ are two modules we define their \underline{direct sum}, $M_1 \oplus M_2$ to be the module consisting of all pairs $(m_1, m_2)$ (usually written $m_1 + m_2$), where addition and scalar multiplication is pointwise. If a non-zero module cannot be written as the direct sum of two non-zero modules we call it \underline{indecomposable}.
\end{defin}

\begin{defin}
A \underline{submodule} is a subset of a module which is also a module. A non-zero module with no non-trivial proper submodules is called \underline{simple} or \underline{irreducible}\footnote{The word simple is used for representations of rings while irreducible is used for representations of groups. Note that for finite groups irreducible and indecomposable are equivalent.}.
\end{defin}

\begin{theorem}
\label{schur}
(Schur's Lemma) Let $G$ be a group and $V$ and $W$ be two irreducible representations of $G$. If $f:V \to W$ is a $G$-linear map then $f$ is a 0 if $V$ and $W$ are not isomorphic, and a scaling of identity (up to change of basis) if they are isomorphic.
\begin{proof}
Start by assuming $f$ is non-zero. Then we will show that $V$ and $W$ are isomorphic. Since the image of $f$ is a non-zero subrepresentation of $W$ and $W$ is irreducible, we have that $\Image f = W$ and $f$ is surjective. Sicne the kernel of $f$ is a proper subrepresentation of $V$ we must have that the kernel is 0, and that $f$ is injective. Thus $f$ is an isomorphism.
Now assume $f: V \to V$ is a $G$-linear map. then we want to show that $f$ is simply a scaling of identity. Since $f$ is a linear map on a complex vector space it must have at least one eigen value, say $\lambda \in \C$. Let $v$ be in the eigenspace $\lambda$. Sicne $f(gv) = g f(v) = \lambda gv$ for all $g$ in $G$ we have that $gv$ is also in the eigenspace. This means the eigenspace is a subrepresentation, and since $V$ is irreducible it must equal all of $V$. This means that $f$ is just scaling by $\lambda$.
\end{proof}
\end{theorem}

\begin{defin}
We call a functor \underline{left exact} if for any short exact sequence
\begin{center}
\begin{tikzcd}
0 \arrow{r} & A \arrow{r}{f} & B \arrow{r}{g} & C
\end{tikzcd}
\end{center}
the image of the sequence under the functor is also exact. For example for any module $M$ the functor $\Hom(M, -)$ is left exact. That is the sequnce
\begin{center}
\begin{tikzcd}
0 \arrow{r} & \Hom(M,A) \arrow{r}{f \circ -} & \Hom(M,B) \arrow{r}{g \circ -} & C
\end{tikzcd}
\end{center}
is exact. Dually we call a functor \underline{right exact} if short exact sequnces of the form
\begin{center}
\begin{tikzcd}
A \arrow{r} & B \arrow{r} & C \arrow{r} & 0
\end{tikzcd}
\end{center}
is mapped to an exact sequence. A functor that is both left exact and right exact is called \underline{exact}.
\end{defin}

\begin{defin}
We say that a module, $P$, is \underline{projective} if for any epimorphism $f: M \twoheadrightarrow N$, and any map $g: P \to N$, there is a map $\varphi: P \to M$ such that $f \varphi = g$. Said another way, the diagram below induces the dotted arrow making the diagram commute
\begin{center}
\begin{tikzcd}
& P \arrow{d}{g} \arrow[dl, dotted, swap]{}{\varphi} \\
M \arrow[twoheadrightarrow, swap]{r}{f} & N
\end{tikzcd}
\end{center} 
Note that $P$ being projective is equivalent to $\Hom(P, -)$ being right exact (i.e. exact).
\end{defin}

Projective cover + radical is  small for noetherian modules

indec proj = summand of ring = idempotent of ring

projective resolution

Ext + Tor
\end{appendices}

\section{Random Thoughts I need to figure out}
I R=S then they have the same depth meaning M is cohen macaulay iff it's projective dimension is 0 (Auslander-Buschsbauw), but that means its a direct summand of S as S(=R)-module, which make sense. If I can show that R = S/(f) fro some polynomial f, can I show R-direct summands of S have projective dimension 1 over S? Can I show R has depth depth(S)-1? Then I also need to prove Auslander-Buschsbauw... Need to show some relation between dimension and depth.

If P is indec finitely generated projective then it is direct summand of $S^n$, then P must either be a direct summand of S or $S^n-1$ then by induction P is a  summand of S. Can I assume P to be finitely generated?????

$P/mP = sum V_i -> P = sum SV_i$, means  all projective $S\#G$-modules can be broken down into sums.

\section{questions}
Why does $0 \to J \to S \otimes_R S \to S \to 0$ split?

Is it true that $dim R \leq dim S$ (this is not true for $\mathbb{Z} \subset \mathbb{Q}$), alternatively how to show that $S$ is MCM?

Direct summand of MCM is MCM?  Use Koszul complex + Ext preserves direct sums.

Are indec proj $S\#G$-modules fin.gen.? Can I state the correspondance in terms of fin.gen, indec projctives instead?

\section{Disposisjon}
Define McKay quiver [check]

Define $S\#G$ [check]

Correspondance with projectives [put in finitely generated to fix argument]

Gabriel Quiver [Make Koszul complex a refference]

$\End_R(S) \cong S\#G$ [understand the proof]
$q \in S$ height one prime implies $q = (f)$ for a homogenous polynomial? Why homogenous?
If $T(q)$ is nontrivial then it acts non-trivially on $S/qS$ if $f$ has degree bigger than 1, then all degree 1 polynomials survive in $S/qS$ and are acted upon trivialy by $T(q)$. Therefor $T(q)$ would be trivial, so $f$ is homogenous of degree 1.
Since the group operations preseve degree $\sigma(f) = a_\sigma f$ for a nonzero constant $a_\sigma$. All finite matrix groups diagonalizeable implies $\sigma = diag(1,1, \cdots a_\sigma)$. Therefor $T(q)$ is iso to finite subgroup of $\C^*$, hence cyclic.
Then $p=q \cap R = (f^n)$ where $n$ is the order of $T(q)$. Thus $q = pS_q$ if and only if $T(q)$ is trivial.
$I/I^2 = \Omega_{S|R}$ is 0 iff $pS = q$, $I/I^2 = 0$ implies idempotent implies spliting.
Why is $\End_R(S)$ reflexive? or rather why does height one iso imply iso.

MCM $R$-summands of $S$ 
[$S$ is MCM using dimension argument, summands are MCM using depth $\leq$ dim]

\nocite{*}
\bibliography{mybib}
\bibliographystyle{apalike}

\end{document}